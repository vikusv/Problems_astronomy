\begin{problem}{IOAA 2009 T3} 
	Оцените радиус планеты, гравитацию которой можно преодолеть лишь подпрыгнув вертикально вверх. Плотность планеты считайте равной плотности Земли.

\begin{solution}
	Пусть на Земле человек в среднем может подпрыгнуть на $h=40$ см. Скорость при отталкивании найдем из кинематики:
\begin{equation}
    h=\frac{V^2-0}{2g} \quad\Longrightarrow\quad V=\sqrt{2gh}
\end{equation}
Чтобы человек смог преодолеть гравитацию, эта скорость должна равняться второй космической на поверхности планеты
\begin{equation}
    \sqrt{2gh}=\sqrt{\frac{2GM_p}{R_p}} \quad\Longleftrightarrow\quad \frac{GM_{\oplus}}{R^2_{\oplus}}\cdot h=\frac{GM_p}{R_p}
\end{equation}
Массу представим как
\begin{equation}
    M=\frac{4}{3}\pi R^3\cdot\rho
\end{equation}
Тогда выражение перепишется в виде
\begin{equation}
    \frac{\frac{4}{3}\pi R^3_{\oplus}\cdot\rho}{R^2_{\oplus}}\cdot h=\frac{\frac{4}{3}\pi R^3_p\cdot\rho}{R_p} \quad\Longrightarrow\quad R_p=\sqrt{R_{\oplus}h}=\sqrt{6400\cdot 0.4\cdot 10^{-3}}~(\text{км})=1.6~\text{км}
\end{equation}
\begin{equation}
    R_p\approx 1.5~\text{км}
\end{equation}
\end{solution}

\begin{answer}
	$1.5$ км
\end{answer}
\end{problem}