\begin{problem}{IOAA 2009 T6} 
	Выведите выражение для скорости убегания объекта, запущенного из центра протозвездного облака. Плотность облака постоянна, масса облака равна $M$, радиус - $R$. Столкновениями частей облака друг с другом и с запущенным объектом пренебрегите. Свободно падающий с края облака объект достигнет центра со скоростью, равной $\sqrt{\frac{GM}{R}}$.

\begin{solution}
	Запишем закон сохранения энергии для свободно падающего с поверхности облака объекта:
    \begin{equation}
        -\frac{GMm}{R}+\frac{mv^2}{2}=E_0+\frac{mv^2_0}{2}
    \end{equation}
    Скорость на поверхности $v=0$, тогда потенциальная энергия объекта и облака в его центре равна
    \begin{equation}
        E_0=-\frac{mv^2_0}{2}-\frac{GMm}{R}=-\frac{GMm}{2R}-\frac{GMm}{R}=-\frac{3GMm}{2R}
    \end{equation}
    Теперь запишем закон сохранения энергии для случая, когда запущенный объект улетает на бесконечность:
    \begin{equation}
        E_0+\frac{mu^2}{2}=0,
    \end{equation}
    где $u$ - искомая скорость убегания.
    \begin{equation}
        u=\sqrt{-\frac{2E_0}{m}}=\sqrt{\frac{3GM}{R}}
    \end{equation}
\end{solution}

\begin{answer}
	$\sqrt{\frac{3GM}{R}}$
\end{answer}
\end{problem}