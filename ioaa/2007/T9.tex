\usepackage{preamble}

\begin{document}
\begin{problem}{IOAA 2007 T9} 
	Рассчитайте полную светимость звезды с температурой поверхности $7500K$ и радиусом в $2.5$ солнечных. Ответ дайте в солнечных светимостях, температуру поверхности Солнца примите равной $5800K$.

\begin{solution}
	Согласно закону Стефана-Больцмана, запишем выражение для светимости звезды
    \begin{equation}
        L=4\pi R^2\sigma T^4,
    \end{equation}
    где $R$ - радиус звезды, $T$ - температура поверхности.\\
    Найдем отношение искомой светимости звезды к светимости Солнца:
    \begin{equation}
        \frac{L}{L_{\odot}}=\left(\frac{R}{R_{\odot}}\right)^2\left(\frac{T}{T_{\odot}}\right)^4
    \end{equation}
    Найдя численное значение этого выражения, получим ответ, т.е. светимость звезды в светимостях Солнца.
    \begin{equation}
        \frac{L}{L_{\odot}}=2.5^2\cdot\left(\frac{7500}{5800}\right)^4\approx17.5
        \quad\Longrightarrow\quad
        L=17.5L_{\odot}
    \end{equation}
\end{solution}

\begin{answer}
	$17.5$
\end{answer}
\end{problem}
\end{document}