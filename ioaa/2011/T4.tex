\begin{problem}{IOAA 2011 T4}
	Считая, что Фобос движется в точности по круговой орбите в экваториальной плоскости Марса, посчитайте время, в течение которого спутник находится над горизонтом для наблюдателя на Марсианском экваторе. Используйте следующие данные:

    Радиус Марса $R_{Mars}=3393$ км, период обращения Марса вокруг своей оси $T_{Mars}=24.623$ ч, масса Марса $M_{Mars}=6.421\cdot 10^{23}$ кг, радиус орбиты Фобоса $R_P=9380$ км.

\begin{solution}
	Изобразим ситуацию\\
    \begin{figure}[h]\centering
        \begin{tikzpicture}[dot/.style = {draw, fill = black, color = black, circle, inner sep=1.5pt}, helpdot/.style = {draw, fill = gray, color = gray, circle, inner sep=1.5pt}, >=triangle 45]

    \def\radius{2.1}%это можно варьировать
    \def\iradius{2.5} %это можно варьировать
    \coordinate [dot, label=above:{Марс}] (O) at (0,0);
    \coordinate [dot, label=below:{Наблюдатель}] (A) at (0,-\iradius);
    \coordinate [dot, label={right:{$B$}}] (west_quadrature) at ($(A)! {sqrt(\radius^2 - 1)} ! -90: (O)$);
    \coordinate [dot, label={left:{$A$}}] (east_quadrature) at ($(A)! {sqrt(\radius^2 - 1)} ! 90: (O)$);
    \node [draw,circle through=(A)] at (O) {};
    \node [draw,circle through=(west_quadrature)] at (O) {};
    \node [draw,circle through=(east_quadrature)] at (O) {};
    %\coordinate [helpdot, label=below:{$A$}] (opposition) at ($(sun)! \radius !(iss)$);
    \draw (O) -- node[left] {$R_{Mars}$} (A) -- (west_quadrature) -- (O) -- (east_quadrature) -- (A);
    %\draw [dashed] (iss) -- node[left] {$\Delta h$} (opposition);
    \tkzMarkAngle[mark](A,O,west_quadrature);
    \tkzLabelAngle[pos = 1.3](A,O,west_quadrature){$\alpha$};
    \tkzMarkRightAngle(O,A,west_quadrature);
    \def\helparc{1.1*\iradius*\radius}
    \def\deltaangle{6}
    \def\angleb{{-asin(1/\radius)-\deltaangle}}
    \def\anglea{{180+\deltaangle+asin(1/\radius)}}
    \draw [->] (O) (-\anglea:\helparc) arc (-\anglea:\angleb:\helparc);
\end{tikzpicture}
        \caption{Марс и орбита Фобоса\label{2011/T4/duration}}
    \end{figure}\\
    На рисунке А - точка восхода, В - точка захода Фобоса. По дуге АВ спутник движется с относительной скоростью 
    \begin{equation}
        \omega=\omega_P-\omega_{Mars}=\frac{360}{T_P}-\frac{360}{T_{Mars}}~(^{\circ}/\text{ч})
    \end{equation}
    Найдем период обращения Фобоса вокруг Марса через III обобщенный закон Кеплера:
    \begin{equation}
        T_P=2\pi\sqrt{\frac{R^3_P}{GM_{Mars}}}=2\pi\sqrt{\frac{9380000^3~(\text{м$^3$})}{G\cdot 6.421\cdot 10^{23}~\text{кг}}}=7.659~\text{ч}
    \end{equation}
    Тогда найдем относительную угловую скорость
    \begin{equation}
        \omega=\frac{360}{7.659}-\frac{360}{24.623}=32.4~(^{\circ}/\text{ч})
    \end{equation}
    Теперь найдем угол $\alpha$:
    \begin{equation}
        \alpha=\arccos{\left(\frac{R_{Mars}}{R_P}\right)}=\arccos{\left(\frac{3393}{9380}\right)}=68.8^{\circ}
    \end{equation}
    За сеанс наблюдения спутник проходит угол $2\alpha=137.6^{\circ}$. Теперь мы можем найти время видимости
    \begin{equation}
        t=\frac{2\alpha}{\omega}=\frac{137.6}{32.4}=4.25~\text{ч}
    \end{equation}
\end{solution}

\begin{answer}
	$4.25$ ч
\end{answer}
\end{problem}