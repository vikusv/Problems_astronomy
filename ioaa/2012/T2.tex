\begin{problem}{IOAA 2012 T2} 
	Вычислите продолжительность звездных суток на Земле. Как долго бы длились звездные и солнечные сутки на Земле в наших единицах измерения времени (часах, минутах и секундах), если бы она вращалась в противоположную сторону с неизменной скоростью? 

\begin{solution}
	Известно, что солнечные сутки на Земле длятся с хорошей точностью $T_{\text{солн.}}=24$ часа, а сидерический год примерно $T_0=365.25$ солнечных дней. Тогда по формуле синодического периода (коим и является для Земли длительность солнечных суток) найдем продолжительность звездного дня:
    \begin{equation}
        \frac{1}{T_{\text{солн.}}}=\frac{1}{T_{\text{зв.}}}\pm\frac{1}{T_0} \quad\Longrightarrow\quad T_{\text{зв.}}=\left(\frac{1}{T_{\text{солн.}}}+\frac{1}{T_0}\right)^{-1}
    \end{equation}
    Знак в первой формуле зависит от того, совпадают ли направления вращения планеты по орбите и вокруг своей оси: совпадают - <<->>, иначе - <<+>>.
    \begin{equation}
        T_{\text{зв.}}=\left(\frac{1}{24~\text{ч}}+\frac{1}{365.25\cdot 24~(\text{ч})}\right)^{-1}=23^h56^m4^s
    \end{equation}
    Вращайся Земля в другую сторону (по часовой стрелке при наблюдении с северного полюса), длительность звездных суток осталась бы неизменной, т.к. это собственный период обращения Земли вокруг своей оси, а длительность солнечных суток находилась бы по ранее написанной формуле:
    \begin{equation}
        T_{\text{солн.}}=\left(\frac{1}{T_{\text{зв.}}}+\frac{1}{T_0}\right)^{-1}=\left(\frac{1}{23.9344~\text{ч}}+\frac{1}{365.25\cdot 24~(\text{ч})}\right)^{-1}=23^h52^m9^s
    \end{equation}

\end{solution}

\begin{answer}
	$23^h56^m4^s$; \
    $23^h56^m4^s$ и $23^h52^m9^s$
\end{answer}
\end{problem}