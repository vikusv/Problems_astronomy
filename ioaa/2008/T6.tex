\begin{problem}{IOAA 2008 T6} 
	Радионаблюдения газового облака, вращающегося вокруг черной дыры в центре нашей галактики показали, что излучение атомов нейтрального водорода (лабораторная частота $1420.41$ МГц) наблюдается на частоте $1421.23$ МГц. Считая, что облако расположено в $0.2$ парсеках от черной дыры и его орбита круговая, определите его скорость и направление движения (от нас или к нам), а также массу черной дыры.

\begin{solution}
	Согласно эффекту Доплера
    \begin{equation}
        \frac{\lambda-\lambda_0}{\lambda_0}=\frac{v}{c} \quad\Longrightarrow\quad v=c\cdot\frac{\lambda-\lambda_0}{\lambda_0}
    \end{equation}
    Из соотношения $\lambda\nu=c$ выразим частоту и подставим в предыдущее выражение
    \begin{equation}
        v=c\cdot\frac{\frac{c}{\nu}-\frac{c}{\nu_0}}{\frac{c}{\nu_0}}=c\cdot\frac{\nu_0-\nu}{\nu}=c\cdot\frac{1420.41-1421.23}{1421.23}=-173~\text{км/с}
    \end{equation}
    Знак <<->> в получившемся значении скорости указывает на то, что облако приближается к нам.

    Раз орбита круговая, то скорость облака запишем как первую космическую:
    \begin{equation}
        v=\sqrt{\frac{GM_{BH}}{R}} 
    \end{equation}
    Отсюда масса черной дыры
    \begin{equation}
        M_{BH}=\frac{V^2R}{G}=\frac{173000^2~(\text{м$^2$/с$^2$})~\cdot 0.2\cdot 206265\cdot 1.5\cdot 10^{11}~\text{м}}{G}=2.78\cdot 10^{36}~\text{кг}=1.4\cdot 10^6~M_{\odot}
    \end{equation}

\end{solution}

\begin{answer*}
	$173$ км/с; \
    к нам; \
    $1.4\cdot 10^6~M_{\odot}$ 
\end{answer*}
\end{problem}