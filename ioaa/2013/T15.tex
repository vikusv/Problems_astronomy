\begin{problem}{IOAA 2013 T15}
	При помощи наблюдения спектров трех далеких галактик было измерено их Доплеровское смещение.\\
    \begin{table}[h]
    \centering
    \begin{tabular}{|c|c|c|}\hline
        Галактика & Красное смещение, z \\ \hline
        $3C279$ & $0.536$ \\ \hline
        $3C245$ & $1.029$ \\ \hline
        $4C41.17$ & $3.8$ \\ \hline
    \end{tabular}
    \end{table}
    \begin{enumerate}
        \item Посчитайте скорости удаления галактик с использованием (1) классической модели (2) часто применяемой космологами формулы приближенного вычисления $v=c\ln{(1+z)}$,~(3) релятивистской модели.
        \item Каково процентное отношение скорости, с которой происходит видимое удаление галактик, к скорости света во всех трех моделях?
        \item Какая из моделей (1), (2) или (3) яляется космологической?
    \end{enumerate} 
\begin{solution}
	Для описания скорости удаления галктик в классической модели используется закон Хаббла 
    \begin{equation}
        v_1=c\cdot z
    \end{equation}
    Вторая модель описана условии
    \begin{equation}
        v_2=c\ln{(1+z)}
    \end{equation}
    В релятивистской модели скорость удаления находится из выражения
    \begin{equation}
        v_3=c\cdot\frac{(1+z)^2-1}{(1+z)^2+1}
    \end{equation}
    Занесем посчитанные скорости в таблицу\\
    \begin{table}[h]
    \centering
    \begin{tabular}{|c|c|c|c|c|}\hline
        Галактика & $v_1$, км/с & $v_2$, км/с & $v_3$, км/с & $u,~\%$ \\ \hline
        $3C279$ & $160700$ & $128700$ & $121300$ & $40$ \\ \hline
        $3C245$ & $308500$ & $212100$ & $182600$ & $61$ \\ \hline
        $4C41.17$ & $1139200$ & $470300$ & $274900$ & $92$ \\ \hline
    \end{tabular}
    \end{table}\\
    При таких больших красных смещениях видимая скорость удаления галактик лучше описывается в релятивистской модели (СТО):
    \begin{equation}
        u=\frac{v_3}{c}\cdot 100\%
    \end{equation}
    Из условия следует, что модель (2) является космологической.

\end{solution}

\begin{answer}
	таблица; \
    (2)
\end{answer}
\end{problem}