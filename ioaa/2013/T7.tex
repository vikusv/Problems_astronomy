\begin{problem}{IOAA 2013 T7} 
	Звезда имеет видимую звездную величину $m_V=12.2^m$, параллакс $\pi_0=0.001"$ и эффективную температуру $T_{eff}=4000K$, а также болометрическую поправку $BC=-0.6^m$
    \begin{itemize}
        \item Найдите светимость звезды в единицах солнечной светимости.
        \item Какому классу принадлежит звезда?
            \begin{enumerate}
                \item красный гигант
                \item голубой гигант
                \item красный карлик
            \end{enumerate}
    \end{itemize}
    Запишите один из ответов (a), (b) или (c) на лист ответов.

\begin{solution}
	Болометрическая поправка - разность болометрической и видимой звездных величин:
    \begin{equation}
        BC=m_{bol}-m_V \quad\Longrightarrow\quad m_{bol}=m_V+BC=12.2^m-0.6^m=11.6^m
    \end{equation}
    Для более удобных вычислений найдем болометрическую абсолютную звездную величину звезды
    \begin{equation}
        M=m_{bol}+5+5\log{(\pi_0)}=1.6^m
    \end{equation}
    Сравним звезду с Солнцем по формуле Погсона:
    \begin{equation}
        \frac{L}{L_{\odot}}=10^{0.4(M_{\odot}-M)},
    \end{equation}
    где $M_{\odot}=4.7^m$ - абсолютная болометрическая звездная величина Солнца. Из этого уравнения выразим светимость звезды в светимостях Солнца:
    \begin{equation}
        \frac{L}{L_{\odot}}=10^{0.4(M_{\odot}-M)}=10^{0.4(4.7^m-1.6^m)}=17.4
    \end{equation}
    Итого $L=17.4L_{\odot}$, а $T_{eff}=4000K$. Звезда имеет большую светимость при довольно скромной температуре, что говорит о ее принадлежности к классу красных гигантов.
\end{solution}

\begin{answer}
	$17.4$; \ 
    (a)
\end{answer}
\end{problem}