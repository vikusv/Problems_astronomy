\documentclass[12pt]{article}


\usepackage{baseset}
\usepackage{multicol}
\usepackage{enumitem}
\usepackage{tabularx}

\begin{document}
\begin{tabularx}{\textwidth}{Xr}
{\Large \textbf{Небесная механика}} & Взлет, $09.05.2024$ \\
\end{tabularx}
\noindent\rule{\textwidth}{0.4pt}
\begin{enumerate}
    \item Существовала гипотеза, что астероиды (малые планеты) образовались после взрыва некоторой планеты в Солнечной системе. Сколько примерно астероидов могло бы образоваться из Луны, если предположить, что все получившиеся астероиды имеют диаметр $1$ км. Радиус Луны $1740$ км. Считать, что все получившиеся астероиды имеют средние плотности, равные средней плотности Луны. %4.4 * 10^10
    \item Определите, за какое время свет от Солнца достигает Земли, если расстояние между Солнцем и Землей составляет $150\cdot10^6$~км?
    \item Звездная система $61$ Лебедя находится от Солнца на расстоянии $11.36$ световых лет или $3.48$ парсека, и приближается к нам со скоростью $64$ км/с. Определите, за какое время до нее может долететь космический аппарат со скоростью $16$ км/с. Как изменится ответ, если удастся разогнать космический аппарат до вдвое большей скорости?
    \item Определите период вращения Бетельгейзе вокруг своей оси. Линейная скорость вращения звезды на экваторе -- $5.5$ км/c. %19.4
    \item Рассчитайте угловую и линейную скорости точки на экваторе относительно оси вращения? Радиус Земли $R_{\oplus}=6~400$ км. 
    \item С какой линейной скоростью движется Санкт-Петербург (широта $60^{\circ}$) относительно оси вращения?
    \item Определите скорость вращения Марса по орбите вокруг Солнца. Во сколько раз эта скорость меньше скорости вращения точки на экваторе Марса? %24 км/с, 0,24 км/с
    \item Период обращения Юпитера вокруг Солнца составляет $12$ лет. Определите большую полуось орбиты Юпитера.
    \item Среднее расстояние Марса от Солнца составляет $228$~млн.~км. Определите период обращение Марса вокруг Солнца.
    \item Астероид обращается вокруг Солнца по круговой орбите за $8$ лет. Чему равен радиус его орбиты?
    \item Во времена Советского Союза летчикам, налетавшим миллион километров, выдавался специальный значок. За какое время обычный житель Земли пролетает $1$ миллион километров вместе с Землей вокруг Солнца?
    \item Найдите длину пути, который пройдет МКС относительно Земли за календарный год.
    \item Определите период обращения термозащищенного спутника находящегося прям над поверхностью Бетельгейзе. Радиус Бетельгейзе $\approx 764 R_{\odot}$. Масса -- $19 M_{\odot}$. %1.55 года
    \item Каков период обращения Луны вокруг Земли?
    \item Как должна была бы мгновенно измениться масса Земли, чтобы оставаясь на прежнем расстоянии, Луна обращалась вокруг Земли за $2$ суток? %186 раз
    \item Оцените пероиод обращения Солнечной системы относительно центра галактики. Масса центра галактики $2\cdot10^{12}$. Расстояния от Солнца до ближайшей черной дыры $8$ кпк. %47 миллионов лет
    \item Экзопланета обращается вокруг звезды массы $0.85 M_{\odot}$ с периодом $0.78$ лет. Определите полуось орбиты планеты в астрономических единицах и километрах. % 0.8 ае, 1.2*10^8
    \item Предположим, мы наблюдаем двойную систему, состоящую из двух звёзд, массы которых $2$ и $3$ массы Солнца, а период системы равен $4$ года. Определите расстояние между звёздами. %4.3 ае
\end{enumerate}
\end{document}