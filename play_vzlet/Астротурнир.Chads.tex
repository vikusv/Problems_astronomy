\documentclass[a4paper,12pt]{article}
\usepackage{baseset_Katya}
\usepackage[bottom= 2 cm, top= 2 cm, right= 1.5 cm, left= 1.5 cm]{geometry}

\begin{document}
	
	\begin{minipage}{0.15 \textwidth}
		\includegraphics[width= \textwidth]{Logo.png}
	\end{minipage}
	\begin{minipage}{0.85 \textwidth}
		\begin{center}
			\begin{Huge}
				Астротурнир
			\end{Huge}
			
			\bigskip
			
			\begin{Large}
				\textit{Московская область} \hspace{25 pt}
				\textit{$22$ ноября $2023$}
			\end{Large}
			
		\end{center}
	\end{minipage}
	
	\bigskip
	
	\hrule
	
\section*{Сферическая астрономия}
	

\subsection*{1.}

Переведите угол в часовую меру $46^{\circ} 46' 45''$.

\subsection*{2. }

Определите широту, на которой верхняя кульминация звезды и нижняя проходят по одну сторону от зенита, при этом зенитное расстояние в верхней кульминации равно высоте в нижней кульминации.


\subsection*{3.}

Определите расстояние между Москвой ($\varphi = 35^{\circ}$ с.ш., $\lambda = 37.5^{\circ}$ в.д.)и  Канберой ($\varphi = 35.3^{\circ}$ ю.ш., $\lambda = 149.1^{\circ}$ в.д.).


\subsection*{4.}

Определите какого числа Солнце кульминирует в $12$ часов по звездному времени. 

\subsection*{5. }

На некоторой широте Вега ($\delta = 38^{\circ} 47'$) кульминирует в зените. Определите высоту кульминации для Сириуса ($\delta = -16^{\circ} 42'$).

\subsection*{6. }

Определите высоту Солнца над горизонтом 23 июня в Долгопрудном ($\varphi = 56^{\circ},~\lambda = 37.5^{\circ}$). В каком городе Солнце кульминирует на горизонте в этот день?  %%Заменила

\subsection*{7. }

Определите склонения звезд, которые являются восходящими на широте $\varphi_0 = 35^{\circ}$.

\subsection*{8.}

Определите широты, на которой звезда, кульминирующая в зените кульминирует на горизонте. Чему равно склонение такой звезды.

\subsection*{9.} Зенитное расстояние верхней кульминации некоторой звезды равно $30^{\circ}$, а полярное расстояние этой звезды $65^{\circ}$. Определите широту места наблюдения.

\subsection*{10.}

В каком случае звезда может наблюдаться и в зените, и в надире?

\subsection*{11.}

У некоторой звезды высота верхней кульминации равна $40^{\circ}$, а в нижней в два раза меньше. Определите склонение звезды. Какой звездой является эта звезда?

\subsection*{12.} 

В этом году День рождение Маши пришлось на понедельник. Когда в следующий раз такое произойдет?  %2028

\subsection*{13.} 

Некоторый год начался во вторник. Определите в какой день недели начнется год, который будет через 8 лет.


\section*{Небесная механика}

\subsection*{14.}

О далекой планетарной системе известно, что вокруг центральной звезды вращаются две экзопланеты, радиусы орбит которых отличаются в 2 раза. Определите во сколько раз различаются скорости этих планет. Орбиты планет считать круговыми.

\subsection*{15.}

Определите чему равен параллакс двойной системы, если ее угловой размер 0,5”, а расстояние между двумя компонентами равно 5 ае

\subsection*{16.}

Оцените, в каком случае максимальное угловое расстояние от Солнца больше: Венеры в элонгации для земного наблюдателя или Земли для наблюдателя на Марсе?

\subsection*{17.}

Большая полуось астероида равна $3.95$ ае. Каким может быть расстояние между астероидом и Марсом , когда он оба в квадратурах при наблюдении с Земли?

\subsection*{18.}

Оцените высоту гелиостационарной орбиты для Солнца. Период вращения Солнца на экваторе составляет примерно $25$ суток.


\subsection*{19.}

Определите время между квадратурами Юпитера.

\subsection*{20.}

Синодический период некоторого астероида равен $500$ дней для наблюдателя на Марсе. Определите период астероида, если у него обратное вращение. Орбиты считать круговыми.


\subsection*{21. }

Определите за какое время произойдет картографирование Венеры, если она находиться в элонгации для наблюдателя на Земле.	

\subsection*{22.}

Как часто Юпитер оказывается в соединении для наблюдателя на Марсе? А Венера?

\section*{Луна}

\subsection*{23.}

Определите сколько раз за сутки Земля восходит для наблюдателя на Луне. Ответ поясните.

\subsection*{24.}

$21$ сентября Луна кульминировала ровно в полночь. Определите в какое время будет ее следующая кульминация.

\subsection*{25.}

Определите плотность Луны.


\section*{Углы и параллаксы}

\subsection*{26.}

Определите каков должен быть радиус планеты, чтобы ее горизонтальный параллакс был вдвое больше ее углового размера.

\subsection*{27.}

Угловой диаметр Марса в противостоянии $25''$, горизонтальный параллакс
$23''$. Определите радиус Марса.

\subsection*{28.}

Определите горизонтальный параллакс Юпитера в западной квадратуре,
если он движется по круговой орбите с большой полуосью 5.2 а.е

\subsection*{29.}

Некоторая звезда двигается по окружности с угловым радиусом $5''$  для наблюдателя на Земле. Определите период обращения планеты. 

\subsection*{30.}

Определите все параллаксы, которые знаете для телескопа Джеймс Вебб, если расстояние до телескопа $1.5 \cdot 10^6$ км.

	
	
\end{document}
