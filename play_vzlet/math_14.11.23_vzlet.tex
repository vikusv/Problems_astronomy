\documentclass[12pt]{article}


\usepackage{baseset}
\usepackage{newcom}
\usepackage{multicol}
\usepackage{enumitem}
\usepackage{tabularx}

\begin{document}
 \begin{tabularx}{\textwidth}{Xr}
  {\Large \textbf{Математика}} & Взлет, $14.11.2023$ \\
 \end{tabularx}
 \noindent\rule{\textwidth}{0.4pt}
 \begin{enumerate}
    \item Вычислите и представьте ответ в экспоненциальной форме:
    \begin{multicols}{2}
		\begin{enumerate}[label=\textbf{\Alph*.}]
			\item{
                $(0.9893:0.13 - 6.4)\cdot62.9 - 7.109.$
            }
			\item{
                $\left(5.2^3 + (-7.6 + 8.9)^2\right)^5.$
            }
		\end{enumerate}
	\end{multicols}

    \item Решите линейные уравнение:
	\begin{multicols}{2}
		\begin{enumerate}[label=\textbf{\Alph*.}]
			\item{
                $\dfrac{8x+1}{3} = \dfrac{5x-1}{7}.$
            } \item{
                $\dfrac{3+7x}{7} = \dfrac{8 + 8x}{8}.$
            }
			\item{
                $(3x - 1)^2 + (4x + 5)^2 = (5x - 7)^2.$
            } \item{
                $21.5x = 8.25x - 16.75.$
            } 
		\end{enumerate}
	\end{multicols}

    \item Найдите значение выражения:
    \begin{multicols}{2}
		\begin{enumerate}[label=\textbf{\Alph*.}]
			\item{
                $\sqrt{\left(\sqrt{16}\right)^2 + \left(\sqrt[3]{8}\right)^3}.$
            } \item{
                $\sqrt{\left(\sqrt{9}\right)^2 - \left(\sqrt[3]{64}\right)^3}.$
            }
			\item{
                $\sqrt[3]{\sqrt{125} + \sqrt{144}}.$
            } \item{
                $\sqrt[4]{\sqrt[3]{27} - \sqrt{81}}.$
            } 
		\end{enumerate}
	\end{multicols}

    \item Найдите значение выражения:
    \begin{multicols}{2}
		\begin{enumerate}[label=\textbf{\Alph*.}]
			\item{
                \(\dfrac{2a^2 + \sqrt{bc}}{a + b}\) при \(a = 3\), \(b = 2\), \(c = 16\).
            } \item{
                \(\dfrac{a^3 - 2b}{c^2 + ab}\) при \(a = 4\), \(b = 5\), \(c = 2\).
            }
			\item{
                \(\dfrac{a^2 + 3b}{\sqrt{c^2 + 1}}\) при \(a = 2\), \(b = 7\), \(c = 5\).
            } \item{
                \(\dfrac{a^2 + 2b^2}{c - 1}\) при \(a = 1\), \(b = 3\), \(c = 4\).
            } 
		\end{enumerate}
	\end{multicols}
     
    \item Выразите углы в часовой мере:
    \begin{multicols}{2}
		\begin{enumerate}[label=\textbf{\Alph*.}]
			\item{
                $45^\circ 30' 15''.$
            } \item{
                $120^\circ 45' 30''.$
            }
			\item{
                $25^\circ 15' 40''.$
            } \item{
                $330^\circ 20' 10''.$
            } 
		\end{enumerate}
	\end{multicols}

    \item Выразите углы в градусной мере:
    \begin{multicols}{2}
		\begin{enumerate}[label=\textbf{\Alph*.}]
			\item{
                $5 \, \text{ч} \, 20 \, \text{мин} \, 45 \, \text{с}.$
            } \item{
                $12 \, \text{ч} \, 35 \, \text{мин} \, 10 \, \text{с}.$
            }
			\item{
                $8 \, \text{ч} \, 15 \, \text{мин} \, 30 \, \text{с}.$
            } \item{
                $3 \, \text{ч} \, 50 \, \text{мин} \, 55 \, \text{с}.$
            } 
		\end{enumerate}
	\end{multicols}

    \item Рассчитайте длину экватора Земли, Солнца, Луны.
    \item Рассчитайте длину $1^{\circ}$ параллели на широтах $10^{\circ}$, $-30^{\circ}$ и $65^{\circ}$.
    \item Чему равна площадь главного зеркала телескопа с диаметром $20$ см? А с диметром $30$ см? А с радиусом $5$ см?
    \item Найдите объемы Земли, Солнца, Луны.
    \item Группа исследователей отправляется в поход по неисследованным районам. Они ставят лагерь на берегу прямоугольного озера и замечают, что на противоположном берегу видны две вершины соседних гор, образующих угол в $90$ градусов относительно точки лагеря. Исследователи измеряют расстояние от лагеря до каждой вершины горы и находят, что одно расстояние равно $3$ км, а другое -- $4$ км. Каково расстояние между вершинами гор?
    \item Представьте, что Солнце, Земля и Марс образовали в пространстве прямоугольный треуголбник. Чему равно расстояние между Землей и Марсом? Решите эту же задачу для Венеры и для Юпитера.
    \item Найдите $\sin$, $\cos$ и $\tg$ угла:
    \begin{multicols}{2}
		\begin{enumerate}[label=\textbf{\Alph*.}]
			\item{
                $50^{\circ}.$
            } \item{
                $12^{\circ}.$
            } \item{
                $0^{\circ}.$
            }
			\item{
                $150^{\circ}.$
            } \item{
                $380^{\circ}.$
            } \item{
                $90^{\circ}.$
            }
		\end{enumerate}
	\end{multicols}

    \item Далеко в космосе существует планетная система из трех планет. При наблюдении с планеты $A$ угол между двумя другими в некоторый момент составляет $30^{\circ}$. А угол при наблюдении с планеты $B$ между двумя оставшимися планетами -- $45^{\circ}$. Расстояние между планетами $A$ и $C$ -- $5$ а.е. Найдите расстояния между всеми планетами.

 \end{enumerate}
\end{document}