\documentclass[12pt]{article}


\usepackage{baseset}
\usepackage{multicol}
\usepackage{enumitem}
\usepackage{tabularx}

\begin{document}
\begin{center}
	\textbf{\LARGE{Астротурнир $\mathbf{\gamma}$ $\mathbf{\delta}$}}
\end{center}
\begin{tabularx}{\textwidth}{Xr}
	{\textit{Взлёт, игровая смена}} & $22$ ноября $2023$ \\
\end{tabularx}

\noindent\rule{\textwidth}{0.4pt}

\section*{Сферическая астрономия}
\begin{enumerate}
	\item Переведите угол в часовую меру: $46^{\circ} 46' 45''$.
    \item Определите, во сколько раз отличаются зенитные расстояния звезды со склонением $30^{\circ}$ при наблюдении на широте $60^{\circ}$ в верхней и нижней кульминациях.
    \item Определите расстояние между Москвой ($\varphi = 35^{\circ}$ с.ш., $\lambda = 37.5^{\circ}$ в.д.) и Канберой ($\varphi = 35.3^{\circ}$ ю.ш., $\lambda = 149.1^{\circ}$ в.д.) по поверхности Земли.
    \item Определите склонения звезд, которые являются восходящими на широте $\varphi = 35^{\circ}$.
    \item Зенитное расстояние верхней кульминации некоторой звезды равно $30^{\circ}$, а полярное расстояние этой звезды $65^{\circ}$. Определите широту места наблюдения.
    \item Определите координаты звезд (прямое восхождение и склонение), которые кульминируют с зенитным расстоянием $z = 30^{\circ}$ в момент восхода точки осеннего равноденствия для наблюдателя в пункте с географической широтой $\varphi = 60^{\circ} 30'$ с.ш.
    \item В каком случае звезда может наблюдаться и в зените, и в надире?
    \item В некоторый момент звезда со склонением $30^{\circ}$ находилась в кульминации для наблюдателя в Санкт-Петербурге ($\varphi = 60^{\circ}$). В тот же момент вторая звезда оказалась также в кульминации, причем сумма высот звезд составила $125^{\circ}$. Определите склонение второй звезды.
    \item У некоторой звезды высота верхней кульминации равна $40^{\circ}$, а в нижней в два раза меньше. Определите склонение звезды. Что можно сказать о видимости этой звезды?
    \item Самолет на высоте $h=10$~км пролетает над Сингапуром (широта $\varphi \approx 0^{\circ}$) в день весеннего равноденствия. Пассажиры видят восход Солнца. Через какое время восход Солнца увидят жители Сингапура?
\end{enumerate}

\section*{Небесная механика}
\begin{enumerate}[resume]
	\item Определите, во сколько раз отличается вес человека на экваторе и на полюсе на Юпитере.
    \item Определите плотность планеты, которую можно облететь за $15$ минут с выключенными двигателями.
    \item Станция управления полетами, расположенная на Земле, наблюдает за спутником Сатурна Титаном. В момент захода Титана за диск Сатурна космический аппарат на орбите Титана отправляет световой сигнал на Землю. Определите, с какой временной задержкой должен быть отправлен ответ от станции управления полетами, чтобы спутник получил ответ сразу, после того, как выйдет из-за диска планеты. Орбиты Земли, Сатурна и Титана считать круговыми и лежащими в одной плоскости. Сатурн в момент наблюдений находится в противостоянии.
    \item Между восточной квадратурой и последующей западной квадратурой некоторой планеты проходит в $1.143$ раз больше времени, чем между ее западной и последующей восточной квадратурой. Что это за планета? Орбиты планет считать круговыми. 
    \item В некоторый момент времени астероид, находившийся в квадратуре на расстоянии $2$ а.е. от Земли, оказался рядом с кометой, скорость которой была равна второй космической. Определите скорость кометы. 
    \item Планеты Венера и Юпитер вступают в соединение друг с другом для наблюдателя на Земле, имея одинаковые экваториальные угловые размеры. Чему равно угловое расстояние между Венерой и Солнцем в этот момент?
    \item Две звезды с массами $1M_{\odot}$ и $1.4M_{\odot}$ вращаются вокруг общего центра масс с периодом $1$~год. Определите угловые размеры обеих звезд для наблюдателя, находящегося в центре масс.
    Первая звезда является звездой, полностью похожим на Солнца, а вторая звезда белый карлик с радиусом $10~000$ км.
    \item Синодический период астероида при его ретроградном движении в два
    раза меньше в случае, если бы он вращался сонаправленно с Землёй. Определите его большую полуось.
    \item Определите, на каких внутренних планетах Солнечной системы день длится больше, чем год (период обращения планеты вокруг Солнца).
\end{enumerate}

\section*{Луна}
\begin{enumerate}[resume]
	\item Определите, сколько раз за сутки Земля восходит для наблюдателя на Луне, который находится неподалеку от кратера Тихо. Ответ поясните.
    \item $21$ сентября Луна кульминировала ровно в полночь. Определите, в какое время будет ее следующая кульминация.
    \item Определите плотность Луны.
\end{enumerate}

\section*{Время и календарь}
\begin{enumerate}[resume]
    \item В этом году день рождения Маши пришелся на понедельник. Когда в следующий раз такое произойдет?
    \item Календарный год начался с понедельника, а закончился вторником. Каким днём недели закончится следующий календарный год?
    \item Определите, в какое звездное время кульминирует Солнце $27$ ноября. Считайте, что орбита Земли круговая.
    \item Определите координаты звезды, которая восходит в точке с азимутом $270^{\circ}$ в $15$ часов по звездному времени.
\end{enumerate}

\section*{Углы и параллаксы}
\begin{enumerate}[resume]
	\item Некоторая планета вращается вокруг звезды, масса которой равна $2M_{\odot}$, при этом при наблюдении с земли угловой радиус ее орбиты равен $5''$. Определите период ее обращения вокруг центральной звезды.
    \item Определите в каком диапазоне лежит суточный параллакс телескопа Джеймс Вебб, если расстояние до телескопа $1.5 \cdot 10^6$ км.
    \item Определите горизонтальный параллакс Меркурия и его угловой размер в момент прохождения Меркурия по диску Солнца. Орбиту Меркурия считать круговой.
\end{enumerate}
\end{document}