\documentclass[12pt]{article}


\usepackage{baseset}
\usepackage{newcom}
\usepackage{multicol}
\usepackage{enumitem}
\usepackage{tabularx}

\begin{document}
\begin{center}
	\textbf{\LARGE{Крестики-нолики}}
\end{center}
\begin{tabularx}{\textwidth}{Xr}
	{\textit{Взлёт, игровая смена}} & $20$ ноября $2023$ \\
\end{tabularx}

\noindent\rule{\textwidth}{0.4pt}

\begin{enumerate}[label=\textbf{A\arabic*.}]
	\item Определите склонения для невосходящих и незаходящих звёзд при наблюдении из Мезмая ($\varphi = 44^{\circ}$).
    %невосходящее - 46, незаходящие 46
    \item Одну и ту же звезду наблюдают в Киеве ($\varphi_K = 50^{\circ}27'$) и Петрозаводске ($\varphi_{\text{П}} = 61^{\circ}47$). Насколько будет отличаться высота этой звезды в нижней кульминации при наблюдении из этих городов?
    % менее чем на 11 град 20 '
    \item Предположим, мы наблюдаем двойную систему, состоящую из двух звёзд, массы которых $2$ и $3$ массы Солнца, а период системы равен $4$ года. Определите расстояние между звёздами.
    %4.3 ае
    \item Экзопланета обращается вокруг звезды массы $0.85 M_{\odot}$ с периодом $0.78$ лет. Определите полуось орбиты планеты в астрономических единицах и километрах.
    % 0.8 ае, 1.2*10^8
    \item Как должна была бы мгновенно измениться масса Земли, чтобы оставаясь на прежнем расстоянии, Луна обращалась вокруг Земли за $2$ суток? %186 раз
	
\end{enumerate}

\begin{enumerate}[label=\textbf{B\arabic*.}]
	\item Определите, во сколько раз отличаются угловые размеры Солнца для наблюдателей с Земли и Марса. %В среднем 1.52 раза
    \item Определите угловое расстояние между Солнцем и Землей для наблюдателя в системе Проксима Центавра, расстояние до системы $4.2$ световых года. %0.78''
    \item Планетарная туманность, находящаяся на расстоянии $2.3$ тыс. св. лет, имеет радиус $0.23$ пк. Чему равен ее угловой диаметр? %2'
    \item Чему равны координаты точки небесной сферы, противоположной точке с координатами $\alpha = 20^h15^m$ и $\delta = 20^{\circ}$? % 8 часов 15 минут, - 20
    \item В некоторой точке Земли звезда Капелла ($\delta_1 = 46^{\circ}$) кульминирует в зените. Определите высоту кульминации Веги ($\delta_2 = 38^{\circ}$) в данном месте. % 82 - вк, - 6 - в нк
\end{enumerate}

\begin{enumerate}[label=\textbf{C\arabic*.}]
	\item В некотором году сред было на одну больше, чем воскресений. Сколько в этом году было вторников? %52 или 53
    \item Отличник Ваня отпраздновал свой $12$-ый день рождения $12$ апреля в воскресение. В том же году $12$ августа был максимум метеорного потока Персеиды. Каким днем недели было $12$ августа? %Среда
    \item Если первое января -- понедельник, то каким днём недели закончится простой и високосный год? %понедельник и вторник
    \item Существовала гипотеза, что астероиды (малые планеты) образовались после взрыва некоторой планеты в Солнечной системе. Сколько примерно астероидов могло бы образоваться из Луны, если предположить, что все получившиеся астероиды имеют диаметр $1$ км. Радиус Луны $1740$ км. Считать, что все получившиеся астероиды имеют средние плотности, равные средней плотности Луны. %4.4 * 10^10
    \item Определите скорость вращения Марса по орбите вокруг Солнца. Во сколько раз эта скорость меньше скорости вращения точки на экваторе Марса? %24 км/с, 0,24 км/с
\end{enumerate}

\begin{enumerate}[label=\textbf{D\arabic*.}]
	\item Параллакс некоторой двойной системы равен ее угловому размеру. Определите параллакс системы. %1"
    \item Оцените пероиод обращения Солнечной системы относительно центра галактики. Масса центра галактики $2\cdot10^{12}$. Расстояния от Солнца до ближайшей черной дыры $8$ кпк. %47 миллионов лет
    \item Нижняя кульминация звезды совпадает с горизонтом, а верхняя кульминация с зенитом. Определите широту места наблюдения и склонение звезды %45
    \item Определите склонение звезды, которая в Долгопрудном ($\varphi = 55^{\circ}56'$) и Владивостоке ($\varphi =43^{\circ}11'$) кульминирует на одной и той же высоте. %δ = 49◦33'30"
    \item Некое светило видно в Санкт-Петербурге ($\varphi = 60^{\circ}$) в зените. Зайдет ли оно за горизонт? %Нет
\end{enumerate}

\begin{enumerate}[label=\textbf{E\arabic*.}]
	\item На какое расстояние нужно приблизиться к Юпитеру, чтобы его угловой размер сравнялся со средним угловым размером полной Луны на Земле? % 158 млн км
    \item Космический зонд «Розетта», который летал к комете Чурюмова-Герасименко, обнаружил, что комета, в среднем, ежесекундно испаряет в пространство примерно стакан воды ($200$ мл) Считая, что комета практически полностью состоит из воды, оцените, какое время она еще будет существовать. Масса кометы Чурюмова-Герасименко равна $10^{13}$ кг. % 1.59 млн. лет. 
    \item Определите период обращения термозащищенного спутника находящегося прям над поверхностью Бетельгейзе. Радиус Бетельгейзе $\approx 764 R_{\odot}$. Масса -- $19 M_{\odot}$. %1.55 года
    \item Определите период вращения Бетельгейзе вокруг своей оси. Линейная скорость вращения звезды на экваторе -- $5.5$ км/c. %19.4 года
    \item Определите скорость вращения наблюдателя, который видит Солнце в зените ровно раз в год. %0.42 кс/с
\end{enumerate}
\end{document}