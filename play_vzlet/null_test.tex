\documentclass[12pt]{article}


\usepackage{baseset}
\usepackage{newcom}
\usepackage{multicol}
\usepackage{enumitem}
\usepackage{tabularx}

\begin{document}
\begin{center}
	\textbf{\LARGE{Распределительный тест}}
\end{center}
\begin{tabularx}{\textwidth}{Xr}
	{\large\textit{Взлёт, игровая смена}} & $13$ ноября $2023$ \\
\end{tabularx}

\noindent\rule{\textwidth}{0.4pt}

Приветствуем тебя на игровой смене! Этот тест нужен, чтобы распределить тебя в группу, которая будет подходить тебе лучше всего. Он состоит из $30$ вопросов. Время на выполнение -- $1$ час $30$ минут. В тесте есть два типа задача: выбрать правильный/правильные ответ и расставить ответы в некотором порядке. Правильным ответом может служить один или несколько вариантов ответа. Разрешено пользоваться только калькулятором (не на телефоне). Не переживай, если чего-то не знаешь -- вопрос можно пропустить. Удачи! 

\section*{Первый уровень}

\begin{enumerate}
	\item Сколько градусов в $10$ часах?
	\begin{multicols}{2}
		\begin{enumerate}[label=\textbf{\Alph*.}]
			\item{$150$} \item{$30$}
			\item{$10$} \item{$600$}
		\end{enumerate}
	\end{multicols}

	\item Какие созвездия являются невосходящими на широте Москвы?
	\begin{multicols}{2}
		\begin{enumerate}[label=\textbf{\Alph*.}]
			\item{Орион} \item{Малый пес} \item{Октант} 
			\item{Лира} \item{Золотая рыба} \item{Все являются восходящими} 
		\end{enumerate}
	\end{multicols}

	\item В какую сторону двигатся звезды в южном полушарии?
	\begin{multicols}{2}
		\begin{enumerate}[label=\textbf{\Alph*.}]
			\item{По часовой стрелке} 
			\item{Против часовой стрелки} 
		\end{enumerate}
	\end{multicols}

	\item Где-то северном полушарии стоит столб. На какую сторону света указывает тень от столба в местный полдень?
	\begin{multicols}{2}
		\begin{enumerate}[label=\textbf{\Alph*.}]
			\item{Север} \item{Юг}
			\item{Запад} \item{Восток} 
		\end{enumerate}
	\end{multicols}

	\item Луна видна в фазе 3-ей четверти, через сколько времени (примерно) можно будет наблюдать ее в полнолунии?
	\begin{multicols}{2}
		\begin{enumerate}[label=\textbf{\Alph*.}]
			\item{$1$ неделя} \item{$2$ недели}
			\item{$3$ недели} \item{$4$ недели} 
		\end{enumerate}
	\end{multicols}

	\item Чему может равняться широта некоторого пункта на Земле?
	\begin{multicols}{2}
		\begin{enumerate}[label=\textbf{\Alph*.}]
			\item{$125^{\circ}$} \item{$-30^{\circ}$} \item{$60^{\circ}$}
			\item{$90^{\circ}$} \item{$256^{\circ}$} \item{$-110^{\circ}$}
		\end{enumerate}
	\end{multicols}

	\item Сколько времени понадобится космическому кораблю, чтобы долететь до некоторой звезды, если расстояние до нее составляет $10$ св. лет, а корабль летит со скоростью в $10$ раз меньше скорости света?
	\begin{multicols}{2}
		\begin{enumerate}[label=\textbf{\Alph*.}]
			\item{$1$ год} \item{$10$ лет}
			\item{$100$ лет} \item{$1000$ лет} 
		\end{enumerate}
	\end{multicols}

	\item Расставьте объекты в порядке увеличения диаметра.
	\begin{multicols}{2}
		\begin{enumerate}[label=\textbf{\Alph*.}]
			\item{Луна} \item{Астероид}
			\item{Земля} \item{Солнце} 
		\end{enumerate}
	\end{multicols}

	\item Расставьте объекты от самого близкого к самому далекому.
	\begin{multicols}{2}
		\begin{enumerate}[label=\textbf{\Alph*.}]
			\item{Солнце} \item{Вега}
			\item{Туманность Андромеды} \item{Церера} 
		\end{enumerate}
	\end{multicols}

	\item Расставьте планеты в порядке уменьшения числа спутников.
	\begin{multicols}{2}
		\begin{enumerate}[label=\textbf{\Alph*.}]
			\item{Плутон} \item{Марс}
			\item{Венера} \item{Земля} 
		\end{enumerate}
	\end{multicols}
\end{enumerate}
\section*{Второй уровень}
\begin{enumerate}[resume]
	\item В какое время года созвездие Тельца лучше всего видно над горизонтов в северном полушарии?
	\begin{multicols}{2}
		\begin{enumerate}[label=\textbf{\Alph*.}]
			\item{Зима} \item{Весна}
			\item{Лето} \item{Осень} 
		\end{enumerate}	
	\end{multicols}

	\newpage

	\item Два катета прямоугольного треугольника равны $4$ и $3$. Определите его гипотенузу. 
	\begin{multicols}{2}
		\begin{enumerate}[label=\textbf{\Alph*.}]
			\item{$5$} \item{$25$}
			\item{$7$} \item{$2$} 
		\end{enumerate}	
	\end{multicols}

	\item Над какими точками может проходить Солнце в течение его суточного движения?
	\begin{multicols}{2}
		\begin{enumerate}[label=\textbf{\Alph*.}]
			\item{Точка Севера} \item{Точка Юга}
			\item{Точка Запада} \item{Точка Востока} 
		\end{enumerate}	
	\end{multicols}

	\item $22$ марта Марс наблюдался в восточной квадратуре. В каком созвездии он мог быть?
	\begin{multicols}{2}
		\begin{enumerate}[label=\textbf{\Alph*.}]
			\item{Рыбы} \item{Близнецы}
			\item{Орион} \item{Телец} 
		\end{enumerate}	
	\end{multicols}

	\item Определите плотность Солнца, если его масса $M = 2\cdot10^{30}$ кг, радиус -- $R = 700000$ км. Ответ дайте в кг/м$^3$.
	\begin{multicols}{2}
		\begin{enumerate}[label=\textbf{\Alph*.}]
			\item{$1400$} \item{$2450$}
			\item{$464$} \item{$9.7\cdot10^{11}$} 
		\end{enumerate}	
	\end{multicols}

	\item Чему равно время между двумя последовательными кульминациями Солнца? Наблюдатель находится в Москве.
	\begin{multicols}{2}
		\begin{enumerate}[label=\textbf{\Alph*.}]
			\item{$12$ часов} \item{$11$ часов $58$ минут}
			\item{$24$ часа} \item{Зависит от даты} 
		\end{enumerate}	
	\end{multicols}

	\item Чему равен объем шара с радиусом $R$?
	\begin{multicols}{2}
		\begin{enumerate}[label=\textbf{\Alph*.}]
			\item{$\dfrac{4}{3}\pi R^2$} \item{$\dfrac{3}{4}\pi R^3$}
			\item{$\dfrac{4}{3}\pi R^3$} \item{$4\pi R^3$} 
		\end{enumerate}
	\end{multicols}

	\item Определите сколько километров должен пройти путешественник вдоль меридиана, чтобы сместиться на $1$ градус по широте?
	\begin{multicols}{2}
		\begin{enumerate}[label=\textbf{\Alph*.}]
			\item{$111$ км} \item{$55$ км}
			\item{$222$ км} \item{$111$ м} 
		\end{enumerate}
	\end{multicols}

	\newpage

	\item Выберите неверные утверждения.
	\begin{multicols}{2}
		\begin{enumerate}[label=\textbf{\Alph*.}]
			\item{Все планеты вращаются вокруг Земли} \item{На Венере возможно наблюдать затмения Солнца ее естественным спутником}
			\item{Период обращения Луны вокруг своей оси меньше периода смены Лунных фаз} \item{Змееносец -- эклиптическое созвездие} 
		\end{enumerate}
	\end{multicols}

	\item Чему равен период геостационарного спутника (спутника, который всегда находится над одной и той же точкой поверхности Земли)?
	\begin{multicols}{2}
		\begin{enumerate}[label=\textbf{\Alph*.}]
			\item{$24$ часа} \item{$23$ часа $56$ минут} 
			\item{$4$ часа} \item{$12$ часов}
		\end{enumerate}
	\end{multicols}
\end{enumerate}
\section*{Третий уровень}
\begin{enumerate}[resume]
	\item Определите, сколько лет лететь до звезды с параллаксом $0.02''$, если корабль летит со скоростью в $10$ раз меньше скорости света.
	\begin{multicols}{2}
		\begin{enumerate}[label=\textbf{\Alph*.}]
			\item{$163.5$ лет} \item{$1635$ лет}
			\item{$50$ лет} \item{$500$ лет} 
		\end{enumerate}	
	\end{multicols}

	\item Определите во сколько раз сильнее Солнце притягивает Марс, чем Юпитер Марс в нижнем соединении (выберите число, максимально близкое к полученному вами ответу).
	\begin{multicols}{2}
		\begin{enumerate}[label=\textbf{\Alph*.}]
			\item{$5862$ раза} \item{$5$ раз}
			\item{$1047$ раз} \item{$574$ раза} 
		\end{enumerate}	
	\end{multicols}

	\item Между какими конфигурацими внешней планеты проходит минимальное время?
	\begin{multicols}{2}
		\begin{enumerate}[label=\textbf{\Alph*.}]
			\item{Западная и  востоная квадратура} \item{Восточная квадратура и противостояние} 
			
			\item{Восточная и западная квадратура} \item{Западная квадратура и противостояние}
		\end{enumerate}	
	\end{multicols}

	\item Венера находится в максимальной элонгации, при этом в вечерней видимости. В какой фазе Луна похожа на Венеру в данный момент?
	\begin{multicols}{2}
		\begin{enumerate}[label=\textbf{\Alph*.}]
			\item{Новолуние} \item{Первая четверть}
			\item{Полнолуние} \item{Третья четверть} 
		\end{enumerate}	
	\end{multicols}

	\item Определите время, за которое происходит радиолокация Луны. Расстояние между центрами Земли и Луны $384~400$ км. Радиус Земли -- $6400$ км, Луны -- $1738$.
	\begin{multicols}{2}
		\begin{enumerate}[label=\textbf{\Alph*.}]
			\item{$2.56$ с} \item{$1.28$ с}
			\item{$0.64$ мин} \item{$1.28$ мин} 
		\end{enumerate}	
	\end{multicols}

	\item Марсианский год больше земного в $1.87$ раз, а год на Венере в $3$ раза меньше Марсианского. Определите сколько длится год на Венере в земных годах.
	\begin{multicols}{2}
		\begin{enumerate}[label=\textbf{\Alph*.}]
			\item{$0.62$ год} \item{$5.61$ год}
			\item{$1.60$ год} \item{$1.00$ год} 
		\end{enumerate}	
	\end{multicols}

	\item На сколько отличаются времена между двумя последовательными верхними кульминациями Солнца и двумя последовательными верхними кульминациями Веги?
	\begin{multicols}{2}
		\begin{enumerate}[label=\textbf{\Alph*.}]
			\item{$3$ мин $56$ с} \item{$2$ мин}
			\item{$12$ часов} \item{Не отличаются} 
		\end{enumerate}	
	\end{multicols}

	\item Каким бывает угловой размер Юпитера при наблюдении с Земли?
	\begin{multicols}{2}
		\begin{enumerate}[label=\textbf{\Alph*.}]
			\item{$21''$} \item{$62''$}
			\item{$15''$} \item{$1'$} 
		\end{enumerate}	
	\end{multicols}

	\item В какое время суток лучше всего наблюдать Меркурий, если его покрыла Луна через $1$ день после новолуния?
	\begin{multicols}{2}
		\begin{enumerate}[label=\textbf{\Alph*.}]
			\item{Утро} \item{День}
			\item{Вечер} \item{Ночь}
		\end{enumerate}	
	\end{multicols}

	\item В какое время суток Луна находится выше всего над горизонтом, если наблюдения происходят весной?
	\begin{multicols}{2}
		\begin{enumerate}[label=\textbf{\Alph*.}]
			\item{Утро} \item{День}
			\item{Вечер} \item{Ночь}
		\end{enumerate}	
	\end{multicols}
\end{enumerate}
\end{document}