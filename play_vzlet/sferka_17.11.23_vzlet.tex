\documentclass[12pt]{article}


\usepackage{baseset}
\usepackage{newcom}
\usepackage{multicol}
\usepackage{enumitem}
\usepackage{tabularx}

\begin{document}
\begin{tabularx}{\textwidth}{Xr}
{\Large \textbf{Сферическая астрономия}} & Взлет, $17.11.2023$ \\
\end{tabularx}
\noindent\rule{\textwidth}{0.4pt}
\begin{enumerate}
        \item Определите высоты верхней и нижней кульминаций звезды Бетельгейзе ($\alpha$ Ori, склонение $7^{\circ}24'$) на широте Долгопрудного $\varphi=56^{\circ}56'$.
		\item Определите высоты верхней и нижней кульминаций звезды Дубхе ($\alpha$ Uma, склонение $61^{\circ}45'$) на широте Долгопрудного $\varphi=56^{\circ}56'$.
		\item Каково склонение звёзд, которые на широте Краснодара ($\varphi= 45^{\circ}$) кульминируют в зените?
		\item Определите склонение звезды, которая в верхней кульминации находится в зените, а в нижней кульминации в надире. Определите широту места наблюдения, если это возможно.
		\item Определите, где Солнце поднимается выше в момент верхней кульминации $22$ июня. В Сингапуре ($1^{\circ} 17'$~с.ш.) или в Куала-Лумпур ($3^{\circ} 08'$~с.ш.).
		\item Определите звездное время в пунктах с географической долготой $2^h13^m23^s$ и $84^{\circ}58'$ в момент, когда в пункте с долготой $4^h37^m11^s$ звезда Кастор ($\alpha$ Близнецов) находится в верхней кульминации. Прямое восхождение Кастора $\alpha=7^h31^m25^s$.
		\item Определите высоту нижней кульминации звезды, у которой в Долгопрудном ($\varphi$=$55^{\circ}56'$) кульминация происходит на высоте $71^{\circ}56'$. Рефракцией пренебречь.
		\item Для наблюдателя в Санкт-Петербурге две звезды наблюдаются на альмукантаратах с разностью высот $20^{\circ}$. При этом одна из звёзд находится в нижней кульминации, а другая - в верхней. Найдите разность склонений этих звёзд.
		\item Звезда Капелла ($\alpha$ Aur, прямое восхождение $\alpha=5^h 16^m 41^s$, склонение $\delta=46^{\circ}00'$) кульминирует строго в зените. Определите высоты обеих кульминаций для звезды Мерак ($\beta$ UMa, $\delta=56^{\circ}17'$) в этом же месте наблюдения.
		\item Верхняя кульминимация звезды Вега ($\alpha$ Lyr, склонение $\delta=38^{\circ} 47'$) происходит на высоте $83^{\circ}18'$. Определите широту места наблюдения. Определите высоту нижней кульминации.
		\item Нижняя кульминация звезды совпадает с горизонтом, а верхняя кульминация с зенитом. Определите широту места наблюдения и склонение звезды.
		\item Определите какая из звезд поднимется выше над горизонтом для наблюдателя на широте $\varphi$=$55^{\circ}56'$. Мерак ($\beta$ UMa, $\delta=56^{\circ}17'$) или Мицар ($\zeta$ Uma, $\delta=54^{\circ}50'$).
		\item Две звезды на широте $\varphi = 23.5^{\circ}$ в верхней кульминации располагаются симметрично относительно зенита. Обе звезды заходящие. На какой минимальной высоте может происходить нижняя кульминация этих звёзд (до какой минимальной высоты может опуститься та из звёзд, которая опускается ниже)? Решение сопроводите чертежом.
		\item Склонение звезды А больше склонения звезды В в $2$ раза. На какой широте верхняя кульминация этих звезд будет происходить на одном альмукантарате, если нижняя кульминация звезды А происходит на горизонте. Рефракцией пренебречь. Наблюдение проводятся в северном полушарии вдали от полюса.
		\item Северное полярное расстояние звезды А равно склонению звезды В. Верхняя кульминация звезды В происходит на той же высоте, что нижняя кульминация звезды А. Будет ли видно звезду В во время ее нижней кульминации, если наблюдатель находится в средней полосе России. 
		\item У одной звезды зенитные расстояния в моменты верхней и нижней кульминации равны $20^{\circ}$ и $30^{\circ}$. А у второй звезды, наблюдаемой в том же месте, высота верхней кульминации $h=80^{\circ}$. Определите высоту нижней кульминации второй звезды.	
\end{enumerate}
\end{document}