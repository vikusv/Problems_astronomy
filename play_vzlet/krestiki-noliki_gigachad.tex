\documentclass[11pt]{article}


\usepackage{baseset}
\usepackage{multicol}
\usepackage{enumitem}
\usepackage{tabularx}

\begin{document}
\begin{center}
	\textbf{\LARGE{Крестики-нолики}}
\end{center}
\begin{tabularx}{\textwidth}{Xr}
	{\textit{Взлёт, игровая смена}} & $20$ ноября $2023$ \\
\end{tabularx}

\noindent\rule{\textwidth}{0.4pt}

\begin{enumerate}[label=\textbf{A\arabic*.}]
	\item Определите склонения для невосходящих и незаходящих звёзд при наблюдении из Мезмая ($\varphi = 44^{\circ}$).
    %невосходящее - 46, незаходящие 46
    \item Одну и ту же звезду наблюдают в Киеве ($\varphi_K = 50^{\circ}27'$) и Петрозаводске ($\varphi_{\text{П}} = 61^{\circ}47$). Насколько будет отличаться высота этой звезды в нижней кульминации при наблюдении из этих городов?
    % менее чем на 11 град 20 '
    \item Предположим, мы наблюдаем двойную систему, состоящую из двух звёзд, массы которых $2$ и $3$ массы Солнца, а период системы равен $4$ года. Определите расстояние между звёздами, а также скорость движения более массивной компоненты.
    % 4.3 ае, 12.8 км/с
    \item Оцените, с какой максимальной угловой скоростью должна иметь возможность поворачиваться антенна дальней космической связи, чтобы она могла сопровождать любой из существующих искусственных спутников Земли при его движении над горизонтом.
    %2.3 градуса в секунду
    \item $11$ февраля $2009$ года на высоте $800$ км над поверхностью Земли столкнулись два спутника: «Космос-2251» и «Iridium 33». В момент столкновения угол между траекториями спутников составлял $60^\circ $. Найдите диапазон возможных значений относительной скорости спутников при столкновении. 
    %8,11 км/с
\end{enumerate}

\begin{enumerate}[label=\textbf{B\arabic*.}]
	\item Отличник Ваня отпраздновал свой $12$-ый день рождения $12$ апреля в воскресение. В том же году $12$ августа был максимум метеорного потока Персеиды. Каким днем недели было $12$ августа? %Среда
    \item Если первое января -- понедельник, то каким днём недели закончится простой и високосный год? %понедельник и вторник
    \item Существовала гипотеза, что астероиды образовались после взрыва некоторой планеты в Солнечной системе. Сколько примерно астероидов могло бы образоваться из Луны, если предположить, что все получившиеся астероиды имеют диаметр $1$ км? Радиус Луны $1740$ км. Считать, что все получившиеся астероиды имеют средние плотности, равные средней плотности Луны. %4.4 * 10^10
    \item Малые спутники двух планет обращаются по своим круговым орбитам с одинаковой линейной скоростью, но период обращения одного спутника вдвое больше периода обращения другого. Как соотносятся массы двух планет? % относятся как 1:2
    \item Определите широту и склонение наблюдаемой в северном полушарии звезды, если верхняя кульминация наблюдается на высоте $75^{\circ}$, а высота нижней кульминации в $3$ раза меньше % широта 50, склонение 65. Широта 65, склонение 50 + южное полушарие и отриц склонения
\end{enumerate}

\begin{enumerate}[label=\textbf{C\arabic*.}]
	\item $22$ июня в солнечный полдень наблюдатель, стоящий вертикально на ровной поверхности, обнаружил, что его тень имеет длину, равную его росту. На какой широте располагался наблюдатель? % - 21.5, + 68.5
    \item Верхняя кульминация светила происходит на высоте $70^{\circ}$, а нижняя кульминация на высоте $40^{\circ}$. Определите широту места наблюдения. %ϕ = ±55◦ и ϕ = ±75◦ 
    \item Крабовидная туманность появилась в результате вспышки Сверхновой $1054$ года, расположенной на расстоянии $2$ кпк от Солнца. Сейчас ее угловой диаметр равен $6'$. Оцените среднюю скорость, с которой края туманности удалялись от места вспышки. %1 700 км/с
    \item\textbf{A1}. Общая масса пыли в некоторой спиральной галактике, похожей на нашу, $M = 10^8M_{\oplus}$. Примерные размеры галактики таковы: диаметр диска $D = 30$ кпк, толщина диска $h = 400$ пк, характерный диаметр гало $D_{\text{г}} = 100$ кпк, а диаметр
    балджа -- $D_{\text{б}} = 1$ кпк. Определите для всего объёма диска среднюю концентрацию $n$ (в единицах число частиц/м$^3$) и среднюю плотность $\rho$ (в единицах кг/м$^3$) пыли. Для справки: $M_{\odot} = 2\cdot10^{30}$ кг, средний радиус пылинки $r = 0.1$ мкм, а
    плотность её вещества $\rho = 3 000$ кг/м$^3$, $1$ пк = $3.08\cdot10^{16}$ м. %n = 2 · 106 м3, ρ = 2.4 · 10−23 кг/м3
    \item\textbf{A1} Некое светило видно в Санкт-Петербурге ($\varphi = 60^{\circ}$) в зените. Зайдет ли оно за горизонт? %Нет
\end{enumerate}

\begin{enumerate}[label=\textbf{D\arabic*.}]
	\item Оцените скорость метеора, угловая длина пути которого на небе составила $30^{\circ}$, если известно, что он загорелся в зените на высоте $100$ км над поверхностью Земли и пролетел весь путь за $1$ секунду. %от 50 до 57.7 км/с 
    \item В исламском лунном календаре год состоит из $12$ лунных месяцев, половина из которых состоит из $29$ дней, половина -- из $30$ дней. За $30$ лет в календарь вставляется 11 високосных дней. Определить, за какой промежуток времени в лунном календаре набежит лишний год по сравнению с григорианским календарем. %Разница между исламским лунным и григорианским календарем составит целый год по прошествии 32.58 лет по григорианскому календарю или, соответственно, 33.58 лет по лунному календаю
    \item Вычислите минимально возможное время, за которое сигнал от межпланетной станции, находящейся на поверхности астероида, достигнет Земли. Приведите рисунок, на котором покажите взаимное положение Земли, Солнца и станции. Радиус орбиты астероида $2.4$ а.е., скорость света равна $300~000$ км/с.
    \item Определите период обращения термозащищенного спутника находящегося прям над поверхностью Бетельгейзе. Радиус Бетельгейзе $\approx 764 R_{\odot}$. Масса -- $19 M_{\odot}$. %1.55 года
    \item Определите период вращения Бетельгейзе вокруг своей оси. Линейная скорость вращения звезды на экваторе -- $5.5$ км/c. %19.4 года
\end{enumerate}

\begin{enumerate}[label=\textbf{E\arabic*.}]
	\item Определите скорость вращения наблюдателя, который видит Солнце в зените ровно раз в год. %0.42 кс/с
    \item Параллакс некоторой двойной системы равен ее угловому размеру. Определите параллакс системы. %1"
    \item Найдите верхние и нижние кульминации Веги ($\delta = 38^{\circ} 47'$) и Дубхе ($\delta =  61^{\circ} 45'$), для наблюдателя в Ноябрьске ($\varphi = 63^{\circ} 12'$ с.ш., $ \lambda = 75^{\circ} 27'$ в.д.).
    %53гр 20',24 гр 14' - Вега, 48 гр 39', 77гр 45' - дубхе
    \item Верхние кульминации двух далеких звезд происходят одновременно, при этом звезды располагаются симметрично относительно зенита. Во время нижней кульминации эти звезды располагаются симметрично относительно горизонта. Определите широту места наблюдения. Атмосферную рефракцию не учитывать. %$45^\circ$
    \item Исскуственный спутник Земли обращается по круговой орбите на высоте $1500$ км. Определите его скорость. %7.12 км/с
\end{enumerate}
\end{document}