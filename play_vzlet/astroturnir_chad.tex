\documentclass[11pt]{article}


\usepackage{baseset}
\usepackage{multicol}
\usepackage{enumitem}
\usepackage{tabularx}

\begin{document}
\begin{center}
	\textbf{\LARGE{Астротурнир $\mathbf{\alpha}$ $\mathbf{\beta}$}}
\end{center}
\begin{tabularx}{\textwidth}{Xr}
	{\textit{Взлёт, игровая смена}} & $22$ ноября $2023$ \\
\end{tabularx}

\noindent\rule{\textwidth}{0.4pt}

\section*{Сферическая астрономия}
\begin{enumerate}
	\item Переведите угол в часовую меру $46^{\circ} 46' 45''$.
    \item Определите широту, на которой верхняя кульминация звезды и нижняя проходят по одну сторону от зенита, при этом зенитное расстояние в верхней кульминации равно высоте в нижней кульминации.
    \item Определите расстояние между Москвой ($\varphi = 35^{\circ}$ с.ш., $\lambda = 37.5^{\circ}$ в.д.) и Канберой ($\varphi = 35.3^{\circ}$ ю.ш., $\lambda = 149.1^{\circ}$ в.д.) по поверхности Земли.
    \item Определите, какого числа Солнце кульминирует в $12$ часов по звездному времени.
    \item На некоторой широте Вега ($\delta = 38^{\circ} 47'$) кульминирует в зените. Определите высоту кульминации для Сириуса ($\delta = -16^{\circ} 42'$).
    \item Определите высоту Солнца над горизонтом 23 июня в Долгопрудном ($\varphi = 56^{\circ},~\lambda = 37.5^{\circ}$). В каком городе Солнце кульминирует на горизонте в этот день?
    \item Определите склонения звезд, которые являются восходящими на широте $\varphi_0 = 35^{\circ}$.
    \item Определите широты, на которых звезда, кульминирующая в зените, также кульминирует на горизонте. Чему равно склонение такой звезды?
    \item Зенитное расстояние верхней кульминации некоторой звезды равно $30^{\circ}$, а полярное расстояние этой звезды $65^{\circ}$. Определите широту места наблюдения.
    \item В каком случае звезда может наблюдаться и в зените, и в надире?
    \item У некоторой звезды высота верхней кульминации равна $40^{\circ}$, а в нижней в два раза меньше. Определите склонение звезды. Какой звездой является эта звезда?
    \item В этом году день рождения Кати пришелся на понедельник. Когда в следующий раз такое произойдет?
    \item Некоторый год начался во вторник. Определите, в какой день недели начнется год, который будет через $8$ лет.
\end{enumerate}

\section*{Небесная механика}
\begin{enumerate}[resume]
	\item О далекой планетарной системе известно, что вокруг центральной звезды вращаются две экзопланеты, радиусы орбит которых отличаются в 2 раза. Определите во сколько раз различаются скорости этих планет. Орбиты планет считать круговыми.
    \item Определите, чему равен параллакс двойной системы, если ее угловой размер $0.5''$, а расстояние между двумя компонентами равно $5$ а.е.
    \item Оцените, в каком случае максимальное угловое расстояние от Солнца больше: Венеры в элонгации для земного наблюдателя или Земли для наблюдателя на Марсе?
    \item Большая полуось астероида равна $3.95$ ае. Каким может быть расстояние между астероидом и Марсом , когда он оба в квадратурах при наблюдении с Земли?
    \item Оцените высоту гелиостационарной орбиты для Солнца. Период вращения Солнца на экваторе составляет примерно $25$ суток.
    \item Определите время между квадратурами Юпитера.
    \item Синодический период некоторого астероида равен $500$ дней для наблюдателя на Марсе. Определите период астероида, если у него обратное вращение. Орбиты считать круговыми.
    \item Определите за какое время произойдет радиолокация Венеры, если она находится в элонгации для наблюдателя на Земле.
    \item Как часто Юпитер оказывается в соединении для наблюдателя на Марсе? А Венера?
\end{enumerate}

\section*{Луна}
\begin{enumerate}[resume]
	\item Определите, сколько раз за сутки Земля восходит для наблюдателя на Луне, который находится неподалеку от кратера Тихо. Ответ поясните.
    \item $21$ сентября Луна кульминировала ровно в полночь. Определите, в какое время будет ее следующая кульминация.
    \item Определите плотность Луны.
\end{enumerate}

\section*{Углы и параллаксы}
\begin{enumerate}[resume]
	\item Определите, каков должен быть радиус планеты, чтобы ее горизонтальный параллакс был вдвое больше ее углового размера.
    \item Угловой диаметр Марса в противостоянии $25''$, горизонтальный параллакс $23''$. Определите радиус Марса.
    \item Определите горизонтальный параллакс Юпитера в западной квадратуре, если он движется по круговой орбите с большой полуосью $5.2$ а.е.
    \item Определите расстояние до звезды, годичный параллакс которой в $100$ раз меньше суточного параллакса Солнца.
    \item Определите суточный параллакс для телескопа Джеймс Вебб, если расстояние до телескопа $1.5 \cdot 10^6$ км.
\end{enumerate}
\end{document}