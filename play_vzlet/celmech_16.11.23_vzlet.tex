\documentclass[12pt]{article}


\usepackage{baseset}
\usepackage{newcom}
\usepackage{multicol}
\usepackage{enumitem}
\usepackage{tabularx}

\begin{document}
\begin{tabularx}{\textwidth}{Xr}
{\Large \textbf{Небесная механика}} & Взлет, $16.11.2023$ \\
\end{tabularx}
\noindent\rule{\textwidth}{0.4pt}
\begin{enumerate}
    \item Определите, за какое время свет от Солнца достигает Земли, если расстояние между Солнцем и Землей составляет $150\cdot10^6$~км?
    \item Звездная система $61$ Лебедя находится от Солнца на расстоянии $11.36$ световых лет или $3.48$ парсека, и приближается к нам со скоростью $64$ км/с. Определите, за какое время до нее может долететь космический аппарат со скоростью $16$ км/с. Как изменится ответ, если удастся разогнать космический аппарат до вдвое большей скорости? Скорость света равна $300~000$~км/с.
    \item С какой линейной скоростью движется точка на экваторе относительно оси вращения? Радиус Земли $R_{\oplus}=6~400$ км. 
    \item С какой линейной скоростью движется Санкт-Петербург (широта $60^{\circ}$) относительно оси вращения?
    \item Период обращения Юпитера вокруг Солнца составляет $12$ лет. Определите большую полуось орбиты Юпитера.
    \item Среднее расстояние Марса от Солнца составляет $228$~млн.~км. Определите период обращение Марса вокруг Солнца.
    \item Астероид обращается вокруг Солнца по круговой орбите за $8$ лет. Чему равен радиус его орбиты?
    \item Во времена Советского Союза летчикам, налетавшим миллион километров, выдавался специальный значок. За какое время обычный житель Земли пролетает $1$ миллион километров вместе с Землей вокруг Солнца?
    \item Найдите длину пути, который пройдет МКС относительно Земли за календарный год.
    \item Каков период обращения Луны вокруг Земли?
\end{enumerate}
\end{document}