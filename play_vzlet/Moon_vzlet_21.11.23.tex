\documentclass[12pt]{article}


\usepackage{baseset}
\usepackage{multicol}
\usepackage{enumitem}
\usepackage{tabularx}

\begin{document}
\begin{tabularx}{\textwidth}{Xr}
{\Large \textbf{Луна}} & Взлет, $21.11.2023$ \\
\end{tabularx}
\noindent\rule{\textwidth}{0.4pt}
\begin{enumerate}
		\item Почему «растущую» Луну обычно видно вечером? 
		\item Наблюдатель видел серп Луны в $10$ часов утра. Луна была растущая, убывающая или находилась в полнолунии?
		\item Какие планеты могут располагаться на небе рядом с полной Луной? 
		\item Определите на сколько градусов смещается Луна относительно звезд за одни солнечные сутки, если период обращения Луны вокруг Земли $27.32$ суток.
		\item Сколько раз в течение года можно увидеть все фазы Луны?
		\item С каким периодом меняются фазы Луны? Выразите ответ в земных и лунных сутках.
		\item Человек идет на запад и видит прямо перед собой Луну в первой четверти. Если человек при этом посмотрит на часы, какое примерно время они ему покажут?
		\item Какие из перечисленных явлений можно наблюдать на Луне: метеоры, кометы, солнечные затмения, полярные сияния? Ответ поясните.	
		\item Определите угловой радиус и угловой размер Луны в полнолунии.
		\item Сколько оборотов сделает Луна вокруг Земли за земной год? А вокруг своей оси.
		\item В одной точке Земли Луна восходит, а в другой заходит. Каково расстояние между этими точками по поверхности Земли?
		\item На каком расстоянии от центра Земли находится центр масс системы Земля-Луна? На каком удалении от центра масс может оказаться наблюдатель на поверхности Земли?
		\item Сколько времени длится восход Солнца на экваторе Луны?
		\item Посчитайте, с какой скоростью двигается граница день-ночь по экватору Луны. В каком направлении происходит это движение?
		\item Вечером, во время захода Солнца, любитель астрономии разглядывает в телескоп кратер на Луне, находящийся на границе светлой и темной частей диска Луны. А что происходит в этот момент в этом кратере -- заход или восход Солнца? Почему?
		\item Рассчитайте плотность Луны. Радиус Луны равен $1738$ км, а масса -- $7.3\cdot10^{22}$ кг.
\end{enumerate}
\end{document}