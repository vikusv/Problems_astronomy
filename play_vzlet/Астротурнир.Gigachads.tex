\documentclass[a4paper,12pt]{article}
\usepackage{baseset_Katya}
\usepackage[bottom= 2 cm, top= 2 cm, right= 1.5 cm, left= 1.5 cm]{geometry}

\begin{document}
	
	\begin{minipage}{0.15 \textwidth}
		\includegraphics[width= \textwidth]{Logo.png}
	\end{minipage}
	\begin{minipage}{0.85 \textwidth}
		\begin{center}
			\begin{Huge}
				Астротурнир
			\end{Huge}
			
			\bigskip
			
			\begin{Large}
				\textit{Московская область} \hspace{25 pt}
				\textit{$22$ ноября $2023$}
			\end{Large}
			
		\end{center}
	\end{minipage}
	
	\bigskip
	
	\hrule
	
\section*{Сферическая астрономия}

\subsection*{1.}

Переведите угол в часовую меру $46^{\circ} 46' 45''$.


\subsection*{2. }

Определите во сколько раз отличаются зенитные расстояния звезды  со склонением $30^{\circ}$ при наблюдении на широте $60^{\circ}$ в верхней и нижней кульминациях.


\subsection*{3. }

Определите расстояние между Москвой ($\varphi = 35^{\circ}$ с.ш., $\lambda = 37.5^{\circ}$ в.д.)и  Канберой ($\varphi = 35.3^{\circ}$ ю.ш., $\lambda = 149.1^{\circ}$ в.д.).

\subsection*{4. }

Определите склонения звезд, которые являются восходящими на широте $\varphi_0 = 35^{\circ}$.

\subsection*{5. } Зенитное расстояние верхней кульминации некоторой звезды равно $30^{\circ}$, а полярное расстояние этой звезды $65^{\circ}$. Определите широту места наблюдения.

\subsection*{6. }

Определите координаты звезд (прямое восхождение и склонение), которые кульминируют с зенитным расстоянием $\varphi = 30^{\circ}$ в момент восхода точки осеннего равноденствия для наблюдателя в пункте с географической широтой $\varphi = 60^{\circ} 30'$ с.ш.. %%c муника


\subsection*{7. }

В каком случае звезда может наблюдаться и в зените, и в надире?

\subsection*{8. }

В некоторый момент звезда со склонением $30^{\circ}$ находилась в кульминации
для наблюдателя в Санкт-Петербурге ($\varphi = 60^{\circ}$). В тот же момент вторая звезда оказалась также в кульминации, причем сумма высот звезд составила $125^{\circ}$. Определите склонение второй звезды. %421

\subsection*{9. }

У некоторой звезды высота верхней кульминации равна $40^{\circ}$, а в нижней в два раза меньше. Определите склонение звезды. Какой звездой является эта звезда?


\subsection*{10.}

Самолет на высоте $h=10$~км пролетает над Сингапуром (широта $\varphi \approx 0^{\circ}$) в день весеннего равноденствия. Пассажиры видят восход Солнца. Через какое время восход Солнца увидят жители Сингапура?

\section*{Небесная механика}

\subsection*{11. } 

Определите во сколько раз отличается вес человека на экваторе и на полюсе на Юпитере.

\subsection*{12. }

Определите плотность планеты, которую можно облететь за $15$ минут с выключенными двигателями.


\subsection*{13. }

Станция управления полетами, расположенная на Земле, наблюдает за спутником Сатурна Титаном. В момент захода Титана за диск Сатурна космический аппарат на орбите Титана отправляет световой сигнал на Землю. Определите, с какой временной задержкой должен быть отправлен ответ от станции управления полетами, чтобы спутник получил ответ сразу, после того, как выйдет из-за диска планеты. Орбиты Земли, Сатурна и Титана считать круговыми и лежащими в одной плоскости. Сатурн в момент наблюдений находится в противостоянии.

\subsection*{14.}

Между восточной квадратурой и последующей западной квадратурой некоторой планеты проходит в $1.143$ раз больше времени, чем между ее западной и последующей восточной квадратурой. Что это за планета? Орбиты планет считать круговыми. 

\subsection*{15.}

В некоторый момент времени астероид, находившийся в квадратуре на расстоянии $2$ ае от Земли, оказался рядом с кометой, скорость которой была равна второй космической. Определите скорость кометы. 

\subsection*{16.}

Планеты Венера и Юпитер вступают в соединение друг с другом для наблюдателя на Земле, имея одинаковые экваториальные угловые размеры. Чему равно угловое расстояние между Венерой и Солнцем в этот момент? %p=43^{\circ}

\subsection*{17.}

Две звезды с массами $1M_{\odot}$ и $1.4M_{\odot}$ вращаются вокруг общего центра масс с периодом $1$~год. Определите угловые размеры обеих звезд для наблюдателя, находящегося в центре масс.
Первая звезда является звездой, полностью похожим на Солнца, а вторая звезда белый карлик с радиусом $10~000$ км.

\subsection*{18.}

Синодический период астероида при его ретроградном движении в два
раза меньше в случае, если бы он вращался сонаправленно с Землёй. Определите
его большую полуось  %%1043

\subsection*{19. }

Определите, на каких внутренних планетах Солнечной системы день длится больше,
чем год (период обращения планеты вокруг Солнца). %1078

\section*{Луна}

\subsection*{20.}

Определите сколько раз за сутки Земля восходит для наблюдателя в кратере Тихо. Ответ поясните.

\subsection*{21.}

$21$ сентября Луна кульминировала ровно в полночь. Определите в какое время будет ее следующая кульминация.

\subsection*{22.}

Определите в каких пределах может изменяться угловой размер Солнца для наблюдателя на Луне. Орбиту Луны считать круговой.

\subsection*{23.}

Определите на какую максимальную и минимальную высоту может подниматься Луна в верхней кульминации на широте $15^{\circ}$. Считайте, что Луна двигается в плоскости эклиптики.

\section*{Время и календарь}

\subsection*{24.} 

В этом году День рождение Маши пришлось на понедельник. Когда в следующий раз такое произойдет?  %2028

\subsection*{25.} 

Календарный год начался с понедельника, а закончился вторником. Каким днём недели закончится следующий календарный год? %210

\subsection*{26.} 

Определите в какое звездное время кульминирует Солнце $27$ ноября. Считайте, что орбита Земли круговая.

\subsection*{27.}

Определите координаты звезды, которая восходит в точке с азимутом $270^{\circ}$ в $15$ часов по звездному времени.

\section*{Углы и параллаксы}

\subsection*{28.}

Некоторая планета вращается вокруг звезды, масса которой равна $2M_{\odot}$, при этом при наблюдении с земли угловой радиус ее орбиты равен $5''$. Определите период ее обращения вокруг центральной звезды.

\subsection*{29.}

Определите в каком диапазоне лежит суточный параллакс телескопа Джеймс Вебб, если расстояние до телескопа $1.5 \cdot 10^6$ км.

\subsection*{30.}

Определите горизонтальный параллакс Меркурия и его угловой размер в
момент прохождения Меркурия по диску Солнца. Орбиту Меркурия считать круговой. %783


	
	
	
\end{document}