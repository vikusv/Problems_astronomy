\documentclass[12pt]{article}
\usepackage{baseset}
\usepackage{myproblem}
\usepackage{stackengine}
\DeclareSymbolFont{operators}{OT1}{ntxtlf}{m}{n}
\SetSymbolFont{operators}{bold}{OT1}{ntxtlf}{b}{n}
\usepackage{wasysym}
\newcommand{\RomanNumeralCaps}[1]
{\MakeUppercase{\romannumeral #1}}
\usepackage{tabularx}
\usepackage{enumitem}

\begin{document}
 \begin{tabularx}{\textwidth}{Xr}
  {\Large \textbf{Сферка. Повторение}} & Дедлайн -- $12.02.2024$ \\
 \end{tabularx}
 \noindent\rule{\textwidth}{0.4pt}
 \begin{enumerate}
        \item Cформулируйте условие, при котором астрономический азимут светила никогда не будет равен нулю.
        \item На какой высоте в Салехарде ($\varphi = 66^{\circ} 38'$) наблюдаются верхняя и нижняя кульминация Веги ($\alpha$ Lyr, $\delta=38^{\circ} 47'$)? А Бенетнаш ($\eta$ UMa, $\delta =49^{\circ} 19'$)?
        \item Звезда Капелла ($\alpha$ Aur, прямое восхождение $\alpha=5^h 16^m 41^s$, склонение $\delta=45^{\circ}59'53''$) кульминирует строго в зените. Какая из звезд, Вега или Мицар будет кульминировать выше? Склонение Веги $\delta=38^{\circ} 47'$. Склонение Мицара $\delta=54^{\circ}50'$.
        \item Верхние кульминации двух далеких звезд происходят одновременно, при этом звезды располагаются симметрично относительно зенита. Во время нижней кульминации эти звезды располагаются симметрично относительно горизонта. Определите широту места наблюдения. Атмосферную рефракцию не учитывать.
        \item У одной звезды зенитные расстояния в моменты верхней и нижней кульминации равны $20^{\circ}$ и $30^{\circ}$. А у второй звезды, наблюдаемой в том же месте, высота верхней кульминации $h=80^{\circ}$. Определите высоту нижней кульминации второй звезды.
        \item Определите широты, на которых созвездие Золотая Рыба (диапазон склонений $-70^{\circ}$ до $-49^{\circ}$) является полностью незаходящим.
        \item Определите высоту верхней и нижней кульминации Солнца в Мурманске ($\varphi=68^{\circ}$) $21$ июня, Санкт-Петербурге ($\varphi=60^{\circ}$) $23$ сентября и Владивостоке ($\varphi=43^{\circ}$) $22$ декабря?
        \item В какое время года Луна в полнолуние поднимается выше всего над горизонтом?
        \item Любители астрономии наблюдали планеты и обнаружили, что Юпитер кульминировал в $6$ часов вечера по местному времени на высоте $15^{\circ}$, а Марс – в $6$ часов утра по местному времени на высоте $62^{\circ}$. В какой сезон года и на какой широте проводились наблюдения?
        \item Определите, в какой день Сириус $\alpha=6^h 43^m$ кульминирует ровно в истинную полночь.
 \end{enumerate}
\end{document}