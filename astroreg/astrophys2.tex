\documentclass[12pt]{article}
\usepackage{baseset}
\usepackage{myproblem}
\usepackage{stackengine}
\DeclareSymbolFont{operators}{OT1}{ntxtlf}{m}{n}
\SetSymbolFont{operators}{bold}{OT1}{ntxtlf}{b}{n}
\usepackage{wasysym}
\newcommand{\RomanNumeralCaps}[1]
{\MakeUppercase{\romannumeral #1}}
\usepackage{tabularx}
\usepackage{enumitem}

\begin{document}
 \begin{tabularx}{\textwidth}{Xr}
  {\Large \textbf{Закон Стефана-Больцмана. Закон смещения Вина}} & Дедлайн -- $11.12.2023$ \\
 \end{tabularx}
 \noindent\rule{\textwidth}{0.4pt}
 \begin{enumerate}
        \item Какова должна быть температура звезды, если при одинаковых с Солнцем размерах ее светимость в $81$ раз больше.

        \item Звезда Фомальгаут имеет видимую звездную величину $1.16^m$ и параллакс $0.130''$. Определите радиус звезды, если температура ее поверхности $8~590$~K.

        \item Звезда Денеб ($\alpha$ Лебедя) имеет температуру $8~500$ K и радиус $210 R_{\odot}$. Определите светимость Денеба.

        \item Две звезды имеют одинаковые светимости,	но температура первой звезды $3~500$~К, а температура второй звезды $8~000$~К. Определите соотношение радиусов двух звезд.

        \item Некоторую звезду разбили на $8$ звезд такой же плотности и температуры. Определите, во сколько раз увеличилась суммарная яркость.

        \item Наблюдение покрытия звезды $4^m$ Луной (поглощение атмосферы считайте учтенным) позволило определить ее угловой диаметр: $10^{-3}$ угловых секунд. Чему равна эффективная температура звезды?

        \item Угловой диаметр звезды Бетельгейзе составляет $0.047''$, а ее болометрическая звездная величина $-2^m$. Определите эффективную температуру Бетельгейзе.

        \item Две звезды имеют одинаковые угловые диаметры, расстояние до них неизвестно. Их температуры различаются в $3$ раза. Найдите разницу болометрических звёздных величин звезд.

        \item Две звезды имеют одинаковые массы и светимости, но поверхность одной из них вдвое горячее поверхности второй. У какой из звезд средняя плотность больше? Во сколько раз?

        \item Во сколько раз красный гигант больше красного карлика, если их светимость отличается в $10^8$ раз?
 \end{enumerate}
\end{document}