\documentclass[12pt]{article}
\usepackage{baseset}
\usepackage{myproblem}
\usepackage{stackengine}
\DeclareSymbolFont{operators}{OT1}{ntxtlf}{m}{n}
\SetSymbolFont{operators}{bold}{OT1}{ntxtlf}{b}{n}
\usepackage{wasysym}
\newcommand{\RomanNumeralCaps}[1]
{\MakeUppercase{\romannumeral #1}}
\usepackage{tabularx}
\usepackage{enumitem}

\begin{document}
 \begin{tabularx}{\textwidth}{Xr}
  {\Large \textbf{Объекты Солнечной системы на небесной сфере}} & Дедлайн -- $4.03.2024$ \\
 \end{tabularx}
 \noindent\rule{\textwidth}{0.4pt}
 \begin{enumerate}
        \item В день зимнего солнцестояния высота нижней кульминации Марса в некотором пункте Земли составила $-34^{\circ}15'$, а верхней кульминации -- $34^{\circ}15'$. Определите склонение Марса и расстояние до него. Орбиты всех планет считайте круговыми и лежащими в одной плоскости. Какова широта места наблюдения
        \item На какой максимальной высоте над горизонтом можно увидеть Меркурий невооруженным глазом? Как это зависит от времени года. Предположите, что Меркурий становится видно на сумеречном небе при погружении Солнца под горизонт на $6^{\circ}$.
        \item В $2003$ году Юпитер вступил в противостояние в начале февраля. Как в этом месяце день ото дня изменяется его максимальная высота над горизонтом на широте Москвы?
        \item Луна в первой четверти наблюдалась $21$ декабря в $18^h 00^m$ истинного солнечного времени на меридиане на высоте $60^{\circ}$ градусов. Определите широту места наблюдений. Наклоном плоскости Луны к эклиптике пренебречь. Можно ли в этот момент увидеть невооруженным глазом туманность Андромеды ($\alpha=0^h 42^m$, $\delta=+41^{\circ}$)?
        \item Новолуние произошло $1$ февраля. Определите в какой день февраля Луна кульминировала выше всего. Предположите, что плоскость орбиты Луны совпадает с эклиптикой. Оцените высоту этой кульминации в городе Долгопрудный ($\varphi=55^{\circ}56'$).
        \item Сатурн в ночь на $23$ сентября кульминировал в полночь в неком городе на высоте $45^{\circ}$. На какой высоте будет кульминировать Сатурн в католическое рождество ($25$ декабря) и сколько в этот момент покажут часы? Наклоном орбиты Сатурна к эклиптике пренебречь.
        \item $7$ мая Венера оказалась в наибольшей восточной элонгации и в каком-то пункте Земли стала незаходящим светилом, ее нижняя кульминация произошла в точке севера. Где в этот момент находилось Солнце?
        \item $1$ мая наступило противостояние Марса. В некоторой точке Земли в момент противостояния Солнце и Марс одновременно взошли над горизонтом. Найдите широту данной точки и определите, над какими сторонами горизонта располагались Солнце и Марс. Наклоном плоскости орбиты Марса к эклиптике и рефракцией пренебречь.
        \item Находясь в северном полушарии, мы $22$ декабря наблюдаем парадоксальное явление: планета Венера, находясь в точке наибольшей элонгации, восходит точно на юге. В каких широтах мы находимся, и какая элонгация у Венеры -- восточная или западная? Где в это время находилось Солнце?
        \item Определите широту места наблюдения, если каждые звёздные сутки Солнце наблюдается на горизонте?
 \end{enumerate}
\end{document}