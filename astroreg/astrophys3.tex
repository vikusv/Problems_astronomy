\documentclass[12pt]{article}
\usepackage{baseset}
\usepackage{myproblem}
\usepackage{stackengine}
\DeclareSymbolFont{operators}{OT1}{ntxtlf}{m}{n}
\SetSymbolFont{operators}{bold}{OT1}{ntxtlf}{b}{n}
\usepackage{wasysym}
\newcommand{\RomanNumeralCaps}[1]
{\MakeUppercase{\romannumeral #1}}
\usepackage{tabularx}
\usepackage{enumitem}

\begin{document}
 \begin{tabularx}{\textwidth}{Xr}
  {\Large \textbf{Звездные величины объектов Солнечной системы}} & Дедлайн -- $17.12.2023$ \\
 \end{tabularx}
 \noindent\rule{\textwidth}{0.4pt}
 \begin{enumerate}
        \item В эпоху среднего противостояния Марса его спутники видны с Земли звездообразными объектами $11.6^m$ (Фобос) и $12.8^m$ (Деймос). Какие примерно угловые размеры и каков блеск спутников в полной фазе по наблюдениям с Марса, если средний диаметр Фобоса равен $21$ км, а диаметр Деймоса -- $12$ км, и они обращаются вокруг планеты соответственно на расстояниях в $9~400$ км и $23~500$ км?
        \item Используя данные предыдущей задачи и эксцентриситет марсианской орбиты, равный $0.0934$, вычислить блеск спутников Марса при его наиболее далеком (афелийном) противостоянии и при наиближайшем (перигелийном) соединении.
        \item В момент каждого противостояния астероида земной наблюдатель измеряет его видимую звездную величину. Период обращения астероида равен $3.9$ года. Оцените эксцентриситет его орбиты, если амплитуда изменения видимой звездной величины составляет $2.5^m$. Орбиту Земли считаем круговой.
        \item Астероид радиусом $50$ метров в некоторый момент времени находился на расстоянии $0.866$~а.е. от Солнца и при наблюдении с Земли угол между астероидом и Солнцем составлял $60^{\circ}$. Оцените видимую звездную величину астероида в этот момент. Возможно ли наблюдать его в телескоп с диаметром объектива $50$ см? Оптические свойства поверхности астероида такие же, как у Луны.
        \item Транснептуновый объект $(174567)$ Варда в настоящее время имеет видимую звездную	величину $21^m$ (при наблюдении с Земли) и находится на расстоянии $48$~а.е. от Солнца.	Оцените диаметр Варды, если ее поверхность отражает $10\%$ падающего на нее света.
        \item Астероид, по свойствам своей поверхности напоминающий Луну, находится на земном небе на расстоянии $20^{\circ}$ от Солнца. Его горизонтальный параллакс составляет $3''$, а видимая звёздная величина $+14^m$. По этим данным оцените размеры астероида.
        \item В настоящее время ведутся поиски возможной девятой планеты Солнечной системы, которая может иметь диаметр в $10$ диаметров Земли и располагаться в $280$ а.е. от Солнца. Астероид какого диаметра в главном поясе будет иметь такую же яркость на Земле в противостоянии, как и эта планета? Отражательную способность поверхности астероида считать аналогичной лунной, а планеты – аналогичной Нептуну. Оба тела располагаются в плоскости эклиптики.
        \item Масса всех астероидов главного пояса оценивается в $50\%$ массы Луны. Допустим, человечество решило очистить Солнечную систему и собрало их все в один планетоид на расстоянии $3$~а. е. от Солнца. Можно ли будет увидеть эту новую планету невооружённым глазом с Земли? Среднюю плотность и отражательную способность астероидов и получившегося планетоида считать одинаковыми и равными соответствующим величинам для Луны. Для справки: расстояние до Луны равно $384 000$~км, видимая звёздная величина Луны в полнолуние составляет $-12.7^m$.
        \item Наблюдая ночную Калугу марсиане заметили, что во время великих противостояний Земли и Марса ($r=0.38$~а.е.) город Калуга выглядит звездой блеском $m=17^m$. Оцените, какое количество фонарей горит ночью в городе Калуга, если в среднем один калужский фонарь с расстояния $r_0=250$~метров светит как полная Луна (звездная величина полной Луны около $-13^m$).
        \item Определите видимые звездные величины Сатурна в противостоянии и в соединении.
    \end{enumerate}
\end{document}