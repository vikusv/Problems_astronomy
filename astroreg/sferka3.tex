\documentclass[12pt]{article}
\usepackage{baseset}
\usepackage{myproblem}
\usepackage{stackengine}
\DeclareSymbolFont{operators}{OT1}{ntxtlf}{m}{n}
\SetSymbolFont{operators}{bold}{OT1}{ntxtlf}{b}{n}
\usepackage{wasysym}
\newcommand{\RomanNumeralCaps}[1]
{\MakeUppercase{\romannumeral #1}}
\usepackage{tabularx}
\usepackage{enumitem}

\begin{document}
 \begin{tabularx}{\textwidth}{Xr}
  {\Large \textbf{Сферка. Плоское приближение}} & Дедлайн -- $27.02.2024$ \\
 \end{tabularx}
 \noindent\rule{\textwidth}{0.4pt}
 \begin{enumerate}
        \item Наблюдатель в северном полушарии одновременно наблюдает восход звезды \textbf{A1} со склонением $-2^{\circ}$ и восход звезды \textbf{A2} со склонением $+8^{\circ}$. Какая из звезд зайдет за горизонт первой?
        \item Наблюдатель находится в северном полушарии и наблюдает восход звезды \textbf{A} со склонением $-8^{\circ}$, и в это же время заходит звезда \textbf{B} со склонением $+8^{\circ}$. Что произойдет раньше: ближайший заход звезды \textbf{A} или восход звезды \textbf{B}? Рефракцией пренебречь.
        \item Две звезды склонением $\delta=+4.5^{\circ}$ и $\delta=-1^{\circ}$ взошли одновременно для наблюдателя на широте $56^{\circ}$. Определите разницу времен заходов этих звезд. Рефракцией пренебречь.
        \item При наблюдении с широты $+55^{\circ}$ звезда \textbf{A} со склонением $-2^{\circ}$ взошла одновременно со звездой \textbf{B}, а зашла одновременно со звездой \textbf{C}. Чему равна разность прямых восхождений звёзд \textbf{B} и \textbf{C}, если они находятся на небесном экваторе? Рефракцией пренебречь. 
        \item Любитель астрономии, не двигаясь по поверхности Земли, заметил, что заход	Солнца за горизонт продолжался ровно $2.5$ минуты. В каком географическом районе России он находился? Орбиту Земли считать круговой, атмосферной рефракцией пренебречь. Наблюдение происходит в сентябре месяца.
        \item $9$ апреля $2015$ года максимума блеска достигает самая известная долгопериодическая переменная звезда Мира («Удивительная») Кита (прямое восхождение $02^h 19^m$, склонение $-3.0^{\circ}$). На какой максимальной северной широте на Земле ее можно будет увидеть в этот день при погружении Солнца под горизонт не менее $12^{\circ}$? Атмосферное поглощение и рефракцию не учитывать. Орбиту Земли считать круговой.
        \item В момент, когда закончились гражданские сумерки в Долгопрудном (высота Солнца была $-6^{\circ}$) астрономический азимут Солнца оказался равен $90^{\circ}$. Определите среднее солнечное время и дату наблюдения.
        \item $19$ апреля Венера находилась в $30^{\circ}$ к западу от Солнца. Она взошла ровно в момент окончания астрономической ночи. Определите широту места наблюдения и местное солнечное время. Определите фазу планеты.
        \item Рыжая панда Миру устала от всеобщего внимания и улетела на Уран. Там, конечно, холодно, зато спокойно. И восходы красивые... В каких пределах может изменяться продолжительность восхода Солн­ца для Миру, находящейся на условной «поверхности» этого гиганта? Ураноцентрическая широта места наблюдения $\varphi = 10^{\circ}$.
        \item В момент захода Солнца азимут центра его диска был равен $A = 98.0^{\circ}$, а модуль скорости изменения этой величины $b = 12.87'$/мин. Найдите дату наблюдения. Рефракцией пренебречь.
 \end{enumerate}
\end{document}