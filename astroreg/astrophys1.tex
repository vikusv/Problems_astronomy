\documentclass[12pt]{article}
\usepackage{baseset}
\usepackage{myproblem}
\usepackage{stackengine}
\DeclareSymbolFont{operators}{OT1}{ntxtlf}{m}{n}
\SetSymbolFont{operators}{bold}{OT1}{ntxtlf}{b}{n}
\usepackage{wasysym}
\newcommand{\RomanNumeralCaps}[1]
{\MakeUppercase{\romannumeral #1}}
\usepackage{tabularx}
\usepackage{enumitem}

\begin{document}
 \begin{tabularx}{\textwidth}{Xr}
  {\Large \textbf{Освещенность. Звездные величины}} & Дедлайн -- $10.11.2023$ \\
 \end{tabularx}
 \noindent\rule{\textwidth}{0.4pt}
 \begin{enumerate}
        \item Компактное рассеянное звездное скопление состоит из $100$ одинаковых звезд и с трудом видно на небе глазом как маленькое пятнышко. Какой видимый блеск имеет каждая из звезд?

        \item Тройная звезда имеет видимый блеск $4.0^m$. Определите звездную величину третьей компоненты, если первые две компоненты имеют звездные величины $4.7^m$ и $5.1^m$.

        \item Шаровое скопление содержит $10^{6}$ звезд звездной величины $22^m$ и $10 000$ сверхгигантов со звездной величиной $17^m$. Сможем ли мы увидеть это шаровое скопление глазом?

        \item На рисунке приведена кривая блеска  затменно-переменной звезды. Определите по графику блеск компонентов двойной системы.

  \begin{figure}[h] 	
  	\centering
	\begin{tikzpicture} 
	\begin{axis}[xlabel=время,
	ylabel=звездная величина, 
	grid=major, 
	y dir=reverse, 
	width=8cm, 
	legend pos = south west,
	ymax=3.2]
	\addplot[line width=1pt, black] coordinates {
		(0, 2)
		(1, 2)
		(1.9, 2.4)
		(2.1, 2.4)
		(3, 2)
		%(4, 2)
		(4, 2)
		(4.9, 3.0)
		(5.1, 3.0)
		(6, 2)
		(7, 2)
		(7.9, 2.4)
		(8.1, 2.4)
		(9, 2)
		(10, 2)
		%(12, 2)
		(10.9, 3.0)
		(11.1, 3.0)
		(12, 2)
		(13, 2)
	};

	\end{axis}
	\end{tikzpicture} 
\end{figure}

        \item Двойная звезда Капелла имеет звездную величину $0.21^m$, блеск ее компонент отличается на $0.5^m$. Определите блеск каждой компоненты.

        \item Визуально тройная звезда состоит из звезд с видимыми звездными величинами $6^m$, $7^m$ и $8^m$. Расстояния до звезд составляют $10$, $15$ и $20$ парсек соответственно. Наблюдатель пролетел $5$ парсек в сторону этой тройной звезды. Оцените суммарный блеск этой тройной системы для наблюдателя после перелета.

        \item Экзопланета может быть обнаружена транзитным методом (изменение яркости звезды в моменты прохождения планеты по диску звезды), если диск планеты перекроет $1\%$ поверхности звезды. Определите, насколько изменяется звездная величина звезды в такие моменты?

        \item Вычислите абсолютную\footnote{Звездная величина звезды, если бы она находилась на расстоянии $10$ пк. Связь абсолютной $M$ и видимой $m$ звездных величин через расстояние от наблюдателя до звезды $r$: $M = m +5 - 5\lg{r}$. Расстояние необходимо подставлять в парсеках.} звёздную величину Антареса, если его параллакс $\pi = 0.0059''$, а видимая звёздная величина $m = + 0.91^{m}$.

        \item В некотором созвездии расстояние между звёздами Альфа и Бета на небесной сфере составляет $18^{\circ}$, а их звёздные величины равны $2.96^m$ и $3.07^m$ соответственно. Известно, что абсолютные звёздные величины этих звёзд одинаковы. Какую звёздную величину будет иметь звезда Альфа, если смотреть на неё из окрестностей звезды Бета?

        \item Телескопу доступны звёзды до $19^m$. Можно ли с его помощью зарегистрировать шаровое скопление из миллиона звезд, подобных Солнцу, находящееся в галактике на расстоянии $10$ Мпк от нас?
 \end{enumerate}
\end{document}