\documentclass[12pt]{article}
\usepackage{baseset}
\usepackage{myproblem}
\usepackage{stackengine}
\DeclareSymbolFont{operators}{OT1}{ntxtlf}{m}{n}
\SetSymbolFont{operators}{bold}{OT1}{ntxtlf}{b}{n}
\usepackage{wasysym}
\newcommand{\RomanNumeralCaps}[1]
{\MakeUppercase{\romannumeral #1}}
\usepackage{tabularx}
\usepackage{enumitem}

\begin{document}
 \begin{tabularx}{\textwidth}{Xr}
  {\Large \textbf{Звездное время + уравнение времени}} & Дедлайн -- $5.03.2024$ \\
 \end{tabularx}
 \noindent\rule{\textwidth}{0.4pt}
 \begin{enumerate}
        \item Звездное время в Москве ($\lambda=37^\circ,~\phi =56^{\circ}$) оказалось равным $15^h$. Определите через сколько минут звезда Альриша ($\alpha=2^h 2^m$) пройдёт верхнюю кульминацию.
        \item Определите звездное время в пунктах с географической долготой $2^h13^m23^s$ и $84^{\circ}58'$ в момент, когда в пункте с долготой $4^h37^m11^s$ звезда Кастор ($\alpha$ Близнецов) находится в верхней кульминации. Прямое восхождение Кастора $\alpha=7^h31^m25^s$.
        \item  Какое  звездное время в Москве $\lambda=37.5^{\circ}$ $27$ февраля в $16^h 30^m$ по поясному времени?
        \item Звезда Капелла была в верхней кульминации в $16^h 00^m 00^s$. Во сколько будет верхняя кульминация этой звезды через $6$ дней? А через $10$ дней?
        \item Определите, на какой широте и в какой день звезда Вега ($\alpha=18^h 36^m$, $\delta=38^{\circ} 47'$) будет кульминировать в зените в полночь.
        \item В момент кульминации звезды Бетельгейзе ($\alpha$ Ori, $\alpha=5^h 55^m$) хронометр, идущий точно по звездному гринвичскому времени, показывает $15^h 09^m$. Определите долготу места наблюдения.
        \item Определите местное время верхней кульминации туманности Андромеды ($\alpha=00^h 42^m$, $\delta=41^{\circ}16'$) в Долгопрудном ($\lambda=37^{\circ}30'$, $\varphi=55^{\circ}56'$, UTC$+3$) $1$~сентября.
        \item В некоторый момент $7$ февраля Регул ($\alpha = 10^h 09^m$, $\delta = 11^{\circ} 53'$) заходит в Бейруте (широта $\varphi = +33^{\circ} 53'$). Чему равно истинное солнечное время?
        \item Для наблюдателя на экваторе высота Солнца в момент равноденствия равна $30^{\circ}$,	азимут $+90^{\circ}$. Определить среднее солнечное время в данной точке, если уравнение времени $\eta =+7^m$. Рефракцией и суточным параллаксом Солнца пренебречь.
        \item В некоторый день в столице Уганды г.Кампала ($\lambda = 32^{\circ} 14'$) Солнце взошло в $7^h2^m16^s$ и зашло в $19^h8^m29^s$ по поясному времени. Найдите дату события, широту Кампалы и её часовой пояс.
 \end{enumerate}
\end{document}