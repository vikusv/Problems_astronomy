\documentclass[12pt]{article}
\usepackage{baseset}
\usepackage{myproblem}
\usepackage{stackengine}
\DeclareSymbolFont{operators}{OT1}{ntxtlf}{m}{n}
\SetSymbolFont{operators}{bold}{OT1}{ntxtlf}{b}{n}
\usepackage{wasysym}
\newcommand{\RomanNumeralCaps}[1]
{\MakeUppercase{\romannumeral #1}}
\usepackage{tabularx}
\usepackage{enumitem}

\begin{document}
 \begin{tabularx}{\textwidth}{Xr}
  {\Large \textbf{Уравнение энергетического баланса}} & Дедлайн -- $17.12.2023$ \\
 \end{tabularx}
 \noindent\rule{\textwidth}{0.4pt}
 \begin{enumerate}
        \item Определите температуры объектов Солнечной системы. Ответы представьте в виде таблицы.
        \item Определите границы зоны обитаемости для Солнца.
        \item На далекой обитаемой планете тепловые условия аналогичны земным, но местное Солнце имеет втрое меньший угловой диаметр. Найдите температуру этой далекой звезды.
        \item Вблизи звезды HD$209458$ спектрального класса G$0$V (температура $T = 6000$ К) обнаружена планета HD$209458$b с круговой орбитой и парами воды в атмосфере. Угловой радиус этой звезды при наблюдении с данной планеты составляет $6.61^{\circ}$. Найдите сферическое альбедо планеты, если ее эффективная температура $1130$~К.
        \item Вокруг некой звезды $A$ вращается планета, с периодом в $100$ лет. Максимум излучения звезды приходится на $3625$ A, радиус звезды $3~R_{\odot}$, также известно, что атмосферы на планете нет, альбедо планеты $A=0.3$. Определите эффективную температуру планеты. Считайте, что центральная звезда принадлежит главной последовательности.
        \item Вокруг звезды главной последовательности вращается планета с таким же периодом, что и Земля. Альбедо планеты равно $0.5$. Масса звезды равна массе Солнца. Найдите эффективную температуру на планете.
        \item Стандартная теория эволюции звезд утверждает, что $4$ миллиарда лет назад наше Солнце излучало на $30\%$ меньше энергии, чем сейчас. На основании этих данных оцените среднюю температуру на Земле в тот период, если считать, что орбита Земли и строение ее атмосферы в тот момент были в точности такими же, как сейчас.
        \item Определите температуру пылинки радиусом $2$~мкм, расположенную на расстоянии $2.5$~а.е от Солнца. Пылинку считать чернотельной.
        \item Вокруг звезды вращается $5$ планет, причем первая планета находится на расстоянии $1$ а.е. от звезды, вторая -- на $2$ а.е., и так далее. Температура звезды равна $10~000$ К, а радиус -- $5~R_{\odot}$. Альбедо первой планеты -- $0.1$, второй -- $0.2$ и т.д. Определите температуры планет в системе.
        \item (К предыдущей задаче). Находятся ли планеты в зоне обитаемости их звезды? Если да, то какие? 
 \end{enumerate}
\end{document}