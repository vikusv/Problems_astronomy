\documentclass[12pt]{article}
\usepackage{baseset}
\usepackage{myproblem}
\usepackage{stackengine}
\DeclareSymbolFont{operators}{OT1}{ntxtlf}{m}{n}
\SetSymbolFont{operators}{bold}{OT1}{ntxtlf}{b}{n}
\usepackage{wasysym}
\newcommand{\RomanNumeralCaps}[1]
{\MakeUppercase{\romannumeral #1}}
\usepackage{tabularx}
\usepackage{enumitem}

\begin{document}
 \begin{tabularx}{\textwidth}{Xr}
  {\Large \textbf{Сферка. Множественность решений}} & Дедлайн -- $22.02.2024$ \\
 \end{tabularx}
 \noindent\rule{\textwidth}{0.4pt}
 \begin{enumerate}
        \item Верхняя кульминация светила происходит на высоте $60^{\circ}$, а нижняя кульминация на высоте $30^{\circ}$. Определите широту места наблюдения.
        \item В некоторый момент звезда со склонением $70^{\circ}$ находилась в кульминации для наблюдателя в Санкт-Петербурге ($\varphi=60^{\circ}$). В тот же момент вторая звезда оказалась также в кульминации, причем сумма высот звезд составила $110^{\circ}$. Определите склонение второй звезды.
        \item В наблюдательном дневнике астронома записаны наблюдения одной и той же звезды. Зенитное расстояние в нижней кульминации $47^{\circ}35'$ и высота в верхней кульминации $84^{\circ}15'$. Найдите широту места наблюдения.
        \item У одной звезды зенитные расстояния в моменты верхней и нижней кульминации равны $20^{\circ}$ и $30^{\circ}$. А у второй звезды, наблюдаемой в том же месте, высота верхней кульминации $h=80^{\circ}$. Определите высоту нижней кульминации второй звезды.
        \item Звезда \textbf{A} кульминирует на высоте, вдвое большей высоты звезды \textbf{B} в верхней кульминации. Верхняя кульминация звезды \textbf{A} происходит на высоте $85^{\circ}$. На какой высоте происходит нижняя  кульминация звезды \textbf{B}? Наблюдения ведутся на широте $70^{\circ}$ с. ш.
        \item Звезда \textbf{A} заходит точно в точке запада. И ее высота верхней кульминации ровно в два раза меньше высоты верхней кульминации звезды \textbf{B}. Широта места наблюдения $\varphi=45^{\circ}$. Определите, какое время над горизонтом проводят звезды \textbf{A} и \textbf{B} для наблюдателя на этой широте. Чему равны их склонения?
        \item Нижняя кульминация звезды \textbf{A} происходит на той же высоте, что и верхняя кульминация звезды \textbf{A}. Известно, что звезда \textbf{A} восходит точно на востоке, а нижняя кульминация звезды \textbf{A} в $2$ раза ниже её верхней. Найти широту и склонение звезды $A$.
        \item Высота звезды в верхней кульминации $40^{\circ}30'$, а в нижней кульминации ее высота $30^{\circ}40'$. Найдите широту места наблюдения и склонение звезды.
        \item В некоторый момент звезда со склонением $30^{\circ}$ находилась в кульминации для наблюдателя в Санкт-Петербурге ($\varphi=60^{\circ}$). В тот же момент вторая звезда оказалась также в кульминации, причем сумма высот звезд составила $125^{\circ}$. Определите склонение второй звезды. 
        \item В некоторый момент звезда со склонением $\delta_0$ находилась в кульминации для наблюдателя в Санкт-Петербурге ($\varphi=60^{\circ}$). В тот же момент вторая звезда оказалась также в кульминации, причем сумма высот звезд составила $h_{\sum}$. Определите, сколько максимально может быть ответов у этой задачи.
 \end{enumerate}
\end{document}