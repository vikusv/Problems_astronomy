\documentclass[12pt]{article}
\usepackage{baseset}
\usepackage{myproblem}
\usepackage{stackengine}
\DeclareSymbolFont{operators}{OT1}{ntxtlf}{m}{n}
\SetSymbolFont{operators}{bold}{OT1}{ntxtlf}{b}{n}
\usepackage{wasysym}
\newcommand{\RomanNumeralCaps}[1]
{\MakeUppercase{\romannumeral #1}}
\usepackage{tabularx}
\usepackage{enumitem}

\begin{document}
\begin{tabularx}{\textwidth}{Xr}
{\Large \textbf{Сферка}} & ФТЛ $27.12.2023$ \\
\end{tabularx}
\noindent\rule{\textwidth}{0.4pt}
\section*{Эффекты, меняющие координаты светил}
\begin{enumerate}
    \item{На какое время действие рефракции удлиняет продолжительность дня на экваторе?}
    \item{Определите склонения звезд для наблюдателя в Долгопрудном, которые будут являться незаходящими и невосходящими с учетом рефракции.}
    \item{Определите ширину полосы в километрах вдоль Северного полярного круга, в которой бывает полярный день, но не бывает полярной ночи.}
    \item{Параллактический эллипс звезды имеет большую полуось $2$ миллисекунды дуги и эксцентриситет $0.87$. Чему равно склонение звезды, если её прямое восхождение равно $6^h$? Чему равно расстояние до звезды?}
    \item{Укажите, какие из перечисленных ярких звезд можно будет увидеть в Москве ($\varphi=+56^{\circ}$) через $13~000$ лет: Сириус, Канопус, Вега, Капелла, Арктур, Ригель, Процион, Альтаир, Спика, Антарес.} 	
    \item{Координаты апекса Солнца $\alpha=18^h$, $\delta=+30^{\circ}$. Определите координаты апекса Солнца через $13~000$~лет.}
    \item{Некоторая звезда имеет координаты $\alpha = 6^h$, $\delta = 23.5^{\circ}$. Однако, как известно, координаты всех звёзд медленно меняются из-за прецессии земной оси (ось Земли описывает конус за период около $26$ тысяч лет). Какие координаты $(\alpha, \delta)$ будет иметь эта звезда через $6500$ лет?}
    \item{Оцените число звезд, которых можно увидеть невооруженным глазом в Москве в ближайшие $100~000$~лет.}
    \item{В некотором пункте Земли верхний край Солнца виден на горизонте в точке севера. На каких широтах такое возможно? Рельефом Земли в данном пункте пренебречь.}
    \item{Искусственный спутник Земли обращается вокруг Земли по круговой орбите. В каждый момент времени спутник видно ровно с половины Земли. Найдите период обращения спутника. Атмосферная рефракция у горизонта составляет $35'$. Атмосферное поглощение не учитывать.}
    \item{Денеб является полярной звездой для марсианских наблюдателей. Определите, во сколько раз отличаются площади аберрационных эллипсов Денеба на Марсе и на Земле. Считать, что орбиты планет круговые и лежат в одной плоскости.}
    \item{За три месяца положение некоторой звезды из-за параллакса изменилось на $0.014''$ по склонению, а по прямому восхождению не изменилось. Найдите расстояние до этой звезды от Земли. Экваториальные координаты звезды: $\alpha=6^h$, $\delta=-66.5^{\circ}$.}
\end{enumerate}
\end{document}