\documentclass[12pt]{article}
\usepackage{baseset}
\usepackage{myproblem}
\usepackage{stackengine}
\DeclareSymbolFont{operators}{OT1}{ntxtlf}{m}{n}
\SetSymbolFont{operators}{bold}{OT1}{ntxtlf}{b}{n}
\usepackage{wasysym}
\newcommand{\RomanNumeralCaps}[1]
{\MakeUppercase{\romannumeral #1}}
\usepackage{tabularx}
\usepackage{enumitem}

\begin{document}
\begin{tabularx}{\textwidth}{Xr}
{\Large \textbf{Оптика}} & ФТЛ $27.12.2023$ \\
\end{tabularx}
\noindent\rule{\textwidth}{0.4pt}
\section*{Телескопы}
\begin{enumerate}
    \item Определите, какое минимальное угловое расстояние между компонентами двойной системы можно разрешить в телескопы $10$ и $20$ см.
    \item Сравните разрешающие способности VLA, телескопа им. Хаббла и космического телескопа Спитцер. Эффективные диаметры этих инструментов $36$ км, $2.5$ м и $85$ см. Рабочая длина волны для этих телескопов $6$ см, $0.6$ мкм и $5$ мкм, соответственно.
    \item Расстояние между звездами в двойной системе $3$~а.е. Какой должен быть диаметр телескопа у наблюдателя, чтобы звезду можно было различить в этот телескоп. Расстояние до системы $21$~пк.
    \item Какое разрешение будет достигнуто при визуальных наблюдениях в Кавказской горной обсерватории (атмосфера $0.6''$) в телескоп $25$ см?
    \item Определите предельную звездную величину для телескопа $25$ см. Увеличение равнозарчковое. Что изменится, если взять окуляр с диаметром $4$ мм?
    \item Небольшое рассеянное скопление состоит из $40$ одинаковых звезд и имеет общий блеск $8^m$. Какой должен быть диаметр объектива телескопа, чтобы в него можно было увидеть отдельные звезды скопления?
    \item Определите минимальный размер деталей на поверхности Луны, которые можно увидеть в бинокль с диаметром объектива $50$ мм и увеличением $10$ крат. А в любительский телескоп с диаметром объектива $130$ мм и увеличением $30$ крат?
    \item Определите относительное отверстие, разрешение, проницающую способность, равнозрачковое увеличение школьного менискового телескопа Максутова и школьного телескопа-рефрактора, если первый имеет диаметр $70$ мм и фокусное расстояние $70.4$ см, а второй -- диаметр $80$ мм и фокусное расстояние $80$ см?
    \item Расстояние между компонентами двойной звезды Капеллы $0.054''$. Какие окуляры нужно применять, чтобы наблюдать ее раздельно в телескоп диаметром $D = 1$ м и фокусом $F = 10$ м и телескоп с $D = 5$ м и $F = 30$ м?
    \item Какой диаметр будет иметь изображение Солнца (видимый диаметр $32'$) в фокусе объектива с фокусным расстоянием $40$ см?
    \item Какой из двух телескопов с диаметром объектива $D$ и фокусным расстоянием $F$ нужно использовать для фотографирования двойной звезды с угловым расстоянием между компонентами $0.8''$, если размер пикселя на ПЗС-матрице $30$ мкм:
	\begin{itemize}
		\item $D = 35$ см, $F = 4$ м;
		\item $D = 10$ см, $F = 12$ м.
	\end{itemize}
    Относительное отверстие телескопа $1/8$. Определите минимальный линейный размер изображения точечного источника в фокальной плоскости.
    \item Какой минимальный размер деталей можно разглядеть на Луне в телескоп им. Хаббла диаметром $2.4$ м с относительным отверстием $f/24$. Камера Wide Field Camera $3$ имеет размер пикселя $5$ мкм.
    \item Определите время прохождения звезд по диаметру поля зрения альт-азимутального менискового телескопа, без часового ведения, диаметром $127$ мм $f/10$ , с окуляром $45^{\circ}$, $f=10$ мм. Наблюдение происходит в моменты верхней кульминации звезд:
	\begin{enumerate}
		\item	Вега  ($\alpha= 18^h 36^m$, $\delta=38^{\circ}47'$)
		\item	Дубхе ($\alpha= 11^h 03^m$, $\delta=61^{\circ}45'$)
	\end{enumerate}
    \item Перед фотографическими наблюдениями с линзовым астрографом (диаметр объектива $40$~см, относительное отверстие $1/4$) была допущена ошибка при фокусировке на $\Delta x=2$ мм. Определите, каким будет угловое разрешение при наблюдениях. Оцените, насколько изменится предельная звездная величина на снимках, если при идеальной фокусировке диаметр звездных изображений в фокальной плоскости равен $0.1$ мм.
    \item Каким должно быть фокусное расстояние наземного телескопа с апертурой $20$ см, чтобы количество энергии, приходящее от Марса и Антареса ($1.1^m$) на один пиксель ПЗС-матрицы, было одинаковым? Считать Марс находящимся в великом противостоянии: его блеск $-2.9^m$, расстояние до Земли $56$~млн км. Размер квадратного пикселя ПЗС-матрицы равен $10$ мкм.
\end{enumerate}
\end{document}