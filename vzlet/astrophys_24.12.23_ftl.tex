\documentclass[12pt]{article}
\usepackage{baseset}
\usepackage{myproblem}
\usepackage{stackengine}
\DeclareSymbolFont{operators}{OT1}{ntxtlf}{m}{n}
\SetSymbolFont{operators}{bold}{OT1}{ntxtlf}{b}{n}
\usepackage{wasysym}
\newcommand{\RomanNumeralCaps}[1]
{\MakeUppercase{\romannumeral #1}}
\usepackage{tabularx}
\usepackage{enumitem}

\begin{document}
\begin{tabularx}{\textwidth}{Xr}
{\Large \textbf{Астрофиз}} & ФТЛ $24.12.2023$ \\
\end{tabularx}
\noindent\rule{\textwidth}{0.4pt}
\section*{Собственное движение}
\begin{enumerate}
    \item Некоторая звезда обладает видимой звездной величиной $7^m$ и ненулевым собственным движением. Какова будет ее видимая звездная величина в тот момент, когда собственное движение звезды уменьшится в $4$ раза? Полная скорость звезды остается постоянной.
    \item Звезда Вега имеет собственное движение $0.35''$ в год, параллакс $0.129"$ и лучевую скорость $-14$ км/c. Через сколько лет Вега окажется к нам вдвое ближе, чем сейчас?
    \item У Альтаира годичный параллакс равен $0.198''$, собственное движение $0.658''$/год, лучевая скорость $V_r =- 26$ км/с и блеск $0.77^m$. Когда и на какое наименьшее расстояние Альтаир сблизится с Солнцем, и каким будет тогда его видимый блеск?
    \item В спектре звезды линия гелия с длиной волны $5016$~\AA~сдвинута на $0.017$ мм к красному концу, при дисперсии спектрограммы на этом участке в $20$~\AA/мм. Эклиптическая долгота звезды равна $47^{\circ}55'$ и ее эклиптическая широта -- $26^{\circ}45'$, а во время фотографирования спектра эклиптическая долгота Солнца была близкой к $223^{\circ}14'$. Параллакс звезды $0.073''$, а компоненты собственного движения $+0.0083^s$ и $-0.427''$. Определите величину и направление пространственной скорости звезды.
\end{enumerate}
\section*{Кинематика Галактики}
\begin{enumerate}[resume]
    \item Звезда, похожая на Солнце, обращается вокруг центра Галактики. Скорость звезды относительно Солнца равна $150$ км/с. Известно, что обе звезды находятся на одинаковом расстоянии от центра Галактики, равном $8$ кпк. Считая, что среднее межзвездное поглощение равно $2^m/\mbox{кпк}$, найдите видимую звездную величину этой звезды.
    \item Вокруг центра Галактики вращается звезда, причем ее максимальное расстояние от диска Галактики составляет $0.3$ кпк. Расстояние звезды от центра Галактики равно $5$ кпк. Чему равна максимальная лучевая скорость этой звезды?
    \item Предположим, что у Галактики наблюдается «плоская» кривая вращения (независимости линейной скорости от расстояния от центра), определите примерную зависимость плотности темной материи в Галактике от расстояния от центра.
    \item Оцените абсолютные звездные величины двух галактик: спиральной со скоростью вращения на палто $150$ км/с и эллиптической с дисперсией скоростей $280$ км/с.
\end{enumerate}
\section*{Двойные звезды}
\begin{enumerate}[resume]
    \item Две звезды солнечной массы вращаются вокруг общего центра масс по круговым орбитам. Промежуток времени между двумя соседними минимумами $30$ дней. Определите длительность минимума, если наблюдатель находится в плоскости орбиты системы.
    \item Затменная переменная состоит из двух звезд с одинаковым блеском $6^m$ и температурами
	поверхности $5~000$~K и $10~000$ K. Чему равен блеск переменной в моменты главного и вторичного минимумов блеска и вне затмений? Считать, что поверхностная яркость звезды одинакова по всему ее диску, а Земля находится точно в плоскости орбит звезд.
    \item Двойная система состоит из двух белых карликов, вращающихся вокруг общего центра масс по круговым орбитам. Известно, что такая система испускает гравитационные волны с частотой, равной удвоенной орбитальной частоте системы. Оцените минимально возможную длину волны гравитационного излучения такой двойной системы.
    \item Система из двух звезд является затменной переменной, а линия водорода $H_{\alpha}$ ($6563$~\AA) каждые $5$ лет сначала раздваивается на $1.0$~\AA~и $0.75$~\AA, а потом вновь сливается воедино. Чему равно расстояние между звездами? Массы звезд? Сколько длятся транзиты? Линия апсид перпендикулярна лучу зрения.
\end{enumerate}
\section*{Фотометрия}
\begin{enumerate}[resume]
    \item Капелла А - спектрально-двойная система. Первый компонент -- красный гигант класса K$0$III -- имеет звездную величину в фильтре $V=0.89^m$ и показатель цвета $(B-V)_{1}=0.93^m$. Второй компонент -- субгигант класса G$0$IV -- имеет звездную величину в фильтре $V_2=0.76^m$ и показатель цвета $(B-V)_{2}=0.67^m$. Определите показатель цвета спектрально-двойной Капелла А и ее звездную величину в фильтре B.
    \item Компоненты собственного движения звезды спектрального класса G$2$V равны $\mu_l=12~mas/\mbox{год}$, $\mu_b=10~mas/\mbox{год}$. Определите ее трансверсальную скорость, если видимая величина звезды $V=14^m$, а избыток цвета $E(B-V)=0.9^m$.
    \item Из наблюдений было получено, что $V = 1.8^m$, а годичный параллакс звезды составил $\pi=0.02''$. Известно, что для данного типа звезд истинный показатель цвета $(B-V)_0=-0.3^m$, однако его измеренное значение оказалось равным $(B-V)=0.5^m$. Найдите истинную $(M_V)_0$ и абсолютную болометрическую звездную величину $M_{bol}$, если известно, что для этого типа звезд болометрическая поправка $BC=-2.8^m$. Оцените спектральный класс звезды.
    \item На рисунке представлена кривая блеска двойной звезды, полученная в фильтре~$V$. Зная, что затмения в системе центральные, один из компонентов двойной имеет спектральный класс A$0$, а второй -- G$2$, и оба компонента являются звёздами главной последовательности, постройте кривую изменения показателя цвета $B-V$ этой системы. Ось ординат Вашего графика направьте вверх, нанесите деления и поставьте соответствующие значения показателей цвета.

    \begin{figure}[ht]
        \begin{tikzpicture}
        \begin{axis}[
        table/col sep = comma,
        height = 0.3\paperheight, 
        width = 0.7\paperwidth,
        minor x tick num = 1,
        minor y tick num = 9,
        ytick = {10,10.1,10.2},
        ymax = 10.12,
        xlabel = {время $t$, условные единицы},
        ylabel = {блеск в фильтре $V$},
        mark size = 0 pt,
        y dir=reverse
        ]
        
        \addplot[black] coordinates {
        (-3,9.984)
        (-2,9.984)
        (-1,10.098)
        (1,10.098)
        (2,9.984)
        (9,9.984)
        (10,10)
        (12,10)
        (13,9.984)
        (20,9.984)
        (21,10.098)
        (23,10.098)
        (24,9.984)
        (25,9.984)};
        \end{axis}
        \end{tikzpicture}
    \end{figure}
\end{enumerate}
\end{document}