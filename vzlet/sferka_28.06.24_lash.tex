\documentclass[12pt]{article}
\usepackage{baseset}
\usepackage{myproblem}
\usepackage{stackengine}
\DeclareSymbolFont{operators}{OT1}{ntxtlf}{m}{n}
\SetSymbolFont{operators}{bold}{OT1}{ntxtlf}{b}{n}
\usepackage{wasysym}
\newcommand{\RomanNumeralCaps}[1]
{\MakeUppercase{\romannumeral #1}}
\usepackage{tabularx}
\usepackage{enumitem}

\begin{document}
\begin{tabularx}{\textwidth}{Xr}
{\Large \textbf{Сферическая астрономия}} & ЛАШ $28.06.2024$ \\
\end{tabularx}
\noindent\rule{\textwidth}{0.4pt}
\begin{enumerate}
    \subsection*{Невосходящие/незаходящие звезды}
    \item Определите диапазоны склонений звезд, которые являются незаходящими для Москвы ($\varphi=55^{\circ}45'$).
    \item Определите широты, на которых звезда Дубхе ($\alpha$ UMa, $\delta = 61^{\circ}45'45''$) является незаходящей звездой. А невосходящей звездой?
    \item Определите, будет ли звезда Сириус ($\delta=-16^{\circ}42'58''$) являться незаходящей или невосходящей для наблюдателей на экваторе, на южном тропике и северном тропике.
    \item Созвездие Золотая Рыба (Dor) южном полушарии неба. В этом созвездии наиболее ярким объектом является спутник нашей галактики -- неправильная галактика Большое Магеланово Облако. Диапазон склонений созвездия $-70^{\circ}$ до $-49^{\circ}$. Определите широты в северном полушарии, где может наблюдаться все созвездие целиком. Определите широты в северном полушарии, где созвездие может быть видно частично?
    \item Определите широты, на которых созвездие Золотая Рыба (диапазон склонений $-70^{\circ}$ до $-49^{\circ}$) является полностью незаходящим. 
    \item Определите широты, на которых созвездие Золотая Рыба (диапазон склонений $-70^{\circ}$ до $-49^{\circ}$) наблюдалось в верхней кульминации только с северной стороны горизонта и было доступно для наблюдений.
    \item Из каких областей земной поверхности возможно одновременное наблюдение Арктура ($\alpha$ Волопаса) и Хадара ($\beta$ Центавра)?  Координаты этих звезд считать равными $\alpha_{1}=14.0^{h}$, $\delta_{1}=+19^{\circ}$; $\alpha_{2}=14.0^h$, $\delta_{2}=-60^{\circ}$ соответственно. Атмосферной рефракцией и поглощением света пренебречь.
    \subsection*{Множественность решений}
    \item Верхняя кульминация светила происходит на высоте $60^{\circ}$, а нижняя кульминация на высоте $30^{\circ}$. Определите широту места наблюдения.
    \item В некоторый момент звезда со склонением $70^{\circ}$ находилась в кульминации для наблюдателя в Санкт-Петербурге ($\varphi=60^{\circ}$). В тот же момент вторая звезда оказалась также в кульминации, причем сумма высот звезд составила $110^{\circ}$. Определите склонение второй звезды.
    \item В наблюдательном дневнике астронома записаны наблюдения одной и той же звезды. Зенитное расстояние в нижней кульминации $47^{\circ}35'$ и высота в верхней кульминации $84^{\circ}15'$. Найдите широту места наблюдения.
    \item У одной звезды зенитные расстояния в моменты верхней и нижней кульминации равны $20^{\circ}$ и $30^{\circ}$. А у второй звезды, наблюдаемой в том же месте, высота верхней кульминации $h=80^{\circ}$. Определите высоту нижней кульминации второй звезды.
    \item Звезда \textbf{A} кульминирует на высоте, вдвое большей высоты звезды \textbf{B} в верхней кульминации. Верхняя кульминация звезды \textbf{A} происходит на высоте $85^{\circ}$. На какой высоте происходит нижняя  кульминация звезды \textbf{B}? Наблюдения ведутся на широте $70^{\circ}$ с. ш.
    \item Звезда \textbf{A} заходит точно в точке запада. И ее высота верхней кульминации ровно в два раза меньше высоты верхней кульминации звезды \textbf{B}. Широта места наблюдения $\varphi=45^{\circ}$. Определите, какое время над горизонтом проводят звезды \textbf{A} и \textbf{B} для наблюдателя на этой широте. Чему равны их склонения?
    \item Нижняя кульминация звезды \textbf{A} происходит на той же высоте, что и верхняя кульминация звезды \textbf{A}. Известно, что звезда \textbf{A} восходит точно на востоке, а нижняя кульминация звезды \textbf{A} в $2$ раза ниже её верхней. Найти широту и склонение звезды $A$.
    \item Высота звезды в верхней кульминации $40^{\circ}30'$, а в нижней кульминации ее высота $30^{\circ}40'$. Найдите широту места наблюдения и склонение звезды.
    \item В некоторый момент звезда со склонением $30^{\circ}$ находилась в кульминации для наблюдателя в Санкт-Петербурге ($\varphi=60^{\circ}$). В тот же момент вторая звезда оказалась также в кульминации, причем сумма высот звезд составила $125^{\circ}$. Определите склонение второй звезды. 
    \item В некоторый момент звезда со склонением $\delta_0$ находилась в кульминации для наблюдателя в Санкт-Петербурге ($\varphi=60^{\circ}$). В тот же момент вторая звезда оказалась также в кульминации, причем сумма высот звезд составила $h_{\sum}$. Определите, сколько максимально может быть ответов у этой задачи.
\end{enumerate}
\end{document}