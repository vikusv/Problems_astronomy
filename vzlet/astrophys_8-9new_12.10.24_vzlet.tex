\documentclass[12pt]{article}
\usepackage{baseset}
\usepackage{myproblem}
\usepackage{stackengine}
\DeclareSymbolFont{operators}{OT1}{ntxtlf}{m}{n}
\SetSymbolFont{operators}{bold}{OT1}{ntxtlf}{b}{n}
\usepackage{wasysym}
\newcommand{\RomanNumeralCaps}[1]
{\MakeUppercase{\romannumeral #1}}
\usepackage{tabularx}
\usepackage{enumitem}

\begin{document}
\begin{tabularx}{\textwidth}{Xr}
{\Large \textbf{Астрофиз}} & Взлёт $12.10.2024$ \\
\end{tabularx}
\noindent\rule{\textwidth}{0.4pt}
\begin{enumerate}
    \item{Определите диапазон звездных величин, которые может иметь объект в поясе Койпера ($30-55$~а.е.) с диаметром $1000$~км и сферическим альбедо $0.07$.}
    \item{Определите эффективную температуру теплового излучения Венеры, если ее сферическое альбедо равно $0.77$.}
    \item{Определите зону «обитаемости» вокруг экзопланеты Проксима Центавра. Радиус звезды $0.15 R_{\odot}$, температура на поверхности звезды $3~000$~К. Парниковым эффектом пренебречь.}
    \item{Вокруг звезды главной последовательности вращается планета с таким же периодом, что и Земля. Альбедо планеты равно $0.4$. Масса звезды в $2$ раза больше массы Солнца. Найдите эффективную температуру на планете.}
    \item{Где-то в нашей солнечной системе летает темный быстровращающийся астероид. Длина волны, на которую приходится максимум энергии его излучения, может меняться в $3$~раза. Определите эксцентриситет орбиты этого небесного тела.}
    \item Определите температуры объектов Солнечной системы. Ответы представьте в виде таблицы.
    \item Определите границы зоны обитаемости для Солнца.
    \item На далекой обитаемой планете тепловые условия аналогичны земным, но местное Солнце имеет втрое меньший угловой диаметр. Найдите температуру этой далекой звезды.
    \item Вблизи звезды HD$209458$ спектрального класса G$0$V (температура $T = 6000$ К) обнаружена планета HD$209458$b с круговой орбитой и парами воды в атмосфере. Угловой радиус этой звезды при наблюдении с данной планеты составляет $6.61^{\circ}$. Найдите сферическое альбедо планеты, если ее эффективная температура $1130$~К.
    \item Вокруг некой звезды $A$ вращается планета, с периодом в $100$ лет. Максимум излучения звезды приходится на $3625$ A, радиус звезды $3~R_{\odot}$, также известно, что атмосферы на планете нет, альбедо планеты $A=0.3$. Определите эффективную температуру планеты. Считайте, что центральная звезда принадлежит главной последовательности.
    \item Вокруг звезды главной последовательности вращается планета с таким же периодом, что и Земля. Альбедо планеты равно $0.5$. Масса звезды равна массе Солнца. Найдите эффективную температуру на планете.
    \item Определите температуру пылинки радиусом $2$~мкм, расположенную на расстоянии $2.5$~а.е от Солнца. Пылинку считать чернотельной.
\end{enumerate}
\end{document}