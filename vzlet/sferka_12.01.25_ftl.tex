\documentclass[12pt]{article}
\usepackage{baseset}
\usepackage{myproblem}
\usepackage{stackengine}
\DeclareSymbolFont{operators}{OT1}{ntxtlf}{m}{n}
\SetSymbolFont{operators}{bold}{OT1}{ntxtlf}{b}{n}
\usepackage{wasysym}
\newcommand{\RomanNumeralCaps}[1]
{\MakeUppercase{\romannumeral #1}}
\usepackage{tabularx}
\usepackage{enumitem}

\begin{document}
\begin{tabularx}{\textwidth}{Xr}
{\Large \textbf{Сферка}} & Физтех-Лицей $12.01.2025$ \\
\end{tabularx}
\noindent\rule{\textwidth}{0.4pt}
\begin{enumerate}
    \item{Астроном заметил, что Нептун кульминирует в полночь на том же альмукантарате, что и Солнце в полдень. Определите все возможные широты наблюдений. Наклоном орбиты Нептуна к эклиптике пренебречь.}
    \item{Любители астрономии наблюдали планеты и обнаружили, что Юпитер кульминировал в $6$ часов вечера по местному времени на высоте $15^{\circ}$, а Марс – в $6$ часов утра по местному времени на высоте $62^{\circ}$. В какой сезон года и на какой широте проводились наблюдения?}
    \item{Когда лучше всего наблюдать зодиакальный свет в северном полушарии?}
    \item{В $2003$ году Юпитер вступил в противостояние в начале февраля. Как в этом месяце день ото дня изменяется его максимальная высота над горизонтом на широте Москвы?}
    \item{Определите, какого числа в Джакарте (столице Индонезии) Солнце кульминирует в зените. ($\varphi=6^{\circ}10'$ ю.ш.). А в Мачу-Пикчу ($\varphi=13^{\circ}09'48''$ ю.ш.)?}
    \item{$5$ мая Венера оказалась в наибольшей восточной элонгации и в каком-то пункте Земли стала незаходящим светилом, ее нижняя кульминация произошла в точке севера. Где в этот момент находилось Солнце?}
    \item{Солнце и Луна в фазе первой четверти одновременно заходят за горизонт. На какой	широте находится наблюдатель? Рефракцией и параллаксом Луны пренебречь.}
    \item{Определите время восхода и захода Солнца $23$ сентября для наблюдателя в городе Москва с координатами $\lambda=37^{\circ}30'$ и $\varphi=56^{\circ}$ по гражданскому времени.}
    \item Ближайшее нижнее соединение Венеры с Солнцем по эклиптической долготе произойдет $13$ августа $2023$ года в $11$ч$10$м по Всемирному времени. Известно, что координаты Солнца в этот момент составят $\alpha = 9^h31.5^m$, $\delta = +14^{\circ}40'$, а Венера пройдет в $7^{\circ}40'$ южнее эклиптики. Определите координаты точки на поверхности Земли, из которой Венера будет лучше всего видна в этот момент. Считать, что Венера видна, если центр диска Солнца расположен не выше горизонта, а критерием качества видимости при этих условиях является высота Венеры над горизонтом. Атмосферной рефракцией и уравнением времени пренебречь.
    \item Во время наибольшей элонгации Венеры в некоторой точке Земли Солнце видно на юге, а Венеру на той же высоте на севере. Может ли такое быть? Если да, то в каких широтных областях Земли и на какой высоте над горизонтом находились Солнце и Венера? Решите ту же задачу для случая, когда Солнце находится на западе, а Венера на той же высоте на востоке.
    \item Находясь в северном полушарии, мы $22$ декабря наблюдаем парадоксальное явление: планета Венера, находясь в точке наибольшей элонгации, восходит точно на юге. В каких широтах мы находимся, и какая элонгация у Венеры – восточная или западная? Где в это время находилось Солнце?
    \item $7$ мая Венера оказалась в наибольшей восточной элонгации и в каком-то пункте Земли стала незаходящим светилом, ее нижняя кульминация произошла в точке севера. Где в этот момент находилось Солнце?
    \item $1$ мая наступило противостояние Марса. В некоторой точке Земли в момент противостояния Солнце и Марс одновременно взошли над горизонтом. Найдите широту данной точки и определите, над какими сторонами горизонта располагались Солнце и Марс. Наклоном плоскости орбиты Марса к эклиптике и рефракцией пренебречь.
    \item Солнце и Луна в фазе первой четверти одновременно заходят за горизонт. На какой	широте находится наблюдатель? Рефракцией и параллаксом Луны пренебречь.
    \item В некотором пункте Земли центр диска Луны взошел на $20$ минут раньше по местному (среднему солнечному) времени, чем в предыдущие сутки, находясь в созвездии Рыб. Определите возможные значения широты этого пункта. Атмосферной рефракцией, суточным параллаксом Луны и эксцентриситетом ее орбиты пренебречь. 
    \item Далекое светило с координатами ($\alpha=0$, $\delta=0$) находится на высоте $0^{\circ}$ над горизонтом в $0^h0^m$ по Всемирному времени $1$ января. Определите координаты всех пунктов на Земле, где такое может быть. Рефракцией и уравнением времени пренебречь. 
    \item Определите местное время верхней кульминации туманности Андромеды ($\alpha=00^h 42^m$, $\delta=41^{\circ}16'$) в Долгопрудном ($\lambda=37^{\circ}30'$, $\varphi=55^{\circ}56'$, UTC$+3$) $1$~сентября. 
\end{enumerate}
\end{document}