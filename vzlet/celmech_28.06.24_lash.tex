\documentclass[12pt]{article}
\usepackage{baseset}
\usepackage{myproblem}
\usepackage{stackengine}
\DeclareSymbolFont{operators}{OT1}{ntxtlf}{m}{n}
\SetSymbolFont{operators}{bold}{OT1}{ntxtlf}{b}{n}
\usepackage{wasysym}
\newcommand{\RomanNumeralCaps}[1]
{\MakeUppercase{\romannumeral #1}}
\usepackage{tabularx}
\usepackage{enumitem}

\begin{document}
\begin{tabularx}{\textwidth}{Xr}
{\Large \textbf{Небесная механика}} & ЛАШ $28.06.2024$ \\
\end{tabularx}
\noindent\rule{\textwidth}{0.4pt}
\begin{enumerate}
    \section*{Базовые задачи}
    \subsection*{Геометрия эллипса}
    \item{Короткопериодическая комета $2$P/Энке имеет орбиту с большой полуосью $2.22$~а.е. и эксцентриситетом $0.847$. Определите максимальное и минимальное расстояния от кометы до Солнца.}
	\item{Гипотетическая планета Меркун в ближайшей к Солнцу точке своей орбиты пересекает орбиту Меркурия, а в самой далёкой точке -- орбиту Нептуна. Найдите большую полуось и эксцентриситет орбиты Меркуна.}
	\item{Угловое удаление от Меркурия до Солнца в момент наибольшей элонгации может составлять от $18$ до $28$ градусов из-за эллиптичности его орбиты. По этим данным определите эксцентриситет орбиты Меркурия.}
	\item{Эксцентриситет орбиты Луны составляет $1/18$. На сколько процентов скорость Луны в перигее больше, чем в апогее?}
	\item{Гелиоцентрическая угловая скорость кометы в перигелии в $930.25$ раз больше, чем в вершине малой полуоси. Во сколько раз она превосходит угловую скорость в афелии?}
	\item{Большая полуось орбиты Юпитера -- $5.204$ а.е. Определите период его обращения вокруг Солнца.}
	\item{Астероид Икар проходит перигелий каждые $409$ суток, приближаясь к Солнцу на расстояние $0.187$ а.е. Как далеко может удаляться от Солнца Икар?}
	\subsection*{\RomanNumeralCaps{3} закон Кеплера}
    \item Планета обращается вокруг звезды по круговой орбите с периодом ровно $10$ лет, в ее небе звезда имеет угловой диаметр ровно $10'$ (десять угловых минут). Найдите среднюю плотность звезды.
    \item Как должна была бы мгновенно измениться масса Земли, чтобы оставаясь на прежнем расстоянии, Луна обращалась вокруг Земли за $2$ суток? %186 раз
    \item Оцените пероиод обращения Солнечной системы относительно центра галактики. Масса центра галактики $2\cdot10^{12}$. Расстояния от Солнца до ближайшей черной дыры $8$ кпк. %47 миллионов лет
    \item Экзопланета обращается вокруг звезды массы $0.85 M_{\odot}$ с периодом $0.78$ лет. Определите полуось орбиты планеты в астрономических единицах и километрах. % 0.8 ае, 1.2*10^8
    \item Предположим, мы наблюдаем двойную систему, состоящую из двух звёзд, массы которых $2$ и $3$ массы Солнца, а период системы равен $4$ года. Определите расстояние между звёздами. %4.3 ае
	
    \section*{Задачи посложнее}
    \item{Определите эксцентриситет орбиты кометы, если большая полуось орбиты $11$~а.е., а фокальный параметр -- $3$ а.е.}
    \item{В некоторый момент времени и Земля, и Луна находятся на расстоянии $1.0000$ а.е. от центра Солнца. В каком созвездии видна Луна земному наблюдателю?}
    \item{Солнце постоянно теряет массу за счёт собственного излучения (излучённая Солнцем энергия понижает его массу покоя по Эйнштейну). Определите, с какой скоростью в связи с этим меняется орбитальный период Земли. Ответ выразите в секундах за миллион лет. Считайте, что орбита Земли сохраняет круговую форму со временем.}
    \item{Небольшая планета обращается вокруг центральной звезды по круговой орбите. На каждом обороте планеты в одной и той же точке ее орбиты она тесно сближается с одной и той же кометой, которая в этот момент проходит точку апоцентра своей орбиты и располагается на небе планеты в $90^{\circ}$ от центральной звезды. Определите эксцентриситет орбиты кометы. Орбитальные периоды планеты и кометы различаются, взаимодействием планеты и кометы пренебречь.}
    \item{После гравитационного маневра около Юпитера зонд «Улисс», предназначенный для изучения солнечной магнитосферы, направился к Солнцу по энергетически выгодной (гомановской) траектории по гелиоцентрической орбите, перпендикулярной плоскости эклиптики, с периодом $6.2$ года. На какой высоте он пролетел над северным полюсом Солнца? Наклоном оси вращения Солнца к оси эклиптики пренебречь.}

\end{enumerate}
\end{document}