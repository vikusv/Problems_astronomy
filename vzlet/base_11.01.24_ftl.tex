\documentclass[12pt]{article}
\usepackage{baseset}
\usepackage{myproblem}
\usepackage{stackengine}
\DeclareSymbolFont{operators}{OT1}{ntxtlf}{m}{n}
\SetSymbolFont{operators}{bold}{OT1}{ntxtlf}{b}{n}
\usepackage{wasysym}
\newcommand{\RomanNumeralCaps}[1]
{\MakeUppercase{\romannumeral #1}}
\usepackage{tabularx}
\usepackage{enumitem}

\begin{document}
\begin{tabularx}{\textwidth}{Xr}
{\Large \textbf{Все подряд}} & ФТЛ $11.01.2024$ \\
\end{tabularx}
\noindent\rule{\textwidth}{0.4pt}
\section*{Углы}
\begin{enumerate}
    \item{Человеческий глаз имеет разрешение примерно $1'$. Определите минимальный размер кратера на Луне, который человек может увидеть без использования специальных оптических приборов.}
    \item{Определите размер солнечного пятна, находящегося недалеко от центра диска Солнца, если оно видно с Земли под углом $4''$}
    \item{Решите предыдущую задачу (найдите наибольший из размеров пятна), если угловое расстояние от пятна до центра составляет $x$.}
    \item{Определите максимальное угловое расстояние между Солнцем и Землёй для наблюдателя в системе Проксима Центавра: расстояние до системы составляет $4.2$ световых года.}
    \item{Для тех же наблюдателей в системе Проксима Центавра наблюдаемое минимальное угловое расстояние Солнцем и Землёй на $30\%$ меньше максимального. Определите модуль их эклиптической широты.}
\end{enumerate}
\section*{Сферка}
\begin{enumerate}[resume]
    \item Звезда Капелла ($\alpha$ Aur, прямое восхождение $\alpha=5^h 16^m 41^s$, склонение $\delta=46^{\circ}00'$) кульминирует строго в зените. Определите высоты обеих кульминаций для звезды Мерак ($\beta$ UMa, $\delta=56^{\circ}17'$) в этом же месте наблюдения.
    \item Верхняя кульминимация звезды Вега ($\alpha$ Lyr, склонение $\delta=38^{\circ} 47'$) происходит на высоте $83^{\circ}18'$. Определите широту места наблюдения. Определите высоту нижней кульминации.
    \item Нижняя кульминация звезды совпадает с горизонтом, а верхняя кульминация с зенитом. Определите широту места наблюдения и склонение звезды.
    \item Определите склонение звезды, которая в Долгопрудном ($\varphi_1 = 55^{\circ}56'$) и во Владивостоке ($\varphi_2 = 43^{\circ}11'$) кульминирует на одно и той же высоте.
    \item Определите широты мест наблюдения, где звезда Фомальгаут ($\delta = -29^{\circ}37'$) является невосходящей. Рефракцией пренебречь.
\end{enumerate}
\section*{Звездные величины}
\begin{enumerate}[resume]
    \item{Экзопланета может быть обнаружена транзитным методом (изменение яркости звезды в моменты прохождения планеты по диску звезды), если диск планеты перекроет $1\%$ поверхности звезды. Определите, насколько изменяется звездная величина звезды в такие моменты?}
	\item{Определите звездные величины компонент A и В звезды $\alpha$ Cen, если суммарная звездная величина -- ($-0.27^m$), а соотношение светимостей компонент -- $3.47$.}
	\item{Определите видимую звездную величину компонентов тройной звезды, если ее суммарный блеск равен $3.7^m$, второй компонент ярче третьего в $2.8$ раза, а первый ярче третьего на $3.32^m$.}
	\item{Шаровое скопление содержит $10^{6}$ звезд звездной величины $22^m$ и $10 000$ сверхгигантов со звездной величиной $17^m$. Сможем ли мы увидеть это шаровое скопление глазом?}
	\item{При фотографировании звездного неба с большой выдержкой неподвижным фотоаппаратом изображения звезд на снимке получаются в виде дуг различной длины. Изображение какой звезды на таком снимке будет ярче -- $\alpha$ Овна ($m = 2.0^m$, $\delta=23.5^{\circ}$) или $\beta$ Малой Медведицы ($m = 2.1^m$, $\delta=74^{\circ}$)? Поглощением света в атмосфере пренебречь.}\
\end{enumerate}
\section*{Телескопы}
\begin{enumerate}[resume]
    \item Чему равно равнозрачковое увеличение телескопа с диаметром объектива $120$ мм?
    \item Рассчитайте разрешающую способность наблюдений с оптическим телескопом с диаметром объектива $200$ мм. Увеличение равнозрачковое.
    Среднюю длину волны оптического диапазона принять равной $550$ нм.
    \item Чему равен диаметр объектива телескопа, если его относительное отверстие $1:15$, а фокусное расстояние равно $3$ м?
    \item Звезды какой звездной величины можно наблюдать в телескоп с диаметром объектива $10$ см?
    \item Небольшое рассеянное скопление состоит из $50$ одинаковых звезд и имеет общий блеск $6^m$. Какой должен быть диаметр объектива телескопа, чтобы в него можно было увидеть отдельные звезды скопления?
\end{enumerate}
\end{document}