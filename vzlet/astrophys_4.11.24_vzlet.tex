\documentclass[12pt]{article}
\usepackage{baseset}
\usepackage{myproblem}
\usepackage{stackengine}
\DeclareSymbolFont{operators}{OT1}{ntxtlf}{m}{n}
\SetSymbolFont{operators}{bold}{OT1}{ntxtlf}{b}{n}
\usepackage{wasysym}
\newcommand{\RomanNumeralCaps}[1]
{\MakeUppercase{\romannumeral #1}}
\usepackage{tabularx}
\usepackage{enumitem}

\begin{document}
\begin{tabularx}{\textwidth}{Xr}
{\Large \textbf{Астрофиз}} & Взлёт $4.11.2024$ \\
\end{tabularx}
\noindent\rule{\textwidth}{0.4pt}
\subsection*{Абсолютно черное тело}
\begin{enumerate}
    \item Максимум энергии в спектре звезды Спика приходится на длину волны $1450$~\AA, а в спектре Поллукса -- на $6580$~\AA. Определите температуры звезд.
    \item На какую длину волны приходится максимум излучения звезды Альдебаран, если температура его поверхности -- $3900K$.
    \item Какова должна быть температура звезды, если при одинаковых с Солнцем размерах ее светимость в $81$ раз больше?
    \item Звезда Фомальгаут имеет видимую звездную величину $1.16^m$ и параллакс $0.130''$. Определите радиус звезды, если температура ее поверхности $8~590$~K.
    \item Радиус Кастора равен $2.3$ радиусам Солнца, температура -- $9900$ К, а видимая звездная величина -- $1.58^m$.
    \begin{enumerate}
        \item Определите длину волны максимума излучения Кастора. Какая звезда краснее: Кастор или Солнце?
        \item Посчитайте светимость Кастора, ответ выразите в светимостях Солнца.
        \item Каков угловой размер Кастора и расстояние до него?
    \end{enumerate}
    \item Определите температуру, цвет и спектральный класс звезды, если она имеет угловой размер $10$ mas и болометрическую звездную величину $0.9^m$.
    \item Две звезды имеют одинаковые угловые размеры, расстояния до них известны. Их температуры отличаются в $3$ раза. Найдите разницу болометрических звездных величин звезд.
    \item Затменная переменная состоит из двух звезд с одинаковым блеском $6^m$ и температурами поверхности $5 000$ К и $10 000$ К. Чему равен блеск переменной в моменты главного и вторичного минимумов блеска и вне затмений? Считать, что поверхностная яркость звезды одинакова по всему ее диску, а Земля находится точно в плоскости орбит звезд.
    \item Длина волны максимума излучения звезды приходится на $3000A$, а её видимая звёздная величина составляет $0^m$. Найдите радиус и массу, если ее годичный параллакс составляет $0.1''$. Поглощением пренебречь. Считать, что звезда находится на главной последовательности.
    \item Почему закон смещения Вина редко используется для определения цветовых температур холодных звезд ($T \textless 3500$ K) или очень горячих звезд ($T \textgreater 10000$ K)?
    \item Почему болометрические поправки наибольшие для горячих и звезд, и наименьшие для звезд типа Солнца?

    \item Звезда $\beta$ Золотой Рыбы -- переменная класса цефеид с периодом пульсации $P = 9.84$ сут. Предположим, что звезда является наиболее яркой в момент наибольшего сжатия (радиус $R_1$) и наиболее слабой в момент наибольшего расширения (радиус $R_2$), сохраняет сферическую форму и ведёт себя подобно абсолютно чёрному телу в каждый момент в течение всего цикла пульсаций. Болометрическая звёздная величина этой звезды меняется от $3.46^m$ до $4.08^m$. По измерениям доплеровского смещения известно, что в течение периода пульсаций поверхность звезды сжимается и расширяется со средней радиальной скоростью $v = 12.8$ км/с; спектральный максимум излучения колеблется от $\lambda_1 = 531.0$ нм до $\lambda_2 = 649.1$ нм.
    \begin{enumerate}
        \item Найдите отношение радиусов звезды $R_1/R_2$ в моменты наибольшего сжатия и наибольшего расширения и оцените величины этих радиусов.
        \item Вычислите освещенность от звезды в момент её наибольшего расширения.
        \item Определите расстояние до звезды.
    \end{enumerate}

    \item Оцените, сколько видно на небе звезд до m звездной величины?
\end{enumerate}
\subsection*{Температуры}
\begin{enumerate}[resume]
    \item Определите равновесную температуру Луны.
    \item Рассчитайте температуры всех больших планет Солнечной системы.
    \item На далекой обитаемой планете тепловые условия аналогичны земным, но местное Солнце имеет вдвое меньший угловой диаметр. Найдите температуру этой далекой звезды.
    \item Определите температуру пылинки радиусом $2$~мкм, расположенную на расстоянии $2.5$ а.е от Солнца. Пылинку считать чернотельной.
    \item Определите границы зоны обитаемости Солнца.
    \item Вокруг звезды главной последовательности вращается планета с таким же периодом, что и Земля. Альбедо планеты равно $0.5$. Масса звезды равна двум массам Солнца. Найдите эффективную температуру на планете.
    \item Вокруг некой звезды $A$ вращается планета, с периодом в $100$ лет. Максимум излучения звезды приходится на $3625$ A, радиус звезды $3~R_{\odot}$, также известно, что атмосферы на планете нет, альбедо планеты $A=0.3$. Определите эффективную температуру планеты. Считайте, что центральная звезда принадлежит главной последовательности.
    \item Вблизи звезды HD$209458$ спектрального класса G$0$V обнаружена планета HD$209458$b с круговой орбитой и парами воды в атмосфере. Угловой радиус этой звезды при наблюдении с данной планеты составляет $6.61^{\circ}$. Найдите сферическое альбедо планеты, если ее эффективная температура $1130$~К.
    \item Стандартная теория эволюции звезд утверждает, что $4$ миллиарда лет назад наше Солнце излучало на $30\%$ меньше энергии, чем сейчас. На основании этих данных оцените среднюю температуру на Земле в тот период, если считать, что орбита Земли и строение ее атмосферы в тот момент были в точности такими же, как сейчас.
    \item Равновесная температура на планете в течение $2.5$ лет меняется в $1.5$ раза. Какова светимость звезды и эксцентриситет орбиты планеты, если альбедо планеты $0.36$, а средняя температура планеты в периастре составляет $0^{\circ}$~C. Считайте, что звезда принадлежит главной последовательности.
\end{enumerate}
\subsection*{Звездные величины}
\begin{enumerate}[resume]
    \item Определите звездную величину Венеры в элонгации и в верхнем соединении.
    \item Определите звездную величину Сатурна в противостоянии, соединении и в квадратуре. 
    \item Наблюдения проводятся на Венера. Во сколько раз отличается освещенность, создаваемая Землей в противостоянии и в квадратуре?
    \item Определите альбедо астероида, который при наблюдении с Земли в соединении виден с блеском $6^m$. Большая полуось астероида -- $3$ а.е.
    \item В настоящее время ведутся поиски возможной девятой планеты Солнечной системы, которая может иметь диаметр в $10$ диаметров Земли и располагаться в $280$ а.е. от Солнца. Астероид какого диаметра в главном поясе будет иметь такую же яркость на Земле в противостоянии, как и эта планета? Отражательную способность поверхности астероида считать аналогичной лунной, а планеты -- аналогичной Нептуну. Оба тела располагаются в плоскости эклиптики.
    \item В момент каждого противостояния астероида земной наблюдатель измеряет его видимую звездную величину. Период обращения астероида равен $3.9$ года. Оцените эксцентриситет его орбиты, если амплитуда изменения видимой звездной величины составляет $2.5^m$. Орбиту Земли считаем круговой.
    \item Транснептуновый объект $(174567)$ Варда в настоящее время имеет видимую звездную	величину $21^m$ (при наблюдении с Земли) и находится на расстоянии $48$~а.е. от Солнца.	Оцените диаметр Варды, если ее поверхность отражает $10\%$ падающего на нее света. Видимая звездная величина Солнца (также при наблюдении с Земли) составляет -- $27^m$.
    \item Вокруг далёкой звезды обращаются три экзопланеты, причем разумные наблюдатели обитают лишь на второй. Большие полуоси орбит планет соотносятся как $1:4:9$. Орбиты планет круговые. Альбедо планет соотносятся как $3:5:4$. Третья находится в восточной квадратуре при наблюдении с первой. Первая в западной элонгации при наблюдения со второй. Все планеты сферической формы. Их радиусы соотносятся как $15:16:25$. Какая планета окажется ярче для наблюдателей на второй, и на сколько звездных величин?
\end{enumerate}
\end{document}