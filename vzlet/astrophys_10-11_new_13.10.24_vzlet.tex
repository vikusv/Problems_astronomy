\documentclass[12pt]{article}
\usepackage{baseset}
\usepackage{myproblem}
\usepackage{stackengine}
\DeclareSymbolFont{operators}{OT1}{ntxtlf}{m}{n}
\SetSymbolFont{operators}{bold}{OT1}{ntxtlf}{b}{n}
\usepackage{wasysym}
\newcommand{\RomanNumeralCaps}[1]
{\MakeUppercase{\romannumeral #1}}
\usepackage{tabularx}
\usepackage{enumitem}

\begin{document}
 \begin{tabularx}{\textwidth}{Xr}
  {\Large \textbf{Поглощение. Многоцветная фотометрия}} & Взлёт $13.10.2024$ \\
 \end{tabularx}
 \noindent\rule{\textwidth}{0.4pt}
 \subsection*{Поглощение}
 \begin{enumerate}
    \item Шаровое скопление содержит миллион звезд главной последовательности, каждая из которых имеет абсолютную звездную величину $M=+6$, а также $10 000$ красных гигантов с $M=+1$. Возможно ли увидеть это скопление невооруженным глазом с расстояния $10$ кпк, если межзвездное поглощение света для этого скопления $2^m \mbox{/кпк}$.
    \item Некоторая звезда имеет температуру $48~000$ К и радиус $1.5$ радиуса Солнца, она находится на расстоянии $3.2$ кпк от Солнца в направлении центра Галактики. Какую видимую звездную величину имеет звезда для наблюдателя с Земли, если поглощение света в плоскости Галактики составляет $2^m/\mbox{кпк}$?
    \item Определите величину падения блеска звезды при прохождении света от нее через облако с оптической толщей $\tau=1$.
    \item Оценить, какой максимальной толщины газопылевое облако может быть размещено между нами и Сириусом, так что бы он остался видимым невооружённым глазом? Видимая звёздная величина Сириуса $-1.44^m$. Плотность частиц облака считать равной плотности воды.
    \item На Марсе разразилась мощная пылевая буря, охватившая в равной степени всю планету и ослабившая блеск Солнца в зените на $1^m$. Определите общую массу поднятой пыли, считая, что она состоит из частиц радиусом $0.1$ мм и плотностью $1.5$ г/см$^3$. Волновые эффекты при взаимодействии света с частицей не учитывать.
    \item Зенитное ослабление света в атмосфере Земли составляет $0.23^m$. Определите оптическую толщину атмосферы в направлении зенита. 
    \item Поглощение света атмосферой Земли при наблюдениях в зените составляет $0.23^m$. Оцените, каким будет поглощение при наблюдении на зенитном расстоянии $60^{\circ}$. 
    \item Звезда имела в зените видимый блеск $0^m$, а на высоте $30^{\circ}$ стала светить вдвое слабее. Какую звездную величину она будет иметь на высоте $20^{\circ}$ над горизонтом? Атмосферные условия считать постоянными и однородными.
 \end{enumerate}
 \newpage
 \subsection*{Многоцветная фотометрия}
 Длины волны основных спектральных полос и их ширины:
\begin{center}	
	\begin{tabular}[h]{|c|c|c|}
		\hline
		Фильтр& $\lambda$, нм&  $\Delta \lambda$, нм\\
		\hline
		$U$ & 365& 66\\
		$B$ & 445 & 94\\
		$V$ & 551 & 88\\
		$R$ & 658 & 138\\
		$I$ & 806 & 149\\
		$Y$ & 1020 & 120\\
		$J$ & 1220 & 213\\
		$H$ & 1630 & 307\\
		$K$ & 2190 & 390\\
		$L$ & 3450 & 472\\
		$M$ & 4750 & 460\\
		$N$ & 10500 & 2500\\
		$Q$ & 21000 & 5800\\
		\hline
	\end{tabular}
\end{center}
 \begin{enumerate}[resume]
        \item Звезды \textbf{A} и \textbf{B} светят одинаково через красный светофильтр, звезды \textbf{B} и \textbf{C} -- одинаково через зеленый, а \textbf{A} и \textbf{C} – одинаково через синий. При этом в зеленых лучах звезда A ярче звезды B. Расположите эти три звезды в порядке возрастания их температуры.
        \item Даны видимые звездные величины звезд Вега ($\alpha$ Lyr), Альдебаран ($\alpha$ Tau) и Спика ($\alpha$ Vir), а также их показатели цвета $B-V$ и $U-B$. Определите для каждой звезды звездные величины в фильтрах $U$ и $B$.
    
        \begin{center}
        \begin{tabular}[t]{|l|c|c|c|}
            \hline
            \hline
            Звезда &  $m_v$ & $B-V$ & $U-B$\\
            \hline
            $\alpha$ Lyr & $V=0.03^m$ & $(B-V) = + 0.00^m$ & $(U-B)=+0.00^m$\\
            \hline
            $\alpha$ Tau, & $V=0.86^m$ & $(B-V) = + 1.54^m$ & $(U-B)=+ 3.46^m$; \\
            \hline 
            $\alpha$ Vir & $V= 0.97^m$ & $(B-V)= -0.23^m$ & $(U-B)=- 1.17^m$ \\
            \hline
            \hline 
        \end{tabular}
        \end{center}
        \item Абсолютная звездная величина Солнца в фильтре $V$ -- $M_V=4.72^m$. Показатель цвета Солнца $(B-V)_{\odot}=0.67^m$. Определите абсолютную звездную величину Солнца в фильтра $B$?
        \item Определите расстояния от Солнца и параллаксы трех звезд созвездия Большой Медведицы (UMa) по их блеску в фильтре $V$ и абсолютной звездной величине в фильтре $B$.
        
        \begin{center}
        \begin{tabular}[t]{|l|c|c|c|}
            \hline
            \hline
        Звезда &  $m_v$ & $B-V$ & $M_B$\\
        \hline
        $\alpha$ UMa & $V=1.79^m$ & $(B-V) = + 1.07^m$ & $M_B=+0.32^m$\\
        \hline
        $\delta$ UMa, & $V=3.31^m$ & $(B-V) = + 0.08^m$ & $M_B=+ 1.97^m$; \\
        \hline 
        $\eta$ UMa & $V= 1.86^m$ & $(B-V)= -0.19^m$ & $M_B=- 5.32^m$ \\
            \hline
            \hline 
        \end{tabular}
        \end{center}
        \item Как астрономы различают горячие звезды, покрасневшие в результате  межзвездного поглощения света, и действительно красные холодные звезды?
        \item Из наблюдений было получено, что $V = 1.8^m$, а годичный параллакс звезды составил $\pi=0.02''$. Известно, что для данного типа звезд истинный показатель цвета $(B-V)_0=-0.3^m$, однако его измеренное значение оказалось равным $(B-V)=0.5^m$. Найдите истинную $(M_V)_0$ и абсолютную болометрическую звездную величину $M_{bol}$, если известно, что для этого типа звезд болометрическая поправка $BC=-2.8^m$. Оцените спектральный класс звезды. 
        \item Видимая звездная величина звезды $V=15.1^m$, показатель цвета $B-V=1.6^m$, а абсолютная звездная величина $M_V=1.3^m$. Межзвездное поглощение в направлении звезды $1$ звездная величина на $1$ кпк. Определите изначальный показатель цвета звезды без учета межзвездного поглощения.
        \item Капелла А -- спектрально-двойная система. Первый компонент -- красный гигант класса K$0$III -- имеет звездную величину в фильтре $V=0.89^m$ и показатель цвета $(B-V)_{1}=0.93^m$. Второй компонент -- субгигант класса G$0$IV -- имеет звездную величину в фильтре $V_2=0.76^m$ и показатель цвета $(B-V)_{2}=0.67^m$. Определите показатель цвета спектрально-двойной Капелла А и ее звездную величину в фильтре B.
        \item Предположим, что Сириус вскоре погрузится в плотное облако межзвездной пыли. На сколько упадет его блеск в полосе $V$, если он станет такого же цвета, как и Арктур? Удельное поглощение в пыли обратно пропорционально длине волны в степени $1.33$. Длина волны середины диапазона $V$ -- $540$ нм, диапазона $B$ -- $442$ нм. Видимые звездные величины Сириуса и Арктура в полосе $V$ составляют $-1.46^m$ и $-0.04^m$, показатели цвета $0.00^m$ и $+1.23^m$ соответственно.
        \item Неразделимая для визуальных наблюдений двойная звезда состоит из двух звезд. Одна из которых похожа на Солнце и имеет показатель цвета $(B-V)_{1}=0.66^m$. Вторая звезда по спектральным характеристикам похожа на Альтаир и имеет показатель цвета $(B-V)_{2}=0.22^m$. Определите показатель цвета всей системы, если одна из звезд ярче второй на $2.5^m$ звездные величины. Межзвездным поглощением пренебречь. 
        \item Наблюдаемый показатель цвета звезды $B-V$ равен $0.22$, но он искажён поглощением межзвёздной пылью, которая ослабляет свет звезды. В спектральном диапазоне $B$ свет ослабляется в $\alpha_B=2.5$ раза, в диапазоне $\alpha_V$ в $A_V=2$ раза. Найдите истинный показатель цвета звезды (в отсутствие поглощения). К какому классу может принадлежать эта звезда?
        \item Две звезды имеют одинаковые физические характеристики, наблюдаются на небе рядом друг с другом, но расстояния до них различаются. Обе звезды и наблюдатель находятся в однородном облаке межзвездной пыли. Фотометрические измерения этих звезд в полосе $B$ дали результат $11^m$ и $17^m$, в полосе $V$: $10^m$ и $15^m$. Во сколько раз одна из звезд дальше от нас, чем другая? Считать, что поглощающая способность пыли пропорциональна длине волны в степени $-1.3$.

        \item На рисунке представлена кривая блеска двойной звезды, полученная в фильтре~$V$. Зная, что затмения в системе центральные, один из компонентов двойной имеет спектральный класс A$0$, а второй -- G$2$, и оба компонента являются звёздами главной последовательности, постройте кривую изменения показателя цвета $B-V$ этой системы. Ось ординат Вашего графика направьте вверх, нанесите деления и поставьте соответствующие значения показателей цвета.

        \begin{figure}[ht]
            \begin{tikzpicture}
            \begin{axis}[
            table/col sep = comma,
            height = 0.3\paperheight, 
            width = 0.7\paperwidth,
            minor x tick num = 1,
            minor y tick num = 9,
            ytick = {10,10.1,10.2},
            ymax = 10.12,
            xlabel = {время $t$, условные единицы},
            ylabel = {блеск в фильтре $V$},
            mark size = 0 pt,
            y dir=reverse
            ]
            
            \addplot[black] coordinates {
            (-3,9.984)
            (-2,9.984)
            (-1,10.098)
            (1,10.098)
            (2,9.984)
            (9,9.984)
            (10,10)
            (12,10)
            (13,9.984)
            (20,9.984)
            (21,10.098)
            (23,10.098)
            (24,9.984)
            (25,9.984)};
            \end{axis}
            \end{tikzpicture}
        \end{figure}
        
        \item На диаграмме показано соотношение показателей цвета $U-B$ и $B-V$ для звезд до $6.5^m$ в полосе $V$. Найдите расстояние до одиночных звезд $1$, $2$ и $3$, расположенных в диске Галактики, отмеченных на диаграмме. Межзвездное поглощение в окрестностях Солнца в диске Галактики в полосе $V$ составляет $A_V=2^m/\text{кпк}$ и зависит от длины волны как 
        $$
        A\propto\lambda^{-1.3}
        $$
        \vspace{-25pt}
        \begin{figure}[h]
            \centering
            \begin{tikzpicture}[dot/.style = {draw, fill = black, color = black, circle, inner sep=1.5pt}, >=stealth]
                \begin{axis}[
                    table/col sep = semicolon,
                    width = 0.8\paperwidth,
                    xlabel = {$B-V$},
                    ylabel = {$U-B$},
                    grid=both,
                    y dir = reverse,
                    ymin=-1.5, ymax=2.5,
                    xmin=-0.5, xmax=2.0,
                    ytick={-1.5,-1,...,2.5}
                    ]
                    
                    \addplot+[gray,only marks,mark = *, 
                    mark options = {
                        scale = 0.3, 
                        fill = gray
                    }] table [y={U-B}, x={B-V}, col sep=comma] {bicolordiagram1.csv};
                    
                    \coordinate [dot,label=right:{$1$}] (A) at (0.40,-0.75);
                    \coordinate [dot,label=right:{$2$}] (B) at (1.10,0.60);
                    \coordinate [dot,label=right:{$3$}] (C) at (1.70,1.55);
                \end{axis}
            \end{tikzpicture}
        \end{figure}
 \end{enumerate}
\end{document}