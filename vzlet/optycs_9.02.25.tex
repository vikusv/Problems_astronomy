\documentclass[12pt]{article}
\usepackage{baseset}
\usepackage{myproblem}
\usepackage{stackengine}
\DeclareSymbolFont{operators}{OT1}{ntxtlf}{m}{n}
\SetSymbolFont{operators}{bold}{OT1}{ntxtlf}{b}{n}
\usepackage{wasysym}
\newcommand{\RomanNumeralCaps}[1]
{\MakeUppercase{\romannumeral #1}}
\usepackage{tabularx}
\usepackage{enumitem}

\begin{document}
\begin{tabularx}{\textwidth}{Xr}
{\Large \textbf{Оптика}} & Взлёт $09.02.2025$ \\
\end{tabularx}
\noindent\rule{\textwidth}{0.4pt}
\begin{enumerate}
    \item Определите диаметр   радиотелескопа, работающего на длине волны $\lambda=1$~см, с тем же разрешением, что и оптический телескоп с диаметром $D=10$ см?
    \item Расстояние между звездами в двойной системе $3$~а.е. Какой должен быть диаметр телескопа у наблюдателя, чтобы звезду можно было различить в этот телескоп. Расстояние до системы $21$~пк.
    \item Определите фотографический диаметр Марса, полученный с помощью рефрактора с фокусным расстоянием $13.9$ м, если в моменты великих противостояний Марс имеет угловой диаметр $24''$.
    \item При наблюдениях на телескопе с фокусным расстоянием $2.5$ м используется ПЗС-матрица размером $1024 \times 1024$ пикселов. Найдите размер одного пиксела, если на матрице получается изображение участка неба с угловыми размерами $20' \times 20'$. 
    \item Какой из двух телескопов с диаметром объектива $D$ и фокусным расстоянием $F$ нужно использовать для фотографирования двойной звезды с угловым расстоянием между компонентами $0.8''$, если размер пикселя на ПЗС-матрице $30$ мкм:
	\begin{itemize}
		\item $D = 35$ см, $F = 4$ м;
		\item $D = 10$ см, $F = 12$ м.
	\end{itemize}
    \item Звезда Сириус ($\alpha$ Большого Пса, $\delta=-16^{\circ}39'$) наблюдается в телескоп с диаметром $20$~см и относительным отверстием $1:15$. При одном окуляре эта звезда проходит диаметр поля зрения неподвижного телескопа за $1^m53^s$, а при другом за $38^s$. Определите фокусное расстояние окуляров и диаметры поля зрения телескопа при их применении.
    \item Астроном-любитель навел свой телескоп системы Шмидт-Кассегрен с диаметром входного отверстия $20$ см на Юпитер. Затем он надел на объектив телескопа крышку с отверстием посередине диаметром $15$ см и обнаружил, что яркость Юпитера упала в $2$ раза. Во сколько раз упадет яркость по сравнению с изначальной, если диаметр отверстия окажется $10$ см? Если $5$ см? Выразите эту величину в звездных величинах во всех трех случаях. Во сколько раз изменится видимый в телескоп угловой размер планеты в каждом случае?
    \item У астронома-любителя есть фотоаппарат с ПЗС-матрицей с квадратными пикселями, а также несколько объективов с различными фокусными расстояниями. В один из солнечных дней $2023$ года он решил понаблюдать пятна на Солнце. Оцените наименьшее возможное фокусное расстояние объектива, с которым на фотографии удастся зарегистрировать пятна на Солнце. Можно считать, что пятно станет заметным, если займет на снимке площадь не менее $4\times4$ пикселя. Общее количество пикселей камеры -- $30$ миллионов. Линейные размеры матрицы $36\times24$ мм.
    \item Каждый телескоп системы KELT (Kilodegree Extremely Little Telescope) оснащен линзовым объективом с диаметром $42$ мм и ПЗС-матрицей размером $37\times37$ мм, содержащей $4096\times4096$ пикселей. Поле зрения телескопа составляет $26^{\circ}\times26^{\circ}$. Максимальная чувствительность матрицы достигается на длине волны $600$ нм. Определите предельное угловое разрешение такого инструмента.
    \item Небольшое рассеянное скопление состоит из $40$ одинаковых звезд и имеет общий блеск $8^m$. Какой должен быть диаметр объектива телескопа, чтобы в него можно было увидеть отдельные звезды скопления? 
    \item Телескоп с диаметром объектива $5$ см и относительным отверстием $f/15$ укомплектован окулярами с фокусным расстоянием $60$ мм и $20$ мм. Какое увеличение обеспечивает использование каждого из окуляров с этим телескопом? Определите минимальное угловое разрешение, доступное для визуальных наблюдений с данными окулярами. Можно ли с его помощью разрешить двойную систему с расстоянием между компонентами $2''$. Считать, что разрешающая способность глаза равна $1'$. 
    \item При наблюдении невооруженным глазом некий близорукий человек в своих очках видит на пределе звезды $6^m$. В тех же условиях, но без очков он видит на пределе звезды $3^m$ звездной величины. Оцените разрешающую способность глаза этого наблюдателя без очков, если с использованием очком она равна $2'$.
    \item Один инопланетянин, оказавшийся случайно на сборах команды России по астрономии, пытается показать школьникам свою родную звезду. Он навел в нужную сторону телескоп (диаметр объектива $D = 150$ мм, фокусное расстояние $F = 450$ мм, фокусное расстояние окуляра $f = 30$ мм) и сказал, что звезда едва видна в центре поля зрения. Известно, что диаметр зрачков инопланетянина $\delta = 20$ мм, а невооруженным глазом он видит звезды до $m_A = 8^m$. Смогут ли участники сборов разглядеть звезду в этот телескоп?
    \item Студент-астроном проводит визуальные наблюдения за двойными звездами в студенческой обсерватории МГУ в рефракторе Цейсс-300 (диаметр объектива $250$-мм, относительное отверстие $1:3.8$). После наблюдений он решил навести телескоп на звезду главного здания МГУ (диаметр $7.5$ м). На сколько миллиметров ему надо сдвинуть фокус окуляра и в какую сторону? Помогите ему определить фокусное расстояние окуляра, в котором звезда займет все поле зрения целиком. Расстояние от телескопа до Главного здания МГУ -- $750$ метров. Поле зрения окуляра -- $60^{\circ}$.
    \item Один астроном после покупки бинокля заметил, что многие люди с плохим зрением не могут сфокусировать этот бинокль на бесконечность из-за ограниченного диапазона фокусировки. Чтобы это исправить, он решился на полную переделку узла фокусировки бинокля. Рассчитайте необходимый диапазон хода фокусировки (максимальное расстояние на которое может перемещаться окуляр), чтобы в этот бинокль могли без проблем наблюдать без очков люди как с близорукостью, так и с дальнозоркостью с очками не менее чем $\pm10$ диоптрий. Помните, что бинокль нужен для наблюдения не только бесконечно удаленных объектов! Можно считать что этот бинокль построен по схеме Кеплера из тонких линз. Необходимые для решения задачи данные можно найти в следующей таблице.
    \begin{table}[h]
        \centering
        \begin{tabular}{|l|c|}\hline
            Фокусное расстояние объектива бинокля & $200$ мм \\ \hline
            Диаметр объектива бинокля & $50$ мм \\ \hline
            Увеличение бинокля & $10^{\times}$ \\ \hline
            Минимальная необходимая дистанция фокусировки бинокля & $5$ м \\ \hline 
            Стандартное расстояние от глаза до линзы очков & $2$ см \\ \hline
            Минимальная дистанция фокусировки здорового глаза & $10$ см \\ \hline
        \end{tabular}
    \end{table}
    \item Один юный астроном все-таки испортил свое зрение и теперь ходит в очках. Исследуя новый аксессуар, он заметил, что без очков видит предметы четкими на расстояниях примерно от $10$ до $22$ см.
    \begin{enumerate}
        \item Помогите ему найти оптическую силу его очков, считая, что они подобраны правильно (диапазон фокусировки глаза используется полностью и в очках астроном четко видит бесконечно удаленные предметы).
        \item На каком минимальном расстоянии астроном будет четко видеть в очках?
        \item Оцените разрешение ничем не вооруженного, даже очками, глаза астронома при наблюдении удаленных объектов.
        Подсказка: Все линзы считать тонкими, и очки расположены примерно в $20$ мм от глаза. Диаметр зрачка $5$ мм, а расстояние от зрачка до сетчатки можно считать равным $20$ мм.
    \end{enumerate}
    \item Один астроном-любитель, проводя наблюдения в самодельный телескоп с окуляром из одиночной линзы, решил попробовать сфотографировать увиденное на камеру мобильного телефона. На каком расстоянии от линзы окуляра ему нужно располагать телефон? Астроном наблюдает в телескоп с фокусным расстоянием объектива $1.5$ м на увеличении в $50$ крат.
    \item Астроном хочет взять с собой в поход лупу для разведения огня. У одной линзы фокусное расстояние 20 см и диаметр 5 см, а у другой -- фокусное расстояние 50 см, а диаметр 10 см. Какой из них будет легче поджечь тонкую деревянную палочку? Во сколько раз будет отличаться время поджига, если пренебречь потерями тепла палочкой и аберрациями линз?
\end{enumerate}
\end{document}