\documentclass[12pt]{article}
\usepackage{baseset}
\usepackage{myproblem}
\usepackage{stackengine}
\DeclareSymbolFont{operators}{OT1}{ntxtlf}{m}{n}
\SetSymbolFont{operators}{bold}{OT1}{ntxtlf}{b}{n}
\usepackage{wasysym}
\newcommand{\RomanNumeralCaps}[1]
{\MakeUppercase{\romannumeral #1}}
\usepackage{tabularx}
\usepackage{enumitem}

\begin{document}
\begin{tabularx}{\textwidth}{Xr}
{\Large \textbf{Антинаучная конференция}} &  \\
\end{tabularx}
\noindent\rule{\textwidth}{0.4pt}
\subsection*{Миссия}
\begin{itemize}
    \item Цель проекта -- показать применимость научного подхода к любой идее и важность критического мышления, а также популяризировать науку, заинтересовать участников и зрителей в изучении новых для них областей через юмористические презентации антинаучных гипотез и утверждений, объясняемых с использованием доказанных фактов.
    \item Главное -- подсветить, как уверенность и харизма докладчика и жонглирование научными терминами может вессти в заблуждение и заставить поверить даже в самые абсурдные утверждения.
    \item Сейчас, в годы процветания астрологии, нумерологии и инфоцыган, важно научить молодых людей мыслить критически, проверять информацию, разбираться и не верить каждой красивой сказке.
    \item Некоторые ВУЗы проводят антинаучную конференцию, и она получается довольно тематическая: например, только проекты по физике. В Вышке много специальностей, поэтому конференция может получиться разноплановой и освещающей разные области науки, от социологии до айти. Существует также возможность создать проект на стыке двух направлений, сформировав команду выступающих из обучающихся разных факультетов. А потребность в обсуждении при разработке проекта даст чтудентам возможность посетить разные корпуса Вышки, если раньше такая возможность не представилась.
\end{itemize} 
\subsection*{Формат}
Ежегодная конференция, на которой могут выступать как студены и преподаватели Вышки, так и гости, проявившие интерес к проекту. Целевая аудитория -- в основном молодые люди: школьники и студенты, интересующиеся наукой, друзья выступающих, но также и преподаватели, родители и просто неравнодушные к науке люди.

Каждый участник должен подготовить доклад с исследованием лженаучных заблуждений, преподносимых как серьезные проблемы науки. Презентация предполагает доказательство этих утверждений академическим языком: с использованием научных фактов, провессиональной терминологии, графиков и диаграмм, ссылок на исследования, но не ограничивает искаженных выводов, некорректных корреляций, подмену понятий и неринятия во внимание некоторых аргументов. Спикер может демонстрировать опыты и проводить интерактив с аудиторией для создания лучшего эффекта. Обязательный элемент -- юмор и интересная подача "исследования".

Пример тем для антинаучной конференции: "Высадку на Луну сняли в Голливуде", "Гомеопатия лечит все", "Аффирмации на успех" и т.д.

Можно пригластить в качестве участника или жюри известного популяризатора науки (Владимира Сурдина, например), это повысит охват и интерес аудитории.

После выступления докладчика аудитория может в формате "открытого микрофона" задать вопрос по рассказанному материалу. По окончании всех выступлений жюри выбирает несколько лучших проектов и проводится награждение. Оценивается оригинальность идеи, подача материала, логика выводов, ответы на вопросы аудитории.
\subsection*{Ожидаемые эффекты}
\begin{enumerate}
    \item \textbf{Популяризация науки через юмор.} Привлечение людей, несвязанных с наукой, или же занимающихся другими ее областями через любимый всеми канал -- юмор.
    \item \textbf{Разоблачение псевдонаучных теорий.} Приобщение общества к научному подходу.
    \item \textbf{Повышение уровня критического мышления.} Введение в привычку недоверия к увереренному рукомаханию и проверки фактов.
    \item \textbf{Развитие навыков публичных выступлений студентов Вышки.} Развитие soft skills, уверенности и раскрытие скрытых талантов.
    \item \textbf{Формирование сообщества и сплочение коллектива.} Поиск новых знакомств, создание группы, увлеченной научной деятельностью и выступлениями.
\end{enumerate}
\subsection*{Оценка охвата}
Предполагается примерно 20-25 спикеров, 5-7 членов жюри, 50-100 зрителей офлайн и около 10 тысяч просмотров записи конференции. С каждым годом охват может увеличиваться при должном подходе к организации мероприятия.
\subsection*{Календарный план, бюджет и оборудование}
\subsubsection*{Бюджет и оборудование}
\begin{itemize}
    \item Аренда площадки для проведения конференции -- 10000-15000 р.
    \item Аренда светового и аудио-оборудования -- 5000-10000 р.
    \item Реклама (баннеры, листовки) -- 5000 р.
    \item Оплата работы организаторов -- от 15000 р. на человека
    \item Оплата работы фотографа и видеографа -- 10000 р.
    \item Призовой фонд -- от 20000 р.
\end{itemize}
Кроме того, можно добавить мерч, нанять SMM-менеджера для ведения соц. сетей, организовать активности в день проведения мероприятия (мастер-классы, фуршет, фотозона).
\subsubsection*{Календарный план}
\begin{table}[h]
    \centering
    \begin{tabular}{|m{2.5 cm}|m{5 cm}|m{8 cm}|} \hline
        \textbf{Время} & \textbf{Этап} & \textbf{Описание} \\ \hline
        1 месяц & Подготовка & Сбор команды организаторов, обсуждение программы и бюджета конференции \\ \hline
        2-3 месяца & Запуск рекламы & Создание соцсетей конференции, анонс мероприятия, реклама в Вышке \\ \hline
        1 месяц (параллельно с рекламой) & Выбор площадки & Поиск места проведения, договоренности с предоставляющими, подготовка \\ \hline
        2 недели (параллельно с рекламой) & Сбор заявок & Запуск формы для регистрации на саму конференцию и в качестве жюри \\ \hline
        1 неделя (параллельно с рекламой) & Обработка заявок & Обработка заявок, отбор участников и судей \\ \hline
        1 день & Проведение конференции & Прослушивание докладов на выбранной площадке, подведение результатов, награждение \\ \hline
        2 недели & Обработка материалов & Обработка и размещение в социальных сетях фото- и видеоматериалов с мероприятия \\ \hline
    \end{tabular}
\end{table}
\end{document}