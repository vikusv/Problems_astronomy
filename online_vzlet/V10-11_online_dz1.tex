\documentclass[12pt]{article}
\usepackage{baseset}
\usepackage{myproblem}
\usepackage{stackengine}
\DeclareSymbolFont{operators}{OT1}{ntxtlf}{m}{n}
\SetSymbolFont{operators}{bold}{OT1}{ntxtlf}{b}{n}
\usepackage{wasysym}
\newcommand{\RomanNumeralCaps}[1]
{\MakeUppercase{\romannumeral #1}}
\usepackage{tabularx}
\usepackage{enumitem}

\begin{document}
\begin{tabularx}{\textwidth}{Xr}
{\Large \textbf{Сферическая тригонометрия}} & Дедлайн -- $29.11.2023$ \\
\end{tabularx}
\noindent\rule{\textwidth}{0.4pt}
\begin{enumerate}
    \item \textbf{!} Определите угловое расстояние между Канопусом ($\alpha =  6^h24^m$, $\delta = -52^{\circ}42'$) и Капеллой ($\alpha = 5^h17^m$6 $\delta = 46^{\circ}00'$).
    \item \textbf{!} Определите самую северную точку, надо которой будет пролетать самолет по пути из Лондона ($\varphi = 51^{\circ}30'$, $\lambda = 0^{\circ}08'$) в Австралию ($\varphi = -35^{\circ}17'$, $\lambda = 149^{\circ}08'$). 
    \item Определите координаты самой северной равноудаленную от городов из предыдущей задачи точку.
    \item Под каким углом Солнце заходит за горизонт на широте $60^{\circ}$ в дни солнцестояний? Рассчитайте длительность восхода/захода Солнца плоским приближением и с использованием сферической тригонометрии. Сравните результаты.
    \item Звездочёт из Цзяи с острова Тайвань ($\varphi = 23.5^{\circ}$ с. ш., $\lambda = 120.4^{\circ}$ в. д., GMT$+8$) в $21:00$ $25.09.2020$ будет иметь удовольствие наблюдать два метеора. Путь первого метеора начался на высоте $15^{\circ}$ над точкой севера и завершился на горизонте в точке востока. Второй метеор пронёсся от точки с высотой $23.5^{\circ}$ и азимутом $210^{\circ}$ к точке с высотой $75^{\circ}$ и азимутом $255^{\circ}$.
    Найдите альт-азимутальные и экваториальные координаты радианта этих двух метеоров.
    \item Наибольшая высота, которой достигает некоторая звезда в Санкт-Петербурге ($\varphi = 60^{\circ}$) составляет $35^{\circ}$. Определите высоту этой звезды в момент, когда ее астрономический азимут $A = 90^{\circ}$.
\end{enumerate}
\end{document}