\documentclass[12pt]{article}
\usepackage{baseset}
\usepackage{myproblem}
\usepackage{stackengine}
\DeclareSymbolFont{operators}{OT1}{ntxtlf}{m}{n}
\SetSymbolFont{operators}{bold}{OT1}{ntxtlf}{b}{n}
\usepackage{wasysym}
\newcommand{\RomanNumeralCaps}[1]
{\MakeUppercase{\romannumeral #1}}
\usepackage{tabularx}
\usepackage{enumitem}

\begin{document}
\begin{tabularx}{\textwidth}{Xr}
{\Large \textbf{Телескопы}} & Дедлайн -- $12.01.2024$ \\
\end{tabularx}
\noindent\rule{\textwidth}{0.4pt}
    \begin{enumerate}
        \item Чему равно равнозрачковое увеличение телескопа с диаметром объектива $120$ мм?
        \item Рассчитайте разрешающую способность наблюдений с оптическим телескопом с диаметром объектива $200$ мм. Увеличение равнозрачковое.
        Среднюю длину волны оптического диапазона принять равной $550$ нм.
        \item Чему равен диаметр объектива телескопа, если его относительное отверстие $1:15$, а фокусное расстояние равно $3$ м?
        \item Звезды какой звездной величины можно наблюдать в телескоп с диаметром объектива $10$ см?
        \item Телескоп с диаметром объектива $6$ см и относительным отверстием $f/15$ укомплектован окулярами с фокусным расстоянием $60$ мм и $24$ мм. Какое увеличение обеспечивает использование каждого из окуляров с этим телескопом? Определите минимальное угловое разрешение, доступное для визуальных наблюдений с данными окулярами. Можно ли с их помощью разрешить двойную систему с расстоянием между компонентами $2''$? Считать, что разрешающая способность глаза равна $1'$.
        \item В телескоп с диаметром $20$ см и фокусным расстоянием $1000$ мм фотографируют Марс в момент великого противостояния (расстояние между Марсом и Землей $0.38$ а.е.) на ПЗС-матрицу с размером пикселя $5$ мкм. Сколько пикселей занимает Марс? Сколько фотонов будет в каждом пикселе при выдержке $1/200$ секунды?
        
        Считайте, что от звезды нулевой звездной величины приходит $10^6$ фотонов за $1$ секунду на $1$ см$^2$. Звездная величина Марса во время великих противостояний $-2.9^m$.
        \item Небольшое рассеянное скопление состоит из $50$ одинаковых звезд и имеет общий блеск $6^m$. Какой должен быть диаметр объектива телескопа, чтобы в него можно было увидеть отдельные звезды скопления?
        \item Толщина диска нашей Галактики составляет $800$ световых лет. Солнце находится вблизи плоскости Млечного пути. Оцените, сколько звезд со светимостью порядка солнечной можно увидеть со всей Земли в телескоп ТАЛ-1 с диаметром главного зеркала $110$ мм? Считать концентрацию таких звезд в диске Млечного пути постоянной и равной $0.01$ пк$^{-3}$. Межзвездным поглощением света пренебречь.
        \item Вечером $9$ мая $2016$ года состоится редкое астрономическое явление -- прохождение Меркурия по диску Солнца, которое будет хорошо видно в Европейской части России. Для его наблюдения телескоп оснастили солнечным экраном, на котором изображение Солнца имеет диаметр $15$ см. Какого диаметра на этом экране будет пятно -- изображение Меркурия? Считать, что во время явления Меркурий будет располагаться в афелии своей орбиты, а орбита Земли круговая.
        \item Расскажите об отличиях телескопов системы Галилея, Кеплера и Ньютона.
    \end{enumerate}
\end{document}