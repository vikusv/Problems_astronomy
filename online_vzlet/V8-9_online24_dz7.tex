\documentclass[12pt]{article}
\usepackage{baseset}
\usepackage{myproblem}
\usepackage{stackengine}
\DeclareSymbolFont{operators}{OT1}{ntxtlf}{m}{n}
\SetSymbolFont{operators}{bold}{OT1}{ntxtlf}{b}{n}
\usepackage{wasysym}
\newcommand{\RomanNumeralCaps}[1]
{\MakeUppercase{\romannumeral #1}}
\usepackage{tabularx}
\usepackage{enumitem}
\usetikzlibrary{hobby}

\begin{document}
 \begin{tabularx}{\textwidth}{Xr}
  {\Large \textbf{Многоцветная фотометрия}} & Дедлайн -- $1.12.2024$ \\
 \end{tabularx}
 \noindent\rule{\textwidth}{0.4pt}
 \begin{enumerate}
    \item Определите расстояния от Солнца и параллаксы трех звезд созвездия Большой Медведицы (UMa) по их блеску в фильтре $V$ и абсолютной звездной величине в фильтре $B$.
    
    \begin{center}
        \begin{tabular}[t]{|l|c|c|c|}
            \hline
            \hline
        Звезда &  $m_v$ & $B-V$ & $M_B$\\
        \hline
        $\alpha$ Aql & $V=0.77^m$ & $(B-V) = + 0.22^m$ & $M_B=+2.44^m$\\
        \hline
        $\alpha$ Vir, & $V=0.97^m$ & $(B-V) = - 0.23^m$ & $M_B=- 3.78^m$; \\
        \hline 
        $\gamma$ Ori & $V= 1.64^m$ & $(B-V)= -0.21^m$ & $M_B=- 2.99^m$ \\
            \hline
            \hline 
        \end{tabular}
        \end{center}
    \item Видимая звездная величина звезды $V=13.5^m$, показатель цвета $B-V=1.8^m$, а абсолютная звездная величина $M_V=0.9^m$. Межзвездное поглощение в направлении звезды в фильтре $V$ -- $1^m$/кпк. Определите изначальный показатель цвета звезды без учета межзвездного поглощения.
    \item Предположим, что Сириус вскоре погрузится в плотное облако межзвездной пыли. На сколько упадет его блеск в полосе $V$, если он станет такого же цвета, как и Арктур? Удельное поглощение в пыли обратно пропорционально длине волны в степени $1.33$. Длина волны середины диапазона $V$ -- $540$ нм, диапазона $B$ -- $442$ нм. Видимые звездные величины Сириуса и Арктура в полосе $V$ составляют $-1.46^m$ и $-0.04^m$, показатели цвета $0.00^m$ и $+1.23^m$ соответственно.
    \item Неразделимая для визуальных наблюдений двойная звезда состоит из двух звезд. Одна из которых похожа на Солнце и имеет показатель цвета $(B-V)_{1}=0.66^m$. Вторая звезда по спектральным характеристикам похожа на Альтаир и имеет показатель цвета $(B-V)_{2}=0.22^m$. Определите показатель цвета всей системы, если одна из звезд ярче второй на $2.5^m$ звездные величины. Межзвездным поглощением пренебречь. 
    \item Наблюдаемый показатель цвета звезды $B-V$ равен $0.22$, но он искажён поглощением межзвёздной пылью, которая ослабляет свет звезды. В спектральном диапазоне $B$ свет ослабляется в $\alpha_B=2.5$ раза, в диапазоне $\alpha_V$ в $A_V=2$ раза. Найдите истинный показатель цвета звезды (в отсутствие поглощения). К какому классу может принадлежать эта звезда?
 \end{enumerate}
\end{document}