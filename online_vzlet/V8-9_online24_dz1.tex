\documentclass[12pt]{article}
\usepackage{baseset}
\usepackage{myproblem}
\usepackage{stackengine}
\DeclareSymbolFont{operators}{OT1}{ntxtlf}{m}{n}
\SetSymbolFont{operators}{bold}{OT1}{ntxtlf}{b}{n}
\usepackage{wasysym}
\newcommand{\RomanNumeralCaps}[1]
{\MakeUppercase{\romannumeral #1}}
\usepackage{tabularx}
\usepackage{enumitem}

\begin{document}
 \begin{tabularx}{\textwidth}{Xr}
  {\Large \textbf{Закон Стефана-Больцмана. Закон смещения Вина}} & Дедлайн -- $12.10.2024$ \\
 \end{tabularx}
 \noindent\rule{\textwidth}{0.4pt}
 \begin{enumerate}
        \item Какова должна быть температура звезды, если при одинаковых с Солнцем размерах ее светимость в $81$ раз больше?

        \item Звезда Фомальгаут имеет видимую звездную величину $1.16^m$ и параллакс $0.130''$. Определите радиус звезды, если температура ее поверхности $8~590$~K.
        
        \item Угловой диаметр звезды Бетельгейзе составляет $0.047''$, а ее болометрическая звездная величина $-2^m$. Определите эффективную температуру Бетельгейзе.

        \item Некоторую звезду разбили на $8$ звезд такой же плотности и температуры. Определите, во сколько раз увеличилась суммарная светимость.
        
        \item Солнце внезапно увеличило свою массу на $30\%$, а плотность и видимая звездная величина остались прежними.
        \begin{enumerate}
            \item Поглотит ли Солнце какие-нибудь планеты Солнечной системы?
            \item Чему станет равна эффективная температура Солнца после «потолстения»? 
            \item Солнце станет более красным или более белым?
        \end{enumerate}

        \item Звезда $\beta$ Золотой Рыбы -- переменная класса цефеид с периодом пульсации $P = 9.84$ сут. Предположим, что звезда является наиболее яркой в момент наибольшего сжатия (радиус $R_1$) и наиболее слабой в момент наибольшего расширения (радиус $R_2$), сохраняет сферическую форму и ведёт себя подобно абсолютно чёрному телу в каждый момент в течение всего цикла пульсаций. Болометрическая звёздная величина этой звезды меняется от $3.46^m$ до $4.08^m$. По измерениям доплеровского смещения известно, что в течение периода пульсаций поверхность звезды сжимается и расширяется со средней радиальной скоростью $v = 12.8$ км/с; спектральный максимум излучения колеблется от $\lambda_1 = 531.0$ нм до $\lambda_2 = 649.1$ нм.
        \begin{enumerate}
            \item Найдите отношение радиусов звезды $R_1/R_2$ в моменты наибольшего сжатия и наибольшего расширения и оцените величины этих радиусов.
            \item Вычислите поток $F_2$ от звезды в момент её наибольшего расширения.
            \item Определите расстояние $D$ до звезды.
        \end{enumerate}
        
        \item Радиус Кастора равен $2.3$ радиусам Солнца, температура -- $9900$ К, а видимая звездная величина -- $1.58^m$.
        \begin{enumerate}
            \item Определите длину волны максимума излучения Кастора. Какая звезда краснее: Кастор или Солнце?
            \item Посчитайте светимость Кастора, ответ выразите в светимостях Солнца.
            \item Каков угловой размер Кастора и расстояние до него?
        \end{enumerate}
 \end{enumerate}
\end{document}