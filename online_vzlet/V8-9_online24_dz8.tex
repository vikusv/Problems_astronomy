\documentclass[12pt]{article}
\usepackage{baseset}
\usepackage{myproblem}
\usepackage{stackengine}
\DeclareSymbolFont{operators}{OT1}{ntxtlf}{m}{n}
\SetSymbolFont{operators}{bold}{OT1}{ntxtlf}{b}{n}
\usepackage{wasysym}
\newcommand{\RomanNumeralCaps}[1]
{\MakeUppercase{\romannumeral #1}}
\usepackage{tabularx}
\usepackage{enumitem}
\usetikzlibrary{hobby}

\begin{document}
 \begin{tabularx}{\textwidth}{Xr}
  {\Large \textbf{Эффект Доплера}} & Дедлайн -- $7.12.2024$ \\
 \end{tabularx}
 \noindent\rule{\textwidth}{0.4pt}
 \begin{enumerate}
   \item Линия водорода $H_{\alpha}$  в спектре галактики имеет длину волны $7500$~\AA. Найдите расстояние до галактики. Лабораторная длина волны линии $H_{\alpha}$ равна $6563$~\AA.
   \item Наклон линий солнечного спектра, наблюдаемых в спектре восточного и западного краев Сатурна, указывает на скорость $19.7$ км/с на экваторе. Определить радиус Сатурна, если наблюдаемый на экваторе его период вращения равен $10^h32^m$.
   \item Звезда находится на эклиптике. Как будут отличаться ее лучевые скорости $15$ октября и $13$ апреля? Эклиптическую долготу звезды примите равной $30^{\circ}$.
   \item При наблюдениях двойной системы, состоящей из нейтронной звезды массой $1.4$ массы Солнца и звезды главной последовательности, были обнаружены рентгеновские пульсации со средним периодом $1$ секунда, отклоняющиеся от него максимум на $10^{-4}$ секунды. При этом спектральные наблюдения в оптическом диапазоне показали, что линия $H_{\alpha}$ также периодически меняет длину волны, отклоняясь от среднего значения максимум на $0.5$~\AA. Оцените светимость такой системы в оптическом диапазоне.

   \textit{Подсказка:} за рентгеновские пульсации ответственна нейтронная звезда, поскольку звезды главной последовательности в рентгеновском диапазоне излучают очень слабо, а за оптическое излучение -- звезда-компаньон (поскольку нейтронные звезды в оптике практически ненаблюдаемы)
   \item Период пульсара в Крабовидной туманности составляет $0.0334$ секунды. В каких пределах и с какой периодичностью будет изменяться значение этого периода, измеренное на Земле? Когда оно будет достигать максимума и минимума?
   \item Самая яркая звезда рассеяного звездного скопления Плеяды (М45), Альциона, является кратной звездой. Звезда Альциона-А принадлежит к классу Ве-звезд и имеет скорость вращения $215$ км/с. Определите уширение линии водорода $H_{\alpha}$ в спектре этой звезды. Масса звезды $6 M_{\odot}$.
 \end{enumerate}
\end{document}