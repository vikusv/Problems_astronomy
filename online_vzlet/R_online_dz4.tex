\documentclass[12pt]{article}
\usepackage{baseset}
\usepackage{myproblem}
\usepackage{stackengine}
\DeclareSymbolFont{operators}{OT1}{ntxtlf}{m}{n}
\SetSymbolFont{operators}{bold}{OT1}{ntxtlf}{b}{n}
\usepackage{wasysym}
\newcommand{\RomanNumeralCaps}[1]
{\MakeUppercase{\romannumeral #1}}
\usepackage{tabularx}
\usepackage{enumitem}

\begin{document}
 \begin{tabularx}{\textwidth}{Xr}
  {\Large \textbf{Звездные величины объектов Солнечной системы}} & Дедлайн -- $27.11.2023$ \\
 \end{tabularx}
 \noindent\rule{\textwidth}{0.4pt}
 \begin{enumerate}
        \item Определите звездную величину Венеры в элонгации и в верхнем соединении.
        \item Определите звездную величину Сатурна в противостоянии, соединении и в квадратуре. 
        \item Наблюдения проводятся на Венера. Во сколько раз отличается освещенность, создаваемая Землей в противостоянии и в квадратуре?
        \item Определите звездную величину Меркурия при наблюдении с Солнца.
        \item Определите звездную величину Земли для наблюдателя на Луне.
        \item Определите диапазон возможных звездных величин объекта в поясе Койпера ($30-55$~а.е.) с диаметром $1000$ км и сферическим альбедо $0.07$.
        \item Определите альбедо астероида, который при наблюдении с Земли в соединении виден с блеском $6^m$. Большая полуось астероида -- $3$ а.е.
        \item Астероид радиусом $50$ метров в некоторый момент времени находился на расстоянии $0.866$ а.е. от Солнца и при наблюдении с Земли угол между астероидом и Солнцем составлял $60^{\circ}$. Оцените видимую звездную величину астероида в этот момент. Оптические свойства поверхности астероида такие же, как у Луны.
        \item В настоящее время ведутся поиски возможной девятой планеты Солнечной системы, которая может иметь диаметр в $10$ диаметров Земли и располагаться в $280$ а.е. от Солнца. Астероид какого диаметра в главном поясе будет иметь такую же яркость на Земле в противостоянии, как и эта планета? Отражательную способность поверхности астероида считать аналогичной лунной, а планеты -- аналогичной Нептуну. Оба тела располагаются в плоскости эклиптики.
        \item Чему равна звездная величина Луны в новолунии?
 \end{enumerate}
\end{document}