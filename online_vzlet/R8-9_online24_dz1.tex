\documentclass[12pt]{article}
\usepackage{baseset}
\usepackage{myproblem}
\usepackage{stackengine}
\DeclareSymbolFont{operators}{OT1}{ntxtlf}{m}{n}
\SetSymbolFont{operators}{bold}{OT1}{ntxtlf}{b}{n}
\usepackage{wasysym}
\newcommand{\RomanNumeralCaps}[1]
{\MakeUppercase{\romannumeral #1}}
\usepackage{tabularx}
\usepackage{enumitem}

\begin{document}
 \begin{tabularx}{\textwidth}{Xr}
  {\Large \textbf{Начала сферической астрономии}} & Дедлайн -- $3.11.2024$ \\
 \end{tabularx}
 \noindent\rule{\textwidth}{0.4pt}
 \begin{enumerate}
        \item Определите координаты точки на Земле, которая диаметрально противоположна точке с координатами $60^{\circ}$~з.д., $10^{\circ}$~с.ш.
        \item Каково расстояние между указанными точками (кратчайшее расстояние, измеренное по поверхности Земли, принимаемой за идеальный шар): $60^{\circ}$~ю.ш. $30^{\circ}$~в.д. и $60^{\circ}$~с.ш. $150^{\circ}$~з.д.? 
        \item Какова длина одного градуса параллели на широте Санкт-Петербурга ($\varphi=60^{\circ}$ с.ш.)?
        \item Выразите углы $117^{\circ}25'$, $28^{\circ}43'$ и $49^{\circ}11'11''$  в часовой мере.
        \item Выразите углы $7^h 15^m$, $3^h 49^m 20^s$ и $14^h 21^m 51^s$  в градусной мере. 
        \item Астроном записал прямое восхождение двух звезд $\alpha_{1}=108^{\circ}18'40''$ и $\alpha_{2}=7^h 05^m 18^s$ в разных формах записи. У какой звезды прямое восхождение меньше?
        \item Определите, чему равны склонения и часовые углы точек севера и юга.
        \item В наблюдательном дневнике астронома имеются следующие записи: высота над горизонтом звезды А в кульминации $47^{\circ}$, зенитное расстояние звезды В в кульминации $48^{\circ}$. Какая звезда была ближе к горизонту в момент своей кульминации. 
        \item Укажите точки на небесной сфере, где небесный меридиан пересекает горизонт. Как они называются?
        \item Корабль плывет вдоль меридиана. Моряк при помощи секстанта измеряет высоту Полярной звезды. За сутки ее высота изменилась с $65^{\circ}$ до $72^{\circ}$. С какой скоростью плывет корабль и в каком направлении, если считать, что его скорость постоянна? 
        \item Cформулируйте условие, при котором астрономический азимут светила никогда не будет равен $180^{\circ}$.
        \item Определите широты, на которых могут быть видны звезды северного полушария.
 \end{enumerate}
\end{document}