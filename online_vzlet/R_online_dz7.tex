\documentclass[12pt]{article}
\usepackage{baseset}
\usepackage{myproblem}
\usepackage{stackengine}
\DeclareSymbolFont{operators}{OT1}{ntxtlf}{m}{n}
\SetSymbolFont{operators}{bold}{OT1}{ntxtlf}{b}{n}
\usepackage{wasysym}
\newcommand{\RomanNumeralCaps}[1]
{\MakeUppercase{\romannumeral #1}}
\usepackage{tabularx}
\usepackage{enumitem}

\begin{document}
\begin{tabularx}{\textwidth}{Xr}
{\Large \textbf{Эффект Доплера. Закон Хаббла}} & Дедлайн -- $21.12.2023$ \\
\end{tabularx}
\noindent\rule{\textwidth}{0.4pt}
    \begin{enumerate}
        \item Линия водорода $H_{\alpha}$  в спектре галактики имеет длину волны $7500$~\AA. Найдите расстояние до галактики. Лабораторная длина волны линии $H_{\alpha}$ равна $6563$~\AA.
        \item Определите расстояние до Галактики, если она удаляется от нас со скоростью $7000$ км/с.
        \item В галактике, красное смещение линий в спектре которой соответствует скорости $2000$ км/с, вспыхнула сверхновая звезда. Ее яркость в максимуме была равна $14^m$. Определите абсолютную звездную величину и светимость сверхновой.
        \item Спиральная галактика с красным смещением $0.05$ видна на Земле как узкая полоска длиной $3$ угловые минуты. Лучевая скорость краевых областей галактики отличается от лучевой скорости ее центра на $50$ км/с. Оцените массу галактики.
        \item Галактика A имеет красное смещение $0.07$. Галактика B, расположенная на небе в $120$ градусах от галактики A, имеет красное смещение $0.02$. Какое красное смещение будет иметь галактика B для наблюдателя в галактике A? 
        \item Галактика, похожая на нашу Галактику Млечный Путь, имеет красное смещение $0.01$. На угловом расстоянии $5'$ от нее виден ее спутник -- карликовая галактика. Оцените период ее обращения вокруг большой галактики.
        \item На какой длине волны приходит к нам излучение атомов межзвёздного водорода от галактики, удалённой на расстояние $750$ Мпк? (Длина волны неподвижного источника -- $21$ см).
        \item Оцените светимость квазара $3$C $48$, если его видимая звездная величина составляет $16.2^m$, а наблюдаемое красное смещение $z=0.3$.
        \item Смещение линии $H_{\gamma}$ ($4341$~\AA) составляет $5$ ангстрем. Определите скорость движения источника.
        \item В галактике на расстоянии $44$ Мпк наблюдается мазерный радиоисточник (излучающий	на фиксированной длине волны), двигающийся вблизи центральной черной дыры. Орбита источника перпендикулярна картинной плоскости, а большая ось лежит в картинной плоскости. Угловые размеры орбиты источника составляют $0.0005''$, относительное смещение ($\Delta \lambda/\lambda$) спектральных линий относительно лабораторной длины волны за вычетом скорости центра масс составляет $0.008$. Определите скорость движения центра масс и массу центральной черной дыры.
    \end{enumerate}
\end{document}