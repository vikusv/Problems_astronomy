\documentclass[12pt]{article}
\usepackage{baseset}
\usepackage{myproblem}
\usepackage{stackengine}
\DeclareSymbolFont{operators}{OT1}{ntxtlf}{m}{n}
\SetSymbolFont{operators}{bold}{OT1}{ntxtlf}{b}{n}
\usepackage{wasysym}
\newcommand{\RomanNumeralCaps}[1]
{\MakeUppercase{\romannumeral #1}}
\usepackage{tabularx}
\usepackage{enumitem}

\begin{document}
 \begin{tabularx}{\textwidth}{Xr}
  {\Large \textbf{Температуры и звездные величины объектов Солнечной системы}} & Дедлайн -- $9.11.2024$ \\
 \end{tabularx}
 \noindent\rule{\textwidth}{0.4pt}
 \begin{enumerate}
        \item Определите равновесную температуру Луны.
        \item На далекой обитаемой планете тепловые условия аналогичны земным, но местное Солнце имеет вдвое меньший угловой диаметр. Найдите температуру этой далекой звезды.
        \item Равновесная температура на планете в течение $2.5$ лет меняется в $1.5$ раза. Какова светимость звезды и эксцентриситет орбиты планеты, если альбедо планеты $0.36$, а средняя температура планеты в периастре составляет $0^{\circ}$~C. Считайте, что звезда принадлежит главной последовательности.
        \item Вблизи звезды HD$209458$ спектрального класса G$0$V обнаружена планета HD$209458$b с круговой орбитой и парами воды в атмосфере. Угловой радиус этой звезды при наблюдении с данной планеты составляет $6.61^{\circ}$. Найдите сферическое альбедо планеты, если ее эффективная температура $1130$~К.
        \item Определите звездную величину Земли в западной квадратуре при наблюдении с Венеры.
        \item В момент каждого противостояния астероида земной наблюдатель измеряет его видимую звездную величину. Период обращения астероида равен $3.9$ года. Оцените эксцентриситет его орбиты, если амплитуда изменения видимой звездной величины составляет $2.5^m$. Орбиту Земли считаем круговой.
        \item Транснептуновый объект $(174567)$ Варда в настоящее время имеет видимую звездную	величину $21^m$ (при наблюдении с Земли) и находится на расстоянии $48$~а.е. от Солнца.	Оцените диаметр Варды, если ее поверхность отражает $10\%$ падающего на нее света. Видимая звездная величина Солнца (также при наблюдении с Земли) составляет -- $27^m$.
        \item Вокруг далёкой звезды обращаются три экзопланеты, причем разумные наблюдатели обитают лишь на второй. Большие полуоси орбит планет соотносятся как $1:4:9$. Орбиты планет круговые. Альбедо планет соотносятся как $3:5:4$. Третья находится в восточной квадратуре при наблюдении с первой. Первая в западной элонгации при наблюдения со второй. Все планеты сферической формы. Их радиусы соотносятся как $15:16:25$. Какая планета окажется ярче для наблюдателей на второй, и на сколько звездных величин?
 \end{enumerate}
\end{document}