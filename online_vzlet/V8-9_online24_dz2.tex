\documentclass[12pt]{article}
\usepackage{baseset}
\usepackage{myproblem}
\usepackage{stackengine}
\DeclareSymbolFont{operators}{OT1}{ntxtlf}{m}{n}
\SetSymbolFont{operators}{bold}{OT1}{ntxtlf}{b}{n}
\usepackage{wasysym}
\newcommand{\RomanNumeralCaps}[1]
{\MakeUppercase{\romannumeral #1}}
\usepackage{tabularx}
\usepackage{enumitem}

\begin{document}
 \begin{tabularx}{\textwidth}{Xr}
  {\Large \textbf{Звездные величины}} & Дедлайн -- $26.10.2024$ \\
 \end{tabularx}
 \noindent\rule{\textwidth}{0.4pt}
 \begin{enumerate}
        \item Определите, во сколько раз меняется освещенность от Марса, если его видимая звездная величина меняется от $+2.0^m$ до $-2.9^m$.
        \item Компактное рассеянное звездное скопление состоит из $100$ одинаковых звезд и с трудом видно на небе глазом как маленькое пятнышко. Какой видимый блеск имеет каждая из звезд?

        \item На рисунке приведена кривая блеска  затменно-переменной звезды. Определите по графику блеск компонентов двойной системы.

        \begin{figure}[h] 	
            \centering
            \begin{tikzpicture} 
            \begin{axis}[xlabel=время,
            ylabel=звездная величина, 
            grid=major, 
            y dir=reverse, 
            width=10cm, 
            legend pos = south west,
            ymax=2.7]
            \addplot[line width=1pt, black] coordinates {
                (0, 1.5)
                (1, 1.5)
                (1.9, 2)
                (2.1, 2)
                (3, 1.5)
                %(4, 2)
                (4, 1.5)
                (4.9, 2.5)
                (5.1, 2.5)
                (6, 1.5)
                (7, 1.5)
                (7.9, 2)
                (8.1, 2)
                (9, 1.5)
                (10, 1.5)
                %(12, 2)
                (10.9, 2.5)
                (11.1, 2.5)
                (12, 1.5)
                (13, 1.5)
            };
        
            \end{axis}
            \end{tikzpicture} 
        \end{figure}
        
        \item Вычислите абсолютную звёздную величину Антареса, если его параллакс $\pi = 0.0059''$, а видимая звёздная величина $m = 0.91^m$.

        \item В некотором созвездии расстояние между звёздами Альфа и Бета на небесной сфере составляет $18^{\circ}$, а их звёздные величины равны $2.96^m$ и $3.07^m$ соответственно. Известно, что абсолютные звёздные величины этих звёзд одинаковы. Какую звёздную величину будет иметь звезда Альфа, если смотреть на неё из окрестностей звезды Бета?
        
        \item Телескопу доступны звёзды до $19^m$. Можно ли с его помощью зарегистрировать шаровое скопление из миллиона звезд, подобных Солнцу, находящееся в галактике на расстоянии $10$ Мпк от нас?
        \item Две звезды имеют одинаковые угловые диаметры, расстояние до них неизвестно. Их температуры различаются в $3$ раза. Найдите разницу болометрических звёздных величин звезд.
 \end{enumerate}
\end{document}