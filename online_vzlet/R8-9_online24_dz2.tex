\documentclass[12pt]{article}
\usepackage{baseset}
\usepackage{myproblem}
\usepackage{stackengine}
\DeclareSymbolFont{operators}{OT1}{ntxtlf}{m}{n}
\SetSymbolFont{operators}{bold}{OT1}{ntxtlf}{b}{n}
\usepackage{wasysym}
\newcommand{\RomanNumeralCaps}[1]
{\MakeUppercase{\romannumeral #1}}
\usepackage{tabularx}
\usepackage{enumitem}

\begin{document}
 \begin{tabularx}{\textwidth}{Xr}
  {\Large \textbf{Кульминации и звездное время}} & Дедлайн -- $16.11.2024$ \\
 \end{tabularx}
 \noindent\rule{\textwidth}{0.4pt}
 \begin{enumerate}
        \item Определите зенитные расстояния верхней и нижней кульминаций звезды $\beta$ UMa ($\delta=56^{\circ}17'$) в Москве ($\varphi$=$55^{\circ}45'$). Рефракцию не учитывать.
        \item Определите высоту нижней кульминации звезды, у которой в Долгопрудном ($\varphi$=$55^{\circ}56'$) кульминация происходит на высоте $71^{\circ}56'$. Рефракцией пренебречь. 
        \item Определите склонение звезды которая в Долгопрудном ($\varphi$=$55^{\circ}56'$) и Владивостоке ($\varphi=43^{\circ}11'$) кульминирует на одной и той же высоте.
        \item Звезда находится над горизонтом $11^h 58^m$ и кульминирует на высоте $40^{\circ}$. Определите широту места наблюдения и склонение звезды.
        \item Высота звезды в верхней кульминации $65^{\circ}$, а в нижней кульминации зенитное расстояние $63^{\circ}$. Определите склонение звезды и широту места наблюдения.
        \item $22$ сентября в некотором городе России Солнце взошло на $6$ часов $40$ минут раньше, чем в Твери ($36^{\circ}$ в.д.). Определите географическую долготу этого города.
        \item Определите разницу местного времени между Чебоксарами и Москвой. Расстояние между городами $600$~км. Широта Москвы $55^{\circ}45'$, широта Чебоксар $56^{\circ}08'$.
        \item Определите звездное время в моменты, когда звезда Альдебаран ($\alpha$ Tau, $\alpha=4^h 35^m 55^s$) имеет часовые углы $-4^h 23^m 15^s$ и $49^{\circ}24'15''$.
        \item Определить звездное время в пунктах с географической долготой $2^h 23^m 37^s$ и $7^h 6^m 20^s$ в момент, когда в пункте с географической долготой $80^{\circ} 05.5'$ у звезды Вега ($\alpha$ Лиры) часовой угол равен $4^h 29^m 48^s$. Прямое восхождение Веги $\alpha=18^h 35^m 15^s$. 
        \item Определите прямое восхождение звезды, которая в момент восхода Солнца $23$ сентября имеет часовой угол  ($t=1^h 45^m$). Чему равно звездное время в этот момент?
 \end{enumerate}
\end{document}