\documentclass[12pt]{article}
\usepackage{baseset}
\usepackage{myproblem}
\usepackage{stackengine}
\DeclareSymbolFont{operators}{OT1}{ntxtlf}{m}{n}
\SetSymbolFont{operators}{bold}{OT1}{ntxtlf}{b}{n}
\usepackage{wasysym}
\newcommand{\RomanNumeralCaps}[1]
{\MakeUppercase{\romannumeral #1}}
\usepackage{tabularx}
\usepackage{enumitem}
\usetikzlibrary{hobby}

\begin{document}
    \begin{tabularx}{\textwidth}{Xr}
    {\Large \textbf{Закон Хаббла}} & Дедлайн -- $14.12.2024$ \\
    \end{tabularx}
    \noindent\rule{\textwidth}{0.4pt}
    \begin{enumerate}
        \item Определите расстояние до Галактики, если она удаляется от нас со скоростью $3500$ км/с.
        \item В галактике, красное смещение линий в спектре которой соответствует скорости $3000$ км/с, вспыхнула сверхновая звезда. Ее яркость в максимуме была равна $15^m$. Определите абсолютную звездную величину и светимость?
        \item Спиральная галактика с красным смещением ($1/60$) видна на Земле как узкая полоска длиной $2$ угловые минуты. Лучевая скорость краевых областей галактики отличается от лучевой скорости ее центра на $60$ км/с. Оцените массу галактики.
        \item Галактика A имеет красное смещение $0.07$. Галактика B, расположенная на небе в $150$ градусах от галактики A, имеет красное смещение $0.09$. Какое красное смещение будет иметь галактика B для наблюдателя в галактике A? 
        \item У некоторой спиральной галактики линия $H_{\alpha}$ наблюдается на длине волны $7900$~\AA, причем ширина этой линии равна $16$~\AA. Оцените видимую звездную величину этой галактики.
        
        \textit{Подсказка:} Закон Талли-Фишера для спиральных галактик:
        $$
        L\sim v^4,
        $$
        где $L$ -- светимость галактики, $v$ -- скорость галактики на плато.
        \item Определите значение красного смещения для реликтового излучения в некоторый момент в будущем, если тогда максимум излучения в его спектре будет приходиться на длину волны $1.5$ мм, а испущено оно было тогда, когда температура вещества составляла $4000$ К.
    \end{enumerate}
\end{document}