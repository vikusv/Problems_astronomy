\documentclass[12pt]{article}
\usepackage{baseset}
\usepackage{myproblem}
\usepackage{stackengine}
\DeclareSymbolFont{operators}{OT1}{ntxtlf}{m}{n}
\SetSymbolFont{operators}{bold}{OT1}{ntxtlf}{b}{n}
\usepackage{wasysym}
\newcommand{\RomanNumeralCaps}[1]
{\MakeUppercase{\romannumeral #1}}
\usepackage{tabularx}
\usepackage{enumitem}

\begin{document}
\begin{tabularx}{\textwidth}{Xr}
{\Large \textbf{Движение Солнца. Звездное время}} & Дедлайн -- $29.11.2023$ \\
\end{tabularx}
\noindent\rule{\textwidth}{0.4pt}
\begin{enumerate}
    \item \textbf{!} Определите время, которое Солнце проводит над горизонтом $5$ ноября в Диксоне ($\varphi = 73^{\circ}30'$). А $15$ июля? Рассчитайте длительность полярного дня и полярной ночи в поселке. Определите дату, когда диск Солнца проводит под горизонтом минимальное время, большее нуля. Также определите это время.
    \item Во сколько раз и на сколько процентов отличаются длительности полярных дня и ночи на широте $\varphi = 85^{\circ}$? 
    \item \textbf{!} Рассчитайте звездное время в момент, когда в Москве отмечают Новый год. Уравнение времени в зависимости от прямого восхождения Солнца:
    $$
    7.3\cos{\alpha} + 1.5\sin{\alpha} - 9.87\sin{2\alpha}.
    $$
    \item Определите звездное время, если Солнце находится в нижней кульминации на высоте $-45^{\circ}$ в пункте с широтой $\varphi = 30^{\circ}$.
    \item Представьте, что вы очнулись после кораблекрушения и обнаружили, что попали на остров. Посмотрев на небо, вы увидели Ригель ($\alpha = 5^h15^m$, $\delta = -8^{\circ}11'$) на высоте $h = 52.5^{\circ}$, а азимут звезды равнялся $A = 109^{\circ}$. Часы, установленые на бангкокское время (UT+7), показывают ровно час ночи $21$ ноября $2017$ года. Найдите часовой угол Ригеля, текущее гринвичское звездное время и географические координаты вашего местоположения, если гринвичское звездное время на $0^h$ UT $1$ января было равно $GST_0 = 6^h43^m$.
    \item В момент захода Солнца азимут центра диска был равен $A = 98^{\circ}$, а модуль скорости изменения этой величины $b = 12.87'$/мин. В момент наблюдений среднее солнечное время опрережало истинное (можно пользоваться формулой из задачи $3$). Найдите дату наблюдений, рефракцией пренебрегите.
\end{enumerate}
\end{document}