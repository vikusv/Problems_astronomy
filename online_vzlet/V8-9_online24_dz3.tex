\documentclass[12pt]{article}
\usepackage{baseset}
\usepackage{myproblem}
\usepackage{stackengine}
\DeclareSymbolFont{operators}{OT1}{ntxtlf}{m}{n}
\SetSymbolFont{operators}{bold}{OT1}{ntxtlf}{b}{n}
\usepackage{wasysym}
\newcommand{\RomanNumeralCaps}[1]
{\MakeUppercase{\romannumeral #1}}
\usepackage{tabularx}
\usepackage{enumitem}

\begin{document}
 \begin{tabularx}{\textwidth}{Xr}
  {\Large \textbf{Поверхностная звездная величина}} & Дедлайн -- $2.11.2024$ \\
 \end{tabularx}
 \noindent\rule{\textwidth}{0.4pt}
 \begin{enumerate}
        \item Как изменилась бы поверхностная яркость туманности Андромеды, если она находилась бы ближе к нам, чем сейчас.
        \item Астроном-любитель навёл телескоп на туманность и увидел её в виде едва заметно светящегося маленького пятнышка. Для того чтобы, разглядеть его лучше, он вставил перед окуляром линзу Барлоу, которая в $3$ раза увеличила эффективное фокусное расстояние его телескопа. Смог ли астроном-любитель лучше разглядеть туманность?
        \item Видимая звёздная величина Венеры в наибольшей элонгации равна $-4.5^m$. Оцените поверхностную яркость Венеры в единицах «звездная величина с квадратной угловой секунды».
        \item Планетарная туманность «Кольцо» имеет видимый диаметр $2'$ и блеск $9^m$. Оцените, насколько светло будет ночью на планете, обращающейся вокруг звезды –
        ядра этой туманности. Сравните по освещенности ночное небо на этой планете с земным ночным небом.
        \item  В некотором городе в результате засветки неба уличным освещением предельная звездная величина звезд, видимых невооруженным глазом, оказалась равной $3^m$. Оцените поверхностную яркость неба (звездную величину, приходящуюся на квадратную угловую секунду небесной сферы) в этом городе.
        \item Эллиптическая галактика M$49$ имеет угловые размеры $10'\times8'$. Ее средняя поверхностная яркость равна $13^m$ с квадратной минуты. Расстояние до M$49$ равно $16$ Мпк. Определите абсолютную звездную величину галактики, пренебрегая поглощением света.
        \item Любитель астрономии решил сфотографировать различные объекты глубокого космоса со своего городского балкона. Для начала он сделал пробные снимки яркого объекта и снял галактику M$51$ («Водоворот», видимая звездная величина $8^m$, угловые размеры $13'\times12'$). В результате обработки снимков выяснилось, что для того, чтобы увидеть галактику на снимке, ему необходимо было сделать и сложить $20$ кадров. Какое минимальное количество кадров надо будет сделать при наблюдении водородной туманности NGC$7000$ («Северная Америка», видимая звездная величина $4^m$, угловые размеры $120'\times100'$), чтобы увидеть ее на снимке? Оба объекта снимались в одних и тех же условиях с одинаковыми параметрами камеры и полностью помещались на снимок.
 \end{enumerate}
\end{document}