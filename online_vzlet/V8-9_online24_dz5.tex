\documentclass[12pt]{article}
\usepackage{baseset}
\usepackage{myproblem}
\usepackage{stackengine}
\DeclareSymbolFont{operators}{OT1}{ntxtlf}{m}{n}
\SetSymbolFont{operators}{bold}{OT1}{ntxtlf}{b}{n}
\usepackage{wasysym}
\newcommand{\RomanNumeralCaps}[1]
{\MakeUppercase{\romannumeral #1}}
\usepackage{tabularx}
\usepackage{enumitem}

\begin{document}
 \begin{tabularx}{\textwidth}{Xr}
  {\Large \textbf{Поглощение}} & Дедлайн -- $16.11.2024$ \\
 \end{tabularx}
 \noindent\rule{\textwidth}{0.4pt}
 \begin{enumerate}
        \item Видимый поперечник звездного скопления составляет $13'$, видимая звездная величина $9^m$, диаметр скопления равен $6$ пк. Считая, что в скоплении содержится $1000$ звезд, похожих на Солнце, оцените поглощение света в звездных величинах на $1$~кпк в направлении на скопление.
        \item В Галактике Млечный Путь раз в $20$ лет вспыхивают Сверхновые II типа с абсолютной звездной величиной $-18^m$. Насколько часто такие Сверхновые появляются в небе Земли с блеском ярче Венеры ($-4^m$)? Радиус Галактики считать равным $15$ кпк, поглощение света составляет $2^m/\text{кпк}$.
        \item На какое расстояние надо удалиться от Солнца, чтобы его звездная величина стала равна $18^m$ без учета межзвездного поглощения? А с учетом межзвездного поглощения $2^m/\text{кпк}$? 
        \item Поглощение света атмосферой Земли при наблюдениях в зените составляет $0.23^m$. Оцените, каким будет поглощение при наблюдении на зенитном расстоянии $30^{\circ}$ и $85^{\circ}$. 
        \item Видимая звездная величина Проциона за пределами земной атмосферы составляет $0.40^m$. Найдите видимую звездную величину Проциона в момент верхней кульминации для наблюдателя в Петербурге ($\varphi=60^{\circ}$), если известно, что склонение Проциона $\delta = +5^{\circ}$
        \item Наблюдатель на Земле исследует некоторую звезду. При наблюдении её в зените её звёздная величина оказалась равной $m_1 = 2.74^m$, а при высоте $45^{\circ}$ над горизонтом $m_2 = 2.85^m$. Чему равна звёздная величина $m_0$ этой звезды при наблюдении вне атмосферы (со спутника, например)?
        \item В период задымления от лесных пожаров в центральной России летом $2010$ года наблюдатель заметил, что Солнце на высоте $20^{\circ}$ над горизонтом имело ту же видимую яркость, какая бывает у полной Луны вблизи зенита на ясном небе при чистой атмосфере. Исходя из этого, оцените суммарную массу дымовых частиц, находившихся над одним квадратным метром земной поверхности в этих районах. Считать, что поглощение света в чистой атмосфере в зените равно $0.2^m$, а дым состоит из черных частиц размером $1$ мкм и плотностью $0.6 \mbox{г/см}^3$. Также считать, что поглощение света соответствует законам геометрической оптики (дифракцией на частицах пренебречь). 
 \end{enumerate}
\end{document}