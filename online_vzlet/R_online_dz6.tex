\documentclass[12pt]{article}
\usepackage{baseset}
\usepackage{myproblem}
\usepackage{stackengine}
\DeclareSymbolFont{operators}{OT1}{ntxtlf}{m}{n}
\SetSymbolFont{operators}{bold}{OT1}{ntxtlf}{b}{n}
\usepackage{wasysym}
\newcommand{\RomanNumeralCaps}[1]
{\MakeUppercase{\romannumeral #1}}
\usepackage{tabularx}
\usepackage{enumitem}

\begin{document}
 \begin{tabularx}{\textwidth}{Xr}
  {\Large \textbf{Собственное движение звезд}} & Дедлайн -- $17.12.2023$ \\
 \end{tabularx}
 \noindent\rule{\textwidth}{0.4pt}
 \begin{enumerate}
        \item Звезда имеет лучевую скорость $v_r=+18$~км/с и трансверсальную скорость $v_{\tau}=8$~км/с. Найдите пространственную скорость звезды.
        \item Собственное движение звезды равно $\mu=0.2''$/год. Расстояние до звезды $20$ пк. Определите трансверсальную скорость звезды.
        \item Полная пространственная скорость звезд Канопус $23$ км/с образует с лучом зрения угол $37^{\circ}$. Определите лучевую и трансверсальную составляющую скорости. Определите собственное движение звезды по небу, если расстояние до Канопуса  $96$ пк.
        \item Звезда движется относительно Солнца под углом $45^{\circ}$ к лучу зрения. При этом ее гелиоцентрическая лучевая скорость равна $20$ км/с, а собственное движение -- $0.10''$ в год. Чему равен тригонометрический параллакс звезды?
        \item Звезда Вега имеет собственное движение $0.35''$ в год, параллакс $0.129''$ и лучевую скорость $-14$ км/c. Через сколько лет Вега окажется к нам вдвое ближе, чем сейчас?
        \item У Альтаира годичный параллакс равен $0.198''$, собственное движение $0.658''$/год, лучевая скорость $V_r =- 26$ км/с и блеск $0.77^m$. Когда и на какое наименьшее расстояние Альтаир сблизится с Солнцем, и каким будет тогда его видимый блеск?
        \item Координаты звезды $(0^h, +60^{\circ})$, лучевая скорость $V_r=-20$ км/с, собственное движение звезды $5''$/год, направлено в сторону увеличения склонения, параллакс $0.1''$. Необходимо найти координаты через $260~000$ лет.
        \item Звезда Вега имеет видимую звездную величину $0.03^m$, годичный параллакс $0.13''$ лучевую скорость $-14$~км/с и собственное движение $0.35''$/год. Звезда Арктур имеет видимую звездную величину $-0.05^m$, годичный параллакс $0.089''$, лучевая скорость $-5.3$~км/с, а собственное движение $2.3''$/год. Станет ли когда нибудь Вега ярче, чем Арктур?  Если станет, то когда? Светимость звезд считать постоянным во времени, межзвездным поглощением пренебречь.
        \item Звезда Каптейна ($VZ$ Живописца, $\alpha=5^h 11^m$, $\delta=-45^{\circ} 01'$) обладает одним из самых больших собственных движений: $\mu_{\alpha}=6.5''$/год, $\mu_{\delta}=-5.7''$/год. Лучевая скорость звезды также достаточно велика $245$ км/с, и звезда движется в другую сторону относительно центра галактики. Определите, через сколько лет звезда сместится на $10^{\circ}$ от текущего положения и на каком расстоянии она будет от Земли, если сейчас параллакс равен $0.255''$.
        \item Некоторая звезда обладает видимой звездной величиной $7^m$ и ненулевым собственным движением. Какова будет ее видимая звездная величина в тот момент, когда собственное движение звезды уменьшится в $4$ раза? Полная скорость звезды остается постоянной.
 \end{enumerate}
\end{document}