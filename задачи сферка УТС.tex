\documentclass[14pt]{extarticle}

\input{settings_valpha}

\newcommand{\tasksetname}{Небесная сфера}
\newcommand{\eventname}{Подготовка к УТС}
\newcommand{\eventdate}{Весна 2022}


\newcommand{\titleofbase}{\header{Базовые задачи}}
\newcommand{\titleoflts}{\header{Задачи с УТС}}
\setauthor{Муратов Василий}



\begin{document}


\pageheader
\titleofbase
\shortproblem{5 баллов} Найдите эклиптические координаты звезды Барнарда ($17^h 57^m$, $4^\circ 41'$).

\shortproblem{5 баллов} Найдите координаты полюсов большого круга, на котором лежат Антарес ($16^h 30^m$, $-26^\circ 25'$) и Бенетнаш ($13^h 47^m$, $49^\circ 19'$).

\shortproblem{10 баллов} Как с помощью векторов можно найти координаты полюса и радиус малого круга, на котором лежат три заданные точки? ОЧЕНЬ важная задача!

\shortproblem{5 баллов} Используя результат, полученный ранее, найдите координаты точек малого круга, проходящего через Вегу, Денеб и Альтаир. Найдите его радиус. Координаты надо знать.

\shortproblem{5 баллов} Пусть в день рождения Васи (19 апреля) высота Проциона ($7^h 39^m$, $5^\circ 13'$) в Санкт-Петербурге составляет $20^\circ$. Какое время показывают часы жителей СПБ?

\shortproblem{5 баллов} Самолет летит из Лимы ($\varphi =-12^\circ$, $\lambda=-77^\circ$) в Нджамену ($\varphi =12^\circ$, $\lambda=15^\circ$) по самому короткому маршруту из возможных. Под каким углом он пересечет экватор?

\shortproblem{5 баллов} Найдите галактические координаты звезды Сердце Карла ($12^h 56^m$, $38^\circ 19'$). СПГ имеет координаты $\alpha = 12^h 49^m$, $\delta = 27^\circ 24'$.

\shortproblem{5 баллов} Посчитайте координаты Веги через 8000 лет (эклиптические и экваториальные).

\titleoflts

\problem{Бешеные псы}{15 баллов}
Астеризм созвездия Гончие Псы образован
двумя относительно яркими звёздами: $\alpha$ CVn ($12^h 56^m$, $38^\circ 19'$)
и $\beta$ CVn ($12^h 34^m$, $41^\circ 21'$). Какова
минимальная и максимальная возможная наблюдаемая продолжительность восхода этого астеризма («отрезка»)? На каких широтах
она достигается?
\problemsource{Лето 2021}

\problem{Белая ночь опустилась как облако}{15 баллов}
В какие дни года ночь в Москве длится дольше дня? В какие дни года центр Солнца пересекает горизонт в Москве и СПБ одновременно?
\problemsource{Осень 2020}

\problem{Собака}{15 баллов}
На каких широтах возможно наблюдать одновременный восход Веги и Сириуса? Их координаты надо знать.
\problemsource{Осень 2020}

\problem{Ковальски, анализ}{15 баллов}
В некоторый день Солнце взошло в Антананариву ($\lambda = 47^\circ 32'$, UT+3)
	в 05:36, а в Москве --- в 06:23.
	Из этих данных определите возможные даты наблюдений и широту
	Антананариву. Уравнением времени и понижением горизонта пренебрегите.
\problemsource{Весна 2021}

\problem{Задача от африканских негров}{15 баллов}
Замбези течёт по широкой низменности почти по прямой
	между городами Тете и Чинде, где впадает в Мозамбикский
	пролив. Можно ли увидеть отражение восходящего (пересекающего линию горизонта) Солнца в водах реки Замбези
	с вершины горы Гогого (1867 м)? Если да, то в какие дни
	года это возможно? Поверхность воды считайте невозмущённой, особенностями рельефа окружающей Гогого
	местности пренебрегите. 
\problemsource{Осень 2019}
    \begin{center}
		\begin{tabular}{ccc}
			Гора & $\varphi$ & $\lambda$ \\
			\hline
			Тете & $-16.16^\circ$ & $33.59^\circ$ \\
			Чинде & $-18.7^\circ$ & $36.39^\circ$ \\
			Гогого & $-18.4^\circ$ & $34.10^\circ$ \\ 

		\end{tabular}
	\end{center}


\problem{МАТАН}{15 баллов}
У звезды склонение $60^\circ$, а прямое восхождение в два раза больше. В какое время суток в Москве ее высота и азимут меняются быстрее всего?
\problemsource{Вася}

\problem{Огромная черная площадка}{15 баллов}
Оцените, сколько энергии от Солнца поглотит абсолютно чёрная горизонтальная площадка
	площадью $1$ квадратный метр.
	за один световой день $21$ апреля на широте $60^\circ$. Солнце считать точечным
	источником. Эффекты, связанные с наличием у Земли атмосферы, не учитывать.
\problemsource{Лето 2019}

\problem{Опять про псов}{15 баллов}
Найдите азимут восхода звезды Адара ($6^h 59^m$, $-29^\circ$) в самой северной точке, равноудаленной от Санкт-Петербурга и Москвы.
\problemsource{Осень 2019}

\problem{Ня кавай}{15 баллов}
Какое наибольшее и наименьшее расстояние может быть от анимешника в Токио ($36^\circ$) до эклиптики?
\problemsource{Весна 2021}

\problem{Бро, тебе надо тренироваться}{ 15 баллов}
Некий сигнал был принят одновременно в $3$ часа ночи по всемирному времени 1 февраля в точках Земли с координатами из таблицы. Найдите экваториальные координаты источника на небе.
\problemsource{Вася}
\begin{center}
    \begin{tabular}{ccc}
        Пункт & $\varphi$ & $\lambda$ \\
        \hline
        Mirage & $40^\circ$ & $-60^\circ$ \\
        Vertigo & $60^\circ$ & $-40^\circ$ \\
        Inferno & $29^\circ$ & $10^\circ$ \\ 

    \end{tabular}
\end{center}

\problem{Ористотиль}{15 баллов}
Оцените, сколько звёзд, которые мог видеть Аристотель (около $500$ лет
	до н. э.) можно увидеть невооружённым глазом в наши дни, находясь
	в пригороде Афин ($\varphi =38^\circ$). Засветкой неба, рельефом и рефракцией
	пренебрегите.
\problemsource{Осень 2021}

\problem{Глаз павука}{15 баллов}
В каком году Антарес будет в ВК в Москве на высоте 45 градусов? Каким будет звездное время в этот момент?
\problemsource{Вася}

\problem{ААААА}{15 баллов}
«Сегодня 24 сентября, самый длинный день в году. Он вдвое длиннее, чем 25 марта.»
	В какой ближайшей к нам точке пространства (где на Земле?) и времени
	(когда?) может находиться наблюдатель? Уравнением времени, рефракцией и угловыми размерами
	Солнца пренебречь.
\problemsource{Осень 2016}


\end{document}